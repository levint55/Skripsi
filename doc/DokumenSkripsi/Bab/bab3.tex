\chapter{Analisis}
\label{chap:analisis}

\section{Tingkat Kepatuhan \textit{SharIF Judge}}
\label{sec:kepatuhan_sharif_judge_terhadap_wcag_2.1}
Kesuksesan aplikasi SharIF Judge dalam mematuhi kriteria sukses \textit{WCAG} 2.1 ditulis dalam tabel-tabel berikut.

\begin{table}[H]
	\centering
	\caption{Tabel kepatuhan \textit{SharIF Judge} terhadap prinsip \textit{Perceivable}}
	\label{tab:kepatuhan_sharif_judge_perceivable}
	\begin{tabular}{|c|c|c|}
		\hline
		Kriteria Sukses & Tingkat Kepatuhan & Hasil \\
		\hline
		1.1.1 & A & TODO \\
		1.2.1 & A & TODO \\
		1.2.2 & A & TODO \\
		1.2.3 & A & TODO \\
		1.2.4 & AA & TODO \\
		1.2.5 & AA & TODO \\
		1.2.6 & AAA & TODO \\
		1.2.7 & AAA & TODO \\
		1.2.8 & AAA & TODO \\
		1.2.9 & AAA & TODO \\
		1.3.1 & A & TODO \\
		1.3.2 & A & TODO \\
		1.3.3 & A & TODO \\
		1.3.4 & A & TODO \\
		1.3.5 & AA & TODO \\
		1.3.6 & AAA & TODO \\
		1.4.1 & A & \\
		1.4.2 & A & \\
		1.4.3 & AA & \\
		1.4.4 & AA & \\
		1.4.5 & AA & \\
		1.4.6 & AAA & \\
		1.4.7 & AAA & \\
		1.4.8 & AAA & \\
		1.4.9 & AAA & \\
		1.4.10 & AA & \\
		1.4.11 & AA & \\
		1.4.12 & AA & \\
		1.4.13 & AA & \\
		\hline
	\end{tabular}
\end{table}

\begin{table}[H]
	\centering
	\caption{Tabel kepatuhan \textit{SharIF Judge} terhadap prinsip \textit{Operable}}
	\label{tab:kepatuhan_sharif_judge_operable}
	\begin{tabular}{|c|c|c|}
		\hline
		Kriteria Sukses & Tingkat Kepatuhan & Hasil \\
		\hline
		2.1.1 & A & \\
		2.1.2 & A & \\
		2.1.3 & AAA & \\
		2.1.4 & A & \\
		2.2.1 & A & \\
		2.2.2 & A & \\
		2.2.3 & AAA & \\
		2.2.4 & AAA & \\
		2.2.5 & AAA & \\
		2.2.6 & AAA & \\
		2.3.1 & A & \\
		2.3.2 & AAA & \\
		2.3.3 & AAA & \\
		2.4.1 & A & \\
		2.4.2 & A & \\
		2.4.3 & A & \\
		2.4.4 & A & \\
		2.4.5 & AA & \\
		2.4.6 & AA & \\
		2.4.7 & AA & \\
		2.4.8 & AAA & \\
		2.4.9 & AAA & \\
		2.4.10 & AAA & \\
		2.5.1 & A & \\
		2.5.2 & A & \\
		2.5.3 & A & \\
		2.5.4 & A & \\
		2.5.5 & AAA & \\
		2.5.6 & AAA & \\
		\hline
	\end{tabular}
\end{table}

\begin{table}[H]
	\centering
	\caption{Tabel kepatuhan \textit{SharIF Judge} terhadap prinsip \textit{Understandable}}
	\label{tab:kepatuhan_sharif_judge_understandable}
	\begin{tabular}{|c|c|c|}
		\hline
		Kriteria Sukses & Tingkat Kepatuhan & Hasil \\
		\hline
		3.1.1 & A & \\
		3.1.2 & AA & \\
		3.1.3 & AAA & \\
		3.1.4 & AAA & \\
		3.1.5 & AAA & \\
		3.1.6 & AAA & \\
		3.2.1 & A & \\
		3.2.2 & A & \\
		3.2.3 & AA & \\
		3.2.4 & AA & \\
		3.2.5 & AAA & \\
		3.3.1 & A & \\
		3.3.2 & A & \\
		3.3.3 & AA & \\
		3.3.4 & AA & \\
		3.3.5 & AAA & \\
		3.3.6 & AAA & \\
		\hline
	\end{tabular}
\end{table}

\begin{table}[H]
	\centering
	\caption{Tabel kepatuhan \textit{SharIF Judge} terhadap prinsip \textit{Robust}}
	\label{tab:kepatuhan_sharif_judge_robust}
	\begin{tabular}{|c|c|c|}
		\hline
		Kriteria Sukses & Tingkat Kepatuhan & Hasil \\
		\hline
		4.1.1 & A & \\
		4.1.2 & A & \\
		4.1.3 & AA & \\
		\hline
	\end{tabular}
\end{table}

\subsection{\textit{Perceivable}}
\label{subsec:perceivable}

\subsubsection{Kriteria Sukses 1.1.1 Non-text Content}
\label{subsubsec:kriteria_1.1.1}

\subsubsection{Kriteria Sukses 1.2.1 Audio-only dan Video-only (Prerecorded)}
\label{subsubsec:kriteria_1.2.1}
(Sukses) \\

Kriteria ini sukses dipatuhi karena pada aplikasi \textit{SharIF Judge} tidak terdapat konten berbasis waktu.
