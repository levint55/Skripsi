\chapter{Analisis}
\label{chap:analisis}

\section{Tingkat Kepatuhan \textit{SharIF Judge}}
\label{sec:kepatuhan_sharif_judge_terhadap_wcag_2.1}
Kesuksesan aplikasi SharIF Judge dalam mematuhi kriteria sukses \textit{WCAG} 2.1 ditulis dalam tabel-tabel berikut.

\begin{table}[H]
	\centering
	\caption{Tabel kepatuhan \textit{SharIF Judge} terhadap prinsip \textit{Perceivable}}
	\label{tab:kepatuhan_sharif_judge_perceivable}
	\begin{tabular}{|c|c|c|}
		\hline
		Kriteria Sukses & Tingkat Kepatuhan & Hasil \\
		\hline
		1.1.1 & A & Tidak Sukses \\
		1.2.1 & A & Sukses \\
		1.2.2 & A & Sukses \\
		1.2.3 & A & Sukses \\
		1.2.4 & AA & Sukses \\
		1.2.5 & AA & Sukses \\
		1.2.6 & AAA & Sukses \\
		1.2.7 & AAA & Sukses \\
		1.2.8 & AAA & Sukses \\
		1.2.9 & AAA & Sukses \\
		1.3.1 & A & Tidak Sukses\\
		1.3.2 & A & Sukses \\
		1.3.3 & A &  \\
		1.3.4 & A & Sukses \\
		1.3.5 & AA &  \\
		1.3.6 & AAA &  \\
		1.4.1 & A & Sukses \\
		1.4.2 & A & Sukses \\
		1.4.3 & AA &  \\
		1.4.4 & AA & Tidak Sukses \\
		1.4.5 & AA & Sukses \\
		1.4.6 & AAA &  \\
		1.4.7 & AAA & Sukses \\
		1.4.8 & AAA &  \\
		1.4.9 & AAA & Sukses \\
		1.4.10 & AA &  \\
		1.4.11 & AA & \\
		1.4.12 & AA & \\
		1.4.13 & AA & Sukses \\
		\hline
	\end{tabular}
\end{table}

\begin{table}[H]
	\centering
	\caption{Tabel kepatuhan \textit{SharIF Judge} terhadap prinsip \textit{Operable}}
	\label{tab:kepatuhan_sharif_judge_operable}
	\begin{tabular}{|c|c|c|}
		\hline
		Kriteria Sukses & Tingkat Kepatuhan & Hasil \\
		\hline
		2.1.1 & A & Tidak Sukses \\
		2.1.2 & A & Tidak Sukses \\
		2.1.3 & AAA & Tidak Sukses \\
		2.1.4 & A & Sukses \\
		2.2.1 & A & Sukses \\
		2.2.2 & A & Sukses \\
		2.2.3 & AAA & \\
		2.2.4 & AAA & \\
		2.2.5 & AAA & \\
		2.2.6 & AAA & \\
		2.3.1 & A & Sukses\\
		2.3.2 & AAA & Sukses\\
		2.3.3 & AAA & Sukses\\
		2.4.1 & A & Tidak Sukses\\
		2.4.2 & A & Sukses \\
		2.4.3 & A & Sukses \\
		2.4.4 & A & Tidak Sukses \\
		2.4.5 & AA & \\
		2.4.6 & AA & \\
		2.4.7 & AA & Tidak Sukses\\
		2.4.8 & AAA & Sukses\\
		2.4.9 & AAA & \\
		2.4.10 & AAA & \\
		2.5.1 & A & \\
		2.5.2 & A & \\
		2.5.3 & A & \\
		2.5.4 & A & \\
		2.5.5 & AAA & \\
		2.5.6 & AAA & \\
		\hline
	\end{tabular}
\end{table}

\begin{table}[H]
	\centering
	\caption{Tabel kepatuhan \textit{SharIF Judge} terhadap prinsip \textit{Understandable}}
	\label{tab:kepatuhan_sharif_judge_understandable}
	\begin{tabular}{|c|c|c|}
		\hline
		Kriteria Sukses & Tingkat Kepatuhan & Hasil \\
		\hline
		3.1.1 & A & Tidak Sukses \\
		3.1.2 & AA & \\
		3.1.3 & AAA & \\
		3.1.4 & AAA & \\
		3.1.5 & AAA & \\
		3.1.6 & AAA & \\
		3.2.1 & A & \\
		3.2.2 & A & \\
		3.2.3 & AA & \\
		3.2.4 & AA & \\
		3.2.5 & AAA & \\
		3.3.1 & A & \\
		3.3.2 & A & \\
		3.3.3 & AA & \\
		3.3.4 & AA & \\
		3.3.5 & AAA & \\
		3.3.6 & AAA & \\
		\hline
	\end{tabular}
\end{table}

\begin{table}[H]
	\centering
	\caption{Tabel kepatuhan \textit{SharIF Judge} terhadap prinsip \textit{Robust}}
	\label{tab:kepatuhan_sharif_judge_robust}
	\begin{tabular}{|c|c|c|}
		\hline
		Kriteria Sukses & Tingkat Kepatuhan & Hasil \\
		\hline
		4.1.1 & A & \\
		4.1.2 & A & Tidak Sukses\\
		4.1.3 & AA & \\
		\hline
	\end{tabular}
\end{table}

\subsection{\textit{Perceivable}}
\label{subsec:kepatuhan_perceivable}

\subsubsection{Kriteria Sukses 1.1.1 Non-text Content}
\label{subsubsec:kepatuhan_kriteria_1.1.1}
(Tidak Sukses) \\

Kriteria ini tidak sukses dipatuhi karena :
\begin{itemize}
	\item Gambar logo \textit{SharIF Judge} tidak memiliki alternatif teks.
	\item Gambar pdf pada halaman \textit{Assignment} tidak memiliki teks alternatif.
\end{itemize}

\subsubsection{Kriteria Sukses 1.2.1 Audio-only dan Video-only (Prerecorded)}
\label{subsubsec:kepatuhan_kriteria_1.2.1}
(Sukses) \\

Kriteria ini sukses dipatuhi karena pada aplikasi \textit{SharIF Judge} tidak terdapat konten berbasis waktu.

\subsubsection{Kriteria Sukses 1.2.2 Captions (Prerecorded)}
\label{subsubsec:kepatuhan_kriteria_1.2.2}
(Sukses) \\

Kriteria ini sukses dipatuhi karena pada aplikasi \textit{SharIF Judge} tidak terdapat konten berbasis waktu.

\subsubsection{Kriteria Sukses 1.2.3 Audio Descriptive atau Media Alternative (Prerecorded)}
\label{subsubsec:kepatuhan_kriteria_1.2.3}
(Sukses) \\

Kriteria ini sukses dipatuhi karena pada aplikasi \textit{SharIF Judge} tidak terdapat konten berbasis waktu.

\subsubsection{Kriteria Sukses 1.2.4 Captions (Live)}
\label{subsubsec:kepatuhan_kriteria_1.2.4}
(Sukses) \\

Kriteria ini sukses dipatuhi karena pada aplikasi \textit{SharIF Judge} tidak terdapat konten berbasis waktu.

\subsubsection{Kriteria Sukses 1.2.5 Audio Description (Prerecorded)}
\label{subsubsec:kepatuhan_kriteria_1.2.5}
(Sukses) \\

Kriteria ini sukses dipatuhi karena pada aplikasi \textit{SharIF Judge} tidak terdapat konten berbasis waktu.

\subsubsection{Kriteria Sukses 1.2.6 Sign Language (Prerecorded)}
\label{subsubsec:kepatuhan_kriteria_1.2.6}
(Sukses) \\

Kriteria ini sukses dipatuhi karena pada aplikasi \textit{SharIF Judge} tidak terdapat konten berbasis waktu.

\subsubsection{Kriteria Sukses 1.2.7 Extended Audio Description (Prerecorded)}
\label{subsubsec:kepatuhan_kriteria_1.2.7}
(Sukses) \\

Kriteria ini sukses dipatuhi karena pada aplikasi \textit{SharIF Judge} tidak terdapat konten berbasis waktu.

\subsubsection{Kriteria Sukses 1.2.8 Media Alternative (Prerecorded)}
\label{subsubsec:kepatuhan_kriteria_1.2.8}
(Sukses) \\

Kriteria ini sukses dipatuhi karena pada aplikasi \textit{SharIF Judge} tidak terdapat konten berbasis waktu.

\subsubsection{Kriteria Sukses 1.2.9 Audio-only (Live)}
\label{subsubsec:kepatuhan_kriteria_1.2.9}
(Sukses) \\

Kriteria ini sukses dipatuhi karena pada aplikasi \textit{SharIF Judge} tidak terdapat konten berbasis waktu.

\subsubsection{Kriteria Sukses 1.3.1 Info dan Relationships}
\label{subsubsec:kepatuhan_kriteria_1.3.1}
(Tidak Sukses) \\

Kriteria ini tidak sukses dipatuhi karena :
\begin{itemize}
	\item Ada beberapa elemen dalam form \textit{Add Users} yang tidak diberi label.
	\item Dalam halaman \textit{Add Assignment} ada elemen dalam form \textit{problems} tidak diberi label.
	\item Dalam halaman \textit{Problem} terdapat elemen form yang tidak diberi label.
\end{itemize}

\subsubsection{Kriteria Sukses 1.3.2 Meaningful Sequence}
\label{subsubsec:kepatuhan_kriteria_1.3.2}
(Sukses) \\

Kriteria ini sukses dipatuhi karena setiap halaman pada \textit{SharIF Judge} memiliki urutan baca dan navigasi yang benar.

\subsubsection{Kriteria Sukses 1.3.3 Sensory Characteristics}
\label{subsubsec:kepatuhan_kriteria_1.3.3}

Tanya : Apakah icon yang dihover memunculkan informasinya dianggap sukses ?

\subsubsection{Kriteria Sukses 1.3.4 Orientation}
\label{subsubsec:kepatuhan_kriteria_1.3.4}
(Sukses) \\

Kriteria ini sukses dipatuhi karena konten dari \textit{SharIF Judge} dapat ditampilkan dalam orientasi \textit{portrait} atau \textit{landscape}.

\subsubsection{Kriteria Sukses 1.3.5 Identify Input Purpose}
\label{subsubsec:kepatuhan_kriteria_1.3.5}

Tanya : Maksudnya input field sudah memiliki label yang menjelaskan tujuannya ?

\subsubsection{Kriteria Sukses 1.3.6 Identify Purpose}
\label{subsubsec:kepatuhan_kriteria_1.3.6}

Tanya : Maksudnya input field sudah memiliki label yang menjelaskan tujuannya ?

\subsubsection{Kriteria Sukses 1.4.1 Use of Color}
\label{subsubsec:kepatuhan_kriteria_1.4.1}
(Sukses) \\

Kriteria ini sukses dipatuhi karena warna tidak digunakan sebagai satu-satunya cara untuk menyampaikan informasi, menunjukkan aksi, menampilkan respon, atau membedakan elemen visual. Selain warna, ada juga teks yang menjelaskan informasi yang ditampilkan.

\subsubsection{Kriteria Sukses 1.4.2 Audio Control}
\label{subsubsec:kepatuhan_kriteria_1.4.2}
(Sukses) \\

Kriteria ini sukses dipatuhi karena pada aplikasi \textit{SharIF Judge} tidak terdapat konten berbasis waktu.

\subsubsection{Kriteria Sukses 1.4.3 Contrast (Minimum)}
\label{subsubsec:kepatuhan_kriteria_1.4.3}

Tanya : Bagaimana cara mengujinya ?

\subsubsection{Kriteria Sukses 1.4.4 Resize text}
\label{subsubsec:kepatuhan_kriteria_1.4.4}
(Tidak Sukses) \\

Kriteria ini tidak sukses dipatuhi karena fungsionalitas \textit{navigation bar} tidak berjalan pada pembesaran 125\%.

\subsubsection{Kriteria Sukses 1.4.5 Images of Text}
\label{subsubsec:kepatuhan_kriteria_1.4.5}
(Sukses) \\

Kriteria ini sukses dipatuhi karena semua gambar teks yang ada pada aplikasi \textit{SharIF Judge} penting untuk informasi yang disampaikan.

\subsubsection{Kriteria Sukses 1.4.6 Contrast (Enhanced)}
\label{subsubsec:kepatuhan_kriteria_1.4.6}

Tanya : Bagaimana cara mengujinya ?

\subsubsection{Kriteria Sukses 1.4.7 Low atau No Background Audio}
\label{subsubsec:kepatuhan_kriteria_1.4.7}
(Sukses) \\

Kriteria ini sukses dipatuhi karena pada aplikasi \textit{SharIF Judge} tidak terdapat konten berbasis waktu.


\subsubsection{Kriteria Sukses 1.4.8 Visual Presentation}
\label{subsubsec:kepatuhan_kriteria_1.4.8}

Tanya : Bagaimana cara mengujinya ?

\subsubsection{Kriteria Sukses 1.4.9 Images of Text (No Exception)}
\label{subsubsec:kepatuhan_kriteria_1.4.9}
(Sukses) \\

Kriteria ini sukses dipatuhi karena semua gambar teks yang ada pada aplikasi \textit{SharIF Judge} penting untuk informasi yang disampaikan.

\subsubsection{Kriteria Sukses 1.4.10 Reflow}
\label{subsubsec:kepatuhan_kriteria_1.4.10}

Tanya : Bagaimana cara mengujinya ?

\subsubsection{Kriteria Sukses 1.4.11 Non-text Contrast}
\label{subsubsec:kepatuhan_kriteria_1.4.11}

\subsubsection{Kriteria Sukses 1.4.12 Text Spacing}
\label{subsubsec:kepatuhan_kriteria_1.4.12}

\subsubsection{Kriteria Sukses 1.4.13 Content on Hover or Focus}
\label{subsubsec:kepatuhan_kriteria_1.4.13}
(Sukses) \\

Kriteria ini sukses dipatuhi karena setiap konten tambahan yang muncul sesaat ketika suatu elemen menerima penunjuk kursor atau fokus \textit{keyboard}, konten tambahan tersebut dapat disingkirkan, dapat ditunjuk, dan persisten.

\subsection{\textit{Operable}}
\label{subsec:kepatuhan_operable}

\subsubsection{Kriteria Sukses 2.1.1 Keyboard}
\label{subsubsec:kepatuhan_kriteria_2.1.1}
(Tidak Sukses) \\

Kriteria ini tidak sukses dipatuhi karena :
\begin{itemize}
	\item Tombol \textit{Collapse Menu} pada \textit{sidebar} tidak dapat dioperasikan dengan keyboard.
	\item Tombol \textit{Tools} pada \textit{Menu} tidak dapat dioperasikan dengan keyboard.
\end{itemize}

\subsubsection{Kriteria Sukses 2.1.2 No Keyboard Trap}
\label{subsubsec:kepatuhan_kriteria_2.1.2}
(Tidak Sukses) \\

Kriteria ini tidak sukses dipatuhi karena :
\begin{itemize}
	\item Dalam halaman \textit{Settings} ada \textit{input field} yang fokusnya tidak dapat dipindahkan menggunakan antarmuka \textit{keyboard}.
	\item Dalam halaman \textit{Add User} ada \textit{input field} yang fokusnya tidak dapat dipindahkan menggunakan antarmuka \textit{keyboard}.
	\item Dalam halaman \textit{Add Assignment} ada \textit{input field} yang fokusnya tidak dapat dipindahkan menggunakan antarmuka \textit{keyboard}.
	\item Dalam halaman \textit{Problems}(\textit{Edit Markdown}) ada \textit{input field} yang fokusnya tidak dapat dipindahkan menggunakan antarmuka \textit{keyboard}.
\end{itemize}

\subsubsection{Kriteria Sukses 2.1.3 Keyboard (No Exception)}
\label{subsubsec:kepatuhan_kriteria_2.1.3}
(Tidak Sukses) \\

Kriteria ini tidak sukses dipatuhi karena :
\begin{itemize}
	\item Tombol \textit{Collapse Menu} pada \textit{sidebar} tidak dapat dioperasikan dengan keyboard.
	\item Tombol \textit{Tools} pada \textit{Menu} tidak dapat dioperasikan dengan keyboard.
\end{itemize}

\subsubsection{Kriteria Sukses 2.1.4 Character Key Shortcuts}
\label{subsubsec:kepatuhan_kriteria_2.1.4}
(Sukses) \\

Kriteria ini sukses dipatuhi karena pada aplikasi \textit{SharIF Judge} tidak terdapat pintasan \textit{keyboard} untuk konten yang ditampilkan.

\subsubsection{Kriteria Sukses 2.2.1 Timing Adjustable}
\label{subsubsec:kepatuhan_kriteria_2.2.1}
(Sukses) \\

Kriteria ini sukses dipatuhi karena pada aplikasi \textit{SharIF Judge} batas waktu untuk mengumpulkan \textit{Assignment} dapat diatur setidaknya sepuluh kali lebih panjang dari setelan standar.

\subsubsection{Kriteria Sukses 2.2.2 Pause, Stop, Hide}
\label{subsubsec:kepatuhan_kriteria_2.2.2}
(Sukses) \\

Kriteria ini sukses dipatuhi karena satu-satunya informasi yang berkedip pada aplikasi \textit{SharIF Judge} adalah batas waktu pengumpulan \textit{Assignment} yang fungsinya penting.

\subsubsection{Kriteria Sukses 2.2.3 No Timing}
\label{subsubsec:kepatuhan_kriteria_2.2.3}

Tanya : Apakah timing mengumpulkan assignment termasuk real-time event ?

\subsubsection{Kriteria Sukses 2.2.4 Interruptions}
\label{subsubsec:kepatuhan_kriteria_2.2.4}

Tanya : Apa interupsi di Sharif Judge ?

\subsubsection{Kriteria Sukses 2.2.5 Re-authenticating}
\label{subsubsec:kepatuhan_kriteria_2.2.5}

Tanya : Bagaimana contoh skenarionya ?

\subsubsection{Kriteria Sukses 2.2.6 Timeouts}
\label{subsubsec:kepatuhan_kriteria_2.2.6}

\subsubsection{Kriteria Sukses 2.3.1 Three Flashes or Below Threshold}
\label{subsubsec:kepatuhan_kriteria_2.3.1}
(Sukses)\\

Kriteria ini sukses dipatuhi karena pada aplikasi \textit{SharIF Judge} tidak ada konten yang berkedip lebih dari tiga detik dalam periode satu detik.

\subsubsection{Kriteria Sukses 2.3.2 Three Flashes}
\label{subsubsec:kepatuhan_kriteria_2.3.2}
(Sukses) \\

Kriteria ini sukses dipatuhi karena pada aplikasi \textit{SharIF Judge} tidak ada konten yang berkedip lebih dari tiga detik dalam periode satu detik.

\subsubsection{Kriteria Sukses 2.3.3 Animation from Interactions}
\label{subsubsec:kepatuhan_kriteria_2.3.3}
(Sukses) \\

Tanya : Apakah pergerakan yang muncul ketika hide/show sidebar termasuk animasi ?

Kriteria ini sukses dipatuhi karena :
\begin{itemize}
	\item Animasi yang terdapat pada batas waktu penting untuk informasi yang disampaikan.
\end{itemize}

\subsubsection{Kriteria Sukses 2.4.1 Bypass Blocks}
\label{subsubsec:kepatuhan_kriteria_2.4.1}
(Tidak Sukses) \\

Kriteria ini tidak sukses dipatuhi karena tidak ada mekanisme untuk meloncati menu pada \textit{sidebar}.

\subsubsection{Kriteria Sukses 2.4.2 Page Titled}
\label{subsubsec:kepatuhan_kriteria_2.4.2}
(Sukses) \\

Kriteria ini sukses dipatuhi karena semua halaman pada aplikasi \textit{SharIF Judge} memiliki judul yang menggambarkan topik atau tujuan.

\subsubsection{Kriteria Sukses 2.4.3 Focus Order}
\label{subsubsec:kepatuhan_kriteria_2.4.3}
(Sukses) \\

Kriteria ini sukses dipatuhi karena semua halaman pada aplikasi \textit{SharIF Judge} yang memiliki urutan navigasi dapat menerima fokus dalam urutan yang menjaga makna dan pengoperasiannya.

\subsubsection{Kriteria Sukses 2.4.4 Link Purpose (In Context)}
\label{subsubsec:kepatuhan_kriteria_2.4.4}
(Tidak Sukses) \\

Kriteria ini tidak sukses dipatuhi karena :
\begin{itemize}
	\item \textit{Link} yang berada pada sidebar tidak memiliki teks yang dapat diakses.
\end{itemize}

\subsubsection{Kriteria Sukses 2.4.5 Multiple Ways}
\label{subsubsec:kepatuhan_kriteria_2.4.5}

\subsubsection{Kriteria Sukses 2.4.6 Headings and Labels}
\label{subsubsec:kepatuhan_kriteria_2.4.6}

\subsubsection{Kriteria Sukses 2.4.7 Focus Visible}
\label{subsubsec:kepatuhan_kriteria_2.4.7}
(Tidak Sukses) \\

Kriteria ini tidak sukses dipatuhi karena fokus tidak tampak ketika fokus sedang berada pada menu \textit{sidebar}.

\subsubsection{Kriteria Sukses 2.4.8 Location}
\label{subsubsec:kepatuhan_kriteria_2.4.8}
(Sukses) \\

Kriteria ini sukses dipatuhi karena pada setiap halaman aplikasi \textit{SharIF Judge} terdapat judul yang menjelaskan lokasi pengguna.

\subsubsection{Kriteria Sukses 2.4.9 Link Purpose (Link Only)}
\label{subsubsec:kepatuhan_kriteria_2.4.9}

\subsubsection{Kriteria Sukses 2.4.10 Section Headings}
\label{subsubsec:kepatuhan_kriteria_2.4.10}

Tanya : 

\subsection{\textit{Understandable}}
\label{subsec:kepatuhan_understandable}

\subsubsection{Kriteria Sukses 3.1.1 Language of Page}
\label{subsubsec:kepatuhan_kriteria_3.1.1}
(Tidak Sukses) \\

Kriteria ini tidak sukses dipatuhi karena elemen \textit{html} ditidak memiliki atribut \textit{lang}.

\subsection{\textit{Robust}}
\label{subsec:kepatuhan_robust}

\subsubsection{Kriteria Sukses 4.1.2 Name, Role, Value}
\label{subsubsec:kepatuhan_kriteria_4.1.2}
(Tidak Sukses) \\

Kriteria ini tidak sukses dipatuhi karena : 
\begin{itemize}
	\item Teknologi bantuan tidak dapat mengambil informasi pada \textit{sidebar} karena tidak ada teks yang menjelaskan link tersebut.
\end{itemize}
