\chapter{Analisis}
\label{chap:analisis}

\section{Tingkat Kepatuhan \textit{SharIF Judge}}

\begin{table}[H]
	\centering
	\begin{tabular}{|l|c|c|}
		\hline
		Kriteria Sukses & Tingkat Kepatuhan & Hasil \\
		\hline
		Kriteria Sukses 1.1.1 Non-text Content & A & TODO \\
		Kriteria Sukses 1.2.1 Audio-only dan Video-only (Prerecorded) & A & TODO \\
		Kriteria Sukses 1.2.2 Captions (Prerecorded) & A & TODO \\
		Kriteria Sukses 1.2.3 Audio Descriptive atau Media Alternative (Prerecorded) & A & TODO \\
		Kriteria Sukses 1.2.4 Captions (Live) & AA & TODO \\
		Kriteria Sukses 1.2.5 Audio Description (Prerecorded) & AA & TODO \\
		Kriteria Sukses 1.2.6 Sign Language (Prerecorded) & AAA & TODO \\
		Kriteria Sukses 1.2.7 Extended Audio Description (Prerecorded) & AAA & TODO \\
		Kriteria Sukses 1.2.8 Media Alternative (Prerecorded) & AAA & TODO \\
		Kriteria Sukses 1.2.9 Audio-only (Live) & AAA & TODO \\
		Kriteria Sukses 1.3.1 Info dan Relationships & A & TODO \\
		Kriteria Sukses 1.3.2 Meaningful Sequence & A & TODO \\
		Kriteria Sukses 1.3.3 Sensory Characteristics & A & TODO \\
		Kriteria Sukses 1.3.4 Orientation & A & TODO \\
		Kriteria Sukses 1.3.5 Identify Input Purpose & AA & TODO \\
		Kriteria Sukses 1.3.6 Identify Purpose & AAA & TODO \\
		\hline
	\end{tabular}
\end{table}