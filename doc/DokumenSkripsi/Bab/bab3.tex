\chapter{Analisis}
\label{chap:analisis}

Bab ini membahas tentang analisis tingkat kepatuhan \textit{SharIF Judge} terhadap \textit{WCAG} 2.1. 

\definecolor{Pink}{rgb}{0.945,0.584,0.607}
\definecolor{LightRed}{rgb}{0.941,0.454,0.439}
\definecolor{Red}{rgb}{0.917,0.298,0.274}
\definecolor{Gray}{gray}{0.7}

\section{Metodologi Pengecekan}
\label{sec:metodologi_pengecekan}
Proses pengecekan kepatuhan setiap kriteria sukses pada aplikasi \textit{SharIF Judge} dilakukan 
dengan tahap-tahap sebagai berikut: 

\begin{enumerate}
	\item Menggunakan \textit{tools} \textit{axe - Web Accessibility Testing} dan \textit{Accessibility Insight for Web} pada seluruh halaman web untuk mendeteksi kepatuhan kriteria sukses.
	\item Jika kepatuhan kriteria sukses tidak terdeteksi dengan \textit{tools}, maka dilakukan pengecekan manual pada \textit{source code}.
\end{enumerate}

\textit{Tools} \textit{axe - Web Accessibility Testing} dan \textit{Accessibility Insight for Web} dapat memeriksa kepatuhan kriteria sukses \ref{subsec:kriteria_1.1.1}, \ref{subsec:kriteria_1.3.1}, \ref{subsec:kriteria_1.4.3}, \ref{subsec:kriteria_2.4.4}, \ref{subsec:kriteria_3.1.1}, \ref{subsec:kriteria_3.3.2}, dan \ref{subsec:kriteria_4.1.2}.

\section{Tingkat Kepatuhan \textit{SharIF Judge}}
\label{sec:kepatuhan_sharif_judge_terhadap_wcag_2.1}
Aplikasi \textit{SharIF Judge} sudah mematuhi 21 dari 30 kriteria sukses tingkat A, 9 dari 20 kriteria sukses tingkat AA, dan 18 dari 28 kriteria sukses tingkat AAA. Jumlah keberhasilan kriteria sukses berdasarkan tingkat dapat dilihat pada gambar \ref{fig:pembagian_kriteria_sukses_tingkat}.

\begin{figure}[H]
	\centering  
	\includegraphics[scale=0.35]{pembagian_tingkat}  
	\caption[Pembagian Kriteria Sukses]{Pembagian Kriteria Sukses Berdasarkan Tingkat} 
	\label{fig:pembagian_kriteria_sukses_tingkat} 
\end{figure}

Untuk membantu pemeriksaan kepatuhan \textit{SharIF Judge} terhadap \textit{WCAG} 2.1 digunakan alat bantu seperti \textit{axe - Web Accessibility Testing}\footnote{https://www.deque.com/axe/} dan \textit{Accessibility Insight for Web}\footnote{https://accessibilityinsights.io/}, tetapi alat bantu tersebut tidak dapat memeriksa seluruh kriteria sukses \textit{WCAG} 2.1. Kesuksesan aplikasi \textit{SharIF Judge} dalam mematuhi kriteria sukses \textit{WCAG} 2.1 ditulis dalam tabel \ref{tab:kepatuhan_sharif_judge_understandable}, \ref{tab:kepatuhan_sharif_judge_robust}, \ref{tab:kepatuhan_sharif_judge_perceivable}, dan \ref{tab:kepatuhan_sharif_judge_operable}. Penjelasan untuk analisis setiap kriteria sukses dapat dilihat pada subbab \ref{subsec:kepatuhan_perceivable}, \ref{subsec:kepatuhan_operable}, \ref{subsec:kepatuhan_understandable}, dan \ref{subsec:kepatuhan_robust}.

\begin{table}[H]
	\centering
	\caption{Kepatuhan \textit{SharIF Judge} terhadap prinsip \textit{Understandable}}
	\label{tab:kepatuhan_sharif_judge_understandable}
	\begin{tabular}{|c|c|c|}
		\hline
		Kriteria Sukses & Tingkat Kepatuhan & Hasil\\
		\hline
		\rowcolor{Pink} 3.1.1 & A & Tidak Sukses\\
		\rowcolor{LightRed} 3.1.2 & AA & Tidak Sukses\\
		3.1.3 & AAA& Sukses\\
		3.1.4 & AAA & Sukses\\
		3.1.5 & AAA & Sukses\\
		3.1.6 & AAA & Sukses\\
		3.2.1 & A & Sukses\\
		3.2.2 & A & Sukses\\
		3.2.3 & AA & Sukses\\
		\rowcolor{LightRed} 3.2.4 & AA & Tidak Sukses\\
		3.2.5 & AAA & Sukses\\
		3.3.1 & A & Sukses\\
		\rowcolor{Pink} 3.3.2 & A & Tidak Sukses\\
		3.3.3 & AA & Sukses\\
		3.3.4 & AA & Sukses\\
		\rowcolor{Red} 3.3.5 & AAA & Tidak Sukses\\
		3.3.6 & AAA & Sukses\\
		\hline
		\multicolumn{2}{|c|}{Tingkat kepatuhan tertinggi yang dicapai} & - \\
		\hline
	\end{tabular}
\end{table}

\begin{table}[H]
	\centering
	\caption{Kepatuhan \textit{SharIF Judge} terhadap prinsip \textit{Robust}}
	\label{tab:kepatuhan_sharif_judge_robust}
	\begin{tabular}{|c|c|c|}
		\hline
		Kriteria Sukses & Tingkat Kepatuhan & Hasil\\
		\hline
		\rowcolor{Pink} 4.1.1 & A & Tidak Sukses\\
		4.1.2 & A & Sukses\\
		\rowcolor{LightRed} 4.1.3 & AA & Tidak Sukses\\
		\hline
		\multicolumn{2}{|c|}{Tingkat kepatuhan tertinggi yang dicapai} & - \\
		\hline
	\end{tabular}
\end{table}

\begin{table}[H]
	\centering
	\caption{Kepatuhan \textit{SharIF Judge} terhadap prinsip \textit{Perceivable}}
	\label{tab:kepatuhan_sharif_judge_perceivable}
	\begin{tabular}{|c|c|c|}
		\hline
		Kriteria Sukses & Tingkat Kepatuhan & Hasil \\
		\hline
		\rowcolor{Pink} 1.1.1 & A & Tidak Sukses\\
		1.2.1 & A & Sukses\\
		1.2.2 & A & Sukses \\
		1.2.3 & A & Sukses \\
		1.2.4 & AA & Sukses \\
		1.2.5 & AA & Sukses \\
		1.2.6 & AAA & Sukses \\
		1.2.7 & AAA & Sukses \\
		1.2.8 & AAA & Sukses \\
		1.2.9 & AAA & Sukses \\
		\rowcolor{Pink} 1.3.1 & A & Tidak Sukses\\
		1.3.2 & A & Sukses \\
		1.3.3 & A & Sukses \\
		1.3.4 & A & Sukses \\
		\rowcolor{LightRed} 1.3.5 & AA & Tidak Sukses\\
		\rowcolor{Red} 1.3.6 & AAA & Tidak Sukses\\
		1.4.1 & A & Sukses \\
		1.4.2 & A & Sukses \\
		\rowcolor{LightRed} 1.4.3 & AA & Tidak Sukses\\
		\rowcolor{LightRed} 1.4.4 & AA & Tidak Sukses\\
		1.4.5 & AA & Sukses\\
		\rowcolor{Red} 1.4.6 & AAA & Tidak Sukses\\
		1.4.7 & AAA & Sukses \\
		\rowcolor{Red} 1.4.8 & AAA & Tidak Sukses\\
		1.4.9 & AAA & Sukses\\
		\rowcolor{LightRed} 1.4.10 & AA & Tidak Sukses\\
		\rowcolor{Gray} 1.4.11 & AA & Diabaikan\\
		1.4.12 & AA & Sukses\\
		1.4.13 & AA & Sukses\\
		\hline
		\multicolumn{2}{|c|}{Tingkat kepatuhan tertinggi yang dicapai} & - \\
		\hline
	\end{tabular}
\end{table}

\begin{center}
	\begin{longtable}{|c|c|c|}
		\caption{Kepatuhan \textit{SharIF Judge} terhadap prinsip \textit{Operable}}
		\label{tab:kepatuhan_sharif_judge_operable}\\
		
		\hline
		Kriteria Sukses & Tingkat Kepatuhan & Hasil\\
		\hline
		\endfirsthead
		
		\multicolumn{3}{c}%
		{{\bfseries \tablename\ \thetable{} -- lanjutan dari halaman sebelumnya}} \\
		\hline \multicolumn{1}{|c|}{Kriteria Sukses} & \multicolumn{1}{c|}{Tingkat Kepatuhan} & \multicolumn{1}{c|}{Hasil} \\ \hline 
		\endhead
		
		\hline \multicolumn{3}{|r|}{{Dilanjutkan pada halaman berikutnya}} \\ \hline
		\endfoot
		
		\hline
		\multicolumn{2}{|c|}{Tingkat kepatuhan tertinggi yang dicapai} & - \\
		\hline
		\endlastfoot
		
		\rowcolor{Pink} 2.1.1 & A & Tidak Sukses\\
		\rowcolor{Pink} 2.1.2 & A & Tidak Sukses\\
		\rowcolor{Red} 2.1.3 & AAA & Tidak Sukses\\
		2.1.4 & A & Sukses\\
		2.2.1 & A & Sukses\\
		2.2.2 & A & Sukses\\
		2.2.3 & AAA & Sukses\\
		2.2.4 & AAA & Sukses\\
		\rowcolor{Red} 2.2.5 & AAA & Tidak Sukses\\
		\rowcolor{Red} 2.2.6 & AAA & Tidak Sukses\\
		2.3.1 & A & Sukses\\
		2.3.2 & AAA & Sukses\\
		2.3.3 & AAA & Sukses\\
		\rowcolor{Pink} 2.4.1 & A & Tidak Sukses\\
		2.4.2 & A & Sukses\\
		2.4.3 & AA & Sukses\\
		\rowcolor{Pink} 2.4.4 & A & Tidak Sukses\\
		\rowcolor{LightRed} 2.4.5 & AA & Tidak Sukses\\
		\rowcolor{LightRed} 2.4.6 & AA & Tidak Sukses\\
		\rowcolor{LightRed} 2.4.7 & AA & Tidak Sukses\\
		2.4.8 & AAA & Sukses\\
		\rowcolor{Red} 2.4.9 & AAA & Tidak Sukses\\
		\rowcolor{Red} 2.4.10 & AAA & Tidak Sukses\\
		2.5.1 & A & Sukses\\
		2.5.2 & A & Sukses\\
		2.5.3 & A & Sukses\\
		2.5.4 & A & Sukses\\
		\rowcolor{Red} 2.5.5 & AAA & Tidak Sukses\\
		2.5.6 & AAA & Sukses\\
	\end{longtable}
\end{center}

\subsection{\textit{Perceivable}}
\label{subsec:kepatuhan_perceivable}

Berikut adalah kepatuhan \textit{SharIF Judge} berdasarkan prinsip \textit{perceivable}.

\subsubsection{\textit{Text Alternatives}}
\label{subsubsec:text_alternatives}

Pada subbab ini dibahas kriteria sukses yang termasuk kedalam kategori \textit{text alternatives}. Kategori ini mensyaratkan agar konten yang bukan teks memiliki teks alternatif.

\paragraph{Kriteria Sukses 1.1.1 \textit{Non-text Content}}
\label{subsubsec:kepatuhan_kriteria_1.1.1}
(Tidak Sukses) \\

Kriteria ini tidak sukses dipatuhi karena:
\begin{itemize}
	\item Gambar logo \textit{SharIF Judge} pada menu bagian atas tidak memiliki alternatif teks. Kesalahan dapat dilihat pada potongan kode \ref{ls_kepatuhan_1_1_1_logo_sharif_judge}.
	\begin{lstlisting}[basicstyle=\ttfamily, frame=single,
	columns=fullflexible, keepspaces=true, breaklines=true, label=ls_kepatuhan_1_1_1_logo_sharif_judge, caption=Pelanggaran Kriteria Sukses 1.1.1 - Logo \textit{SharIF Judge} Tidak Diberi Alternatif Teks]
	...
	<a href="{{ site_url('/') }}">
		<img src="{{ base_url('assets/images/logo_small.png') }}"/>
		<h1 class="shjlogo-text">SharIF <span>Judge</span></h1>
	</a>
	...
	\end{lstlisting}
	\item Gambar pdf pada halaman \textit{Assignment} tidak memiliki alternatif teks. Kesalahan dapat dilihat pada potongan kode \ref{ls_kepatuhan_1_1_1_logo_pdf}.
	\begin{lstlisting}[basicstyle=\ttfamily, frame=single,
	columns=fullflexible, keepspaces=true, breaklines=true, label=ls_kepatuhan_1_1_1_logo_pdf, caption=Pelanggaran Kriteria Sukses 1.1.1 - Gambar PDF Tidak Diberi Alternatif Teks]
	...
	<td>
	<a href="{{ site_url('assignments/pdf/'~item.id) }}"><img src="{{ base_url('assets/images/pdf.svg') }}" /></a>
	</td>
	...
	\end{lstlisting}
\end{itemize}

\subsubsection{\textit{Time-based Media}}
\label{subsubsec:time-based_media}

Pada subbab ini dibahas kriteria sukses yang termasuk kedalam kategori \textit{time-based media}. Kategori ini mensyaratkan agar konten berbasis waktu memiliki versi alternatifnya.

\paragraph{Kriteria Sukses 1.2.1 \textit{Audio-only dan Video-only (Prerecorded)}}
\label{subsubsec:kepatuhan_kriteria_1.2.1}
(Sukses) \\

Kriteria ini sukses dipatuhi karena pada aplikasi \textit{SharIF Judge} tidak terdapat konten berbasis waktu.

\paragraph{Kriteria Sukses 1.2.2 \textit{Captions (Prerecorded)}}
\label{subsubsec:kepatuhan_kriteria_1.2.2}
(Sukses) \\

Kriteria ini sukses dipatuhi karena pada aplikasi \textit{SharIF Judge} tidak terdapat konten berbasis waktu.

\paragraph{Kriteria Sukses 1.2.3 \textit{Audio Descriptive atau Media Alternative (Prerecorded)}}
\label{subsubsec:kepatuhan_kriteria_1.2.3}
(Sukses) \\

Kriteria ini sukses dipatuhi karena pada aplikasi \textit{SharIF Judge} tidak terdapat konten berbasis waktu.

\paragraph{Kriteria Sukses 1.2.4 \textit{Captions (Live)}}
\label{subsubsec:kepatuhan_kriteria_1.2.4}
(Sukses) \\

Kriteria ini sukses dipatuhi karena pada aplikasi \textit{SharIF Judge} tidak terdapat konten berbasis waktu.

\paragraph{Kriteria Sukses 1.2.5 \textit{Audio Description (Prerecorded)}}
\label{subsubsec:kepatuhan_kriteria_1.2.5}
(Sukses) \\

Kriteria ini sukses dipatuhi karena pada aplikasi \textit{SharIF Judge} tidak terdapat konten berbasis waktu.

\paragraph{Kriteria Sukses 1.2.6 \textit{Sign Language (Prerecorded)}}
\label{subsubsec:kepatuhan_kriteria_1.2.6}
(Sukses) \\

Kriteria ini sukses dipatuhi karena pada aplikasi \textit{SharIF Judge} tidak terdapat konten berbasis waktu.

\paragraph{Kriteria Sukses 1.2.7 \textit{Extended Audio Description (Prerecorded)}}
\label{subsubsec:kepatuhan_kriteria_1.2.7}
(Sukses) \\

Kriteria ini sukses dipatuhi karena pada aplikasi \textit{SharIF Judge} tidak terdapat konten berbasis waktu.

\paragraph{Kriteria Sukses 1.2.8 \textit{Media Alternative (Prerecorded)}}
\label{subsubsec:kepatuhan_kriteria_1.2.8}
(Sukses) \\

Kriteria ini sukses dipatuhi karena pada aplikasi \textit{SharIF Judge} tidak terdapat konten berbasis waktu.

\paragraph{Kriteria Sukses 1.2.9 \textit{Audio-only (Live)}}
\label{subsubsec:kepatuhan_kriteria_1.2.9}
(Sukses) \\

Kriteria ini sukses dipatuhi karena pada aplikasi \textit{SharIF Judge} tidak terdapat konten berbasis waktu.

\subsubsection{\textit{Adaptable}}
\label{subsubsec:adaptable}

Pada subbab ini dibahas kriteria sukses yang termasuk kedalam kategori \textit{adaptable}. Kategori ini mensyaratkan agar konten dapat ditampilkan dengan berbagai cara (contohnya tata letak yang lebih sederhana) tanpa kehilangan informasi atau struktur konten tersebut.

\paragraph{Kriteria Sukses 1.3.1 \textit{Info and Relationships}}
\label{subsubsec:kepatuhan_kriteria_1.3.1}
(Tidak Sukses) \\

Kriteria ini tidak sukses dipatuhi karena:
\begin{itemize}
	\item Ada elemen dalam form \textit{Add Users} yang tidak diberi label. Tampilan halaman web dapat dilihat pada gambar \ref{fig:kepatuhan_1_3_1_add_user}.
	\begin{figure}[H]
		\centering  
		\includegraphics[width=0.5\textwidth]{kepatuhan_1_3_1_add_user}  
		\caption[Pelanggaran Kriteria Sukses 1.3.1 - Elemen yang Tidak Diberi Label Pada Halaman \textit{Add Users}]{Pelanggaran Kriteria Sukses 1.3.1 - Elemen yang Tidak Diberi Label Pada Halaman \textit{Add Users}} 
		\label{fig:kepatuhan_1_3_1_add_user} 
	\end{figure}
	\item Dalam halaman \textit{Add Assignment} ada elemen dalam form \textit{problems} tidak diberi label. Kesalahan dapat dilihat pada potongan kode \ref{ls_kepatuhan_1_3_1_add_assignment}.
	\begin{lstlisting}[basicstyle=\ttfamily, frame=single,
	columns=fullflexible, keepspaces=true, breaklines=true, label=ls_kepatuhan_1_3_1_add_assignment, caption=Pelanggaran Kriteria Sukses 1.3.1 - Elemen Dalam Form yang Tidak Diberi Label]
	...
	<td><input type="text" name="name[]" class="sharif_input short" value="Problem "/></td>\
	...
	\end{lstlisting}
	
	\item Dalam halaman \textit{Problem} terdapat elemen form yang tidak diberi label.
	\begin{figure}[H]
		\centering  
		\includegraphics[width=0.5\textwidth]{kepatuhan_1_3_1_problem_edit_markdown}  
		\caption[Pelanggaran Kriteria Sukses 1.3.1 - Elemen yang Tidak Diberi Label Pada Halaman \textit{Problems} Bagian \textit{Edit Markdown}]{Pelanggaran Kriteria Sukses 1.3.1 - Elemen yang Tidak Diberi Label Pada Halaman \textit{Problems} Bagian \textit{Edit Markdown}} 
		\label{fig:kepatuhan_1_3_1_problem_edit_markdown} 
	\end{figure}
	\begin{figure}[H]
		\centering  
		\includegraphics[width=0.5\textwidth]{kepatuhan_1_3_1_edit_plain_html}  
		\caption[Pelanggaran Kriteria Sukses 1.3.1 - Elemen yang Tidak Diberi Label Pada Halaman \textit{Problems} Bagian \textit{Edit Plain Html}]{Pelanggaran Kriteria Sukses 1.3.1 - Elemen yang Tidak Diberi Label Pada Halaman \textit{Problems} Bagian \textit{Edit Plain Html}} 
		\label{fig:kepatuhan_1_3_1_plain_html} 
	\end{figure}
\end{itemize}

\paragraph{Kriteria Sukses 1.3.2 \textit{Meaningful Sequence}}
\label{subsubsec:kepatuhan_kriteria_1.3.2}
(Sukses) \\

Kriteria ini sukses dipatuhi karena setiap halaman pada \textit{SharIF Judge} memiliki urutan baca dan navigasi yang benar.

\paragraph{Kriteria Sukses 1.3.3 \textit{Sensory Characteristics}}
\label{subsubsec:kepatuhan_kriteria_1.3.3}
(Sukses)\\

Kriteria ini sukses dipatuhi karena petunjuk yang diberikan pada aplikasi \textit{SharIF Judge} tidak hanya bergantung pada komponen karakteristik sensorik seperti bentuk, warna, ukuran, lokasi visual, orientasi, atau suara.

\paragraph{Kriteria Sukses 1.3.4 \textit{Orientation}}
\label{subsubsec:kepatuhan_kriteria_1.3.4}
(Sukses) \\

Kriteria ini sukses dipatuhi karena konten dari \textit{SharIF Judge} dapat ditampilkan dalam orientasi \textit{portrait} atau \textit{landscape}.

\paragraph{Kriteria Sukses 1.3.5 \textit{Identify Input Purpose}}
\label{subsubsec:kepatuhan_kriteria_1.3.5}
(Tidak Sukses)\\

Kriteria ini tidak sukses karena masih ada bidang masukan yang tidak memiliki label, salah satunya pada halaman \textit{Problems} bagian \textit{Edit Markdown}. Kesalahan dapat dilihat pada potongan kode \ref{ls_kepatuhan_1_3_5}.
\begin{lstlisting}[basicstyle=\ttfamily, frame=single,
columns=fullflexible, keepspaces=true, breaklines=true, label=ls_kepatuhan_1_3_5, caption=Pelanggaran Kriteria Sukses 1.3.5 - Elemen Tidak Diberi Label Pada Halaman \textit{Problems} Bagian \textit{Edit Markdown}]
...
<p class="input_p">
	<textarea name="text">{{ problem.description }}</textarea>
</p>
...
\end{lstlisting}

\paragraph{Kriteria Sukses 1.3.6 \textit{Identify Purpose}}
\label{subsubsec:kepatuhan_kriteria_1.3.6}
(Tidak Sukses) \\

Kriteria ini tidak sukses dipatuhi karena ada elemen \textit{HTML} yang seharusnya dapat dipakai tetapi tidak digunakan, seperti bagian navigasi dan judul bagian. Kesalahan dapat dilihat pada potongan kode \ref{ls_kepatuhan_1_3_6}.
\begin{lstlisting}[basicstyle=\ttfamily, frame=single,
columns=fullflexible, keepspaces=true, breaklines=true, label=ls_kepatuhan_1_3_6, caption=Pelanggaran Kriteria Sukses 1.3.6 - \textit{Sidebar} Tidak Memakai \textit{Tag} \textit{Nav}]
...
<div id="side_bar" class="sidebar_open">
	<ul>
		<li class="color-dashboard{{ selected=='dashboard' ? ' selected' }}">
...
\end{lstlisting}

\subsubsection{\textit{Distinguishable}}
\label{subsubsec:distinguishable}

Pada subbab ini dibahas kriteria sukses yang termasuk kedalam kategori \textit{distinguishable}. Kategori ini mensyaratkan agar konten dapat dilihat dan didengar dengan mudah, termasuk memisahkan latar depan dari latar belakang.

\paragraph{Kriteria Sukses 1.4.1 \textit{Use of Color}}
\label{subsubsec:kepatuhan_kriteria_1.4.1}
(Sukses) \\

Kriteria ini sukses dipatuhi karena warna tidak digunakan sebagai satu-satunya cara untuk menyampaikan informasi, menunjukkan aksi, menampilkan respon, atau membedakan elemen visual. Selain warna, ada juga teks yang menjelaskan informasi yang ditampilkan.

\paragraph{Kriteria Sukses 1.4.2 \textit{Audio Control}}
\label{subsubsec:kepatuhan_kriteria_1.4.2}
(Sukses) \\

Kriteria ini sukses dipatuhi karena pada aplikasi \textit{SharIF Judge} tidak terdapat konten berbasis waktu.

\paragraph{Kriteria Sukses 1.4.3 \textit{Contrast (Minimum)}}
\label{subsubsec:kepatuhan_kriteria_1.4.3}
(Tidak Sukses) \\

Kriteria ini tidak sukses dipatuhi karena pada aplikasi \textit{SharIF Judge} terdapat beberapa teks dengan rasio kontras kurang dari 4.5:1, antara lain:
\begin{itemize}
	\item Navigasi Atas: teks tempat waktu yang tersisa untuk mengumpulkan suatu \textit{Assignment} memiliki rasio kontras 2.52:1 terhadap warna latar belakangnya. Tampilan halaman web dapat dilihat pada gambar \ref{fig:kepatuhan_1_4_3_navigasi_atas}.
	\begin{figure}[H]
		\centering  
		\includegraphics[scale=0.25]{kepatuhan_1_4_3_navigasi_atas}  
		\caption[Pelanggaran Kriteria Sukses 1.4.3 - Kontras Elemen Navigasi Atas]{Pelanggaran Kriteria Sukses 1.4.3 - Kontras Elemen Navigasi Atas} 
		\label{fig:kepatuhan_1_4_3_navigasi_atas} 
	\end{figure}

	\item Halaman \textit{Dashboard}: teks pada tombol \textit{today} memiliki rasio kontras 4.01:1 terhadap warna latar belakangnya. Teks pada kalender memiliki rasio kontras 2.69:1 terhadap warna latar belakangnya. Teks pada kalender yang menunjukkan tanggal memiliki rasio kontras 1.57:1 terhadap warna latar belakangnya. Teks yang menunjukkan waktu notifikasi dibuat memiliki rasio kontras 2.07:1 terhadap warna latar belakangnya. Tampilan halaman web dapat dilihat pada gambar \ref{fig:kepatuhan_1_4_3_dashboard}.
	\begin{figure}[H]
		\centering  
		\includegraphics[scale=0.5]{kepatuhan_1_4_3_dashboard}  
		\caption[Pelanggaran Kriteria Sukses 1.4.3 - Kontras Elemen \textit{Dashboard}]{Pelanggaran Kriteria Sukses 1.4.3 - Kontras Elemen \textit{Dashboard}} 
		\label{fig:kepatuhan_1_4_3_dashboard} 
	\end{figure}

	\item Halaman \textit{Settings}: informasi tambahan pada bidang masukkan memiliki rasio kontras 3.55:1 terhadap warna latar belakangnya. Informasi peringatan pada bidang masukkan memiliki rasio kontras 3.69:1 terhadap warna latar belakangnya. Tampilan halaman web dapat dilihat pada gambar \ref{fig:kepatuhan_1_4_3_settings}.
	\begin{figure}[H]
		\centering  
		\includegraphics[scale=0.5]{kepatuhan_1_4_3_settings}  
		\caption[Pelanggaran Kriteria Sukses 1.4.3 - Kontras Elemen \textit{Settings}]{Pelanggaran Kriteria Sukses 1.4.3 - Kontras Elemen \textit{Settings}} 
		\label{fig:kepatuhan_1_4_3_settings} 
	\end{figure}

	\item Halaman \textit{Notifications}: teks yang menunjukkan waktu notifikasi dibuat memiliki rasio kontras 1.91:1 terhadap warna latar belakangnya. Tampilan halaman web dapat dilihat pada gambar \ref{fig:kepatuhan_1_4_3_notifications}.
	\begin{figure}[H]
		\centering  
		\includegraphics[scale=0.5]{kepatuhan_1_4_3_notifications}  
		\caption[Pelanggaran Kriteria Sukses 1.4.3 - Kontras Elemen \textit{Notifications}]{Pelanggaran Kriteria Sukses 1.4.3 - Kontras Elemen \textit{Notifications}} 
		\label{fig:kepatuhan_1_4_3_notifications} 
	\end{figure}
	
	\item Halaman \textit{Assignment}: teks yang berwana merah pada tabel \textit{Assignment} memiliki rasio kontras 3.89:1 terhadap warna latar belakangnya. Tampilan halaman web dapat dilihat pada gambar \ref{fig:kepatuhan_1_4_3_assignments}.
	\begin{figure}[H]
		\centering  
		\includegraphics[scale=0.3]{kepatuhan_1_4_3_assignments}  
		\caption[Pelanggaran Kriteria Sukses 1.4.3 - Kontras Elemen \textit{Assignments}]{Pelanggaran Kriteria Sukses 1.4.3 - Kontras Elemen \textit{Assignments}} 
		\label{fig:kepatuhan_1_4_3_assignments} 
	\end{figure}
	
	\item Halaman \textit{Add Assignment}: informasi tambahan pada bidang masukkan memiliki rasio kontras 3.55:1 terhadap warna latar belakangnya. Informasi peringatan pada bidang masukkan memiliki rasio kontras 3.69:1 terhadap warna latar belakangnya. Tampilan halaman web dapat dilihat pada gambar \ref{fig:kepatuhan_1_4_3_add_assignment}.
	\begin{figure}[H]
		\centering  
		\includegraphics[scale=0.5]{kepatuhan_1_4_3_add_assignment}  
		\caption[Pelanggaran Kriteria Sukses 1.4.3 - Kontras Elemen \textit{Add Assignments}]{Pelanggaran Kriteria Sukses 1.4.3 - Kontras Elemen \textit{Add Assignments}} 
		\label{fig:kepatuhan_1_4_3_add_assignment} 
	\end{figure}
	
\end{itemize}

\paragraph{Kriteria Sukses 1.4.4 \textit{Resize text}}
\label{subsubsec:kepatuhan_kriteria_1.4.4}
(Tidak Sukses) \\

Kriteria ini tidak sukses dipatuhi karena konten \textit{navigation bar} akan hilang pada pembesaran 125 persen. Tampilan halaman web dapat dilihat pada gambar \ref{fig:kepatuhan_1_4_4_nav_bar}.
\begin{figure}[H]
	\centering  
	\includegraphics[width=0.5\textwidth]{kepatuhan_1_4_4_nav_bar}  
	\caption[Pelanggaran Kriteria Sukses 1.4.4 - \textit{Sidenav}]{Pelanggaran Kriteria Sukses 1.4.4 - \textit{Sidenav}} 
	\label{fig:kepatuhan_1_4_4_nav_bar} 
\end{figure}


\paragraph{Kriteria Sukses 1.4.5 \textit{Images of Text}}
\label{subsubsec:kepatuhan_kriteria_1.4.5}
(Sukses) \\

Kriteria ini sukses dipatuhi karena pada aplikasi \textit{SharIF Judge} tidak ada gambar teks.	

\paragraph{Kriteria Sukses 1.4.6 \textit{Contrast (Enhanced)}}
\label{subsubsec:kepatuhan_kriteria_1.4.6}

(Tidak Sukses) \\

Kriteria ini tidak sukses dipatuhi karena pada aplikasi \textit{SharIF Judge} terdapat beberapa teks dengan rasio kontras kurang dari 4.5:1, antara lain:
\begin{itemize}
	\item Navigasi Atas: teks tempat waktu yang tersisa untuk mengumpulkan suatu \textit{Assignment} memiliki rasio kontras 2.52:1 terhadap warna latar belakangnya. Tampilan halaman web dapat dilihat pada gambar \ref{fig:kepatuhan_1_4_6_navigasi_atas}.
	\begin{figure}[H]
		\centering  
		\includegraphics[scale=0.25]{kepatuhan_1_4_6_navigasi_atas}  
		\caption[Pelanggaran Kriteria Sukses 1.4.6 - Kontras Elemen Navigasi Atas]{Pelanggaran Kriteria Sukses 1.4.6 - Kontras Elemen Navigasi Atas} 
		\label{fig:kepatuhan_1_4_6_navigasi_atas} 
	\end{figure}

	\item Halaman \textit{Dashboard}: teks pada tombol \textit{today} memiliki rasio kontras 4.01:1 terhadap warna latar belakangnya. Teks pada kalender memiliki rasio kontras 2.69:1 terhadap warna latar belakangnya. Teks pada kalender yang menunjukkan tanggal memiliki rasio kontras 1.57:1 terhadap warna latar belakangnya. Teks yang menunjukkan waktu notifikasi dibuat memiliki rasio kontras 2.07:1 terhadap warna latar belakangnya. Tampilan halaman web dapat dilihat pada gambar \ref{fig:kepatuhan_1_4_6_dashboard}.
	\begin{figure}[H]
		\centering  
		\includegraphics[scale=0.5]{kepatuhan_1_4_6_dashboard}  
		\caption[Pelanggaran Kriteria Sukses 1.4.6 - Kontras Elemen \textit{Dashboard}]{Pelanggaran Kriteria Sukses 1.4.6 - Kontras Elemen \textit{Dashboard}} 
		\label{fig:kepatuhan_1_4_6_dashboard} 
	\end{figure}

	\item Halaman \textit{Settings}: informasi tambahan pada bidang masukkan memiliki rasio kontras 3.55:1 terhadap warna latar belakangnya. Informasi peringatan pada bidang masukkan memiliki rasio kontras 3.69:1 terhadap warna latar belakangnya. Tampilan halaman web dapat dilihat pada gambar \ref{fig:kepatuhan_1_4_6_settings}.
	\begin{figure}[H]
		\centering  
		\includegraphics[scale=0.5]{kepatuhan_1_4_6_settings}  
		\caption[Pelanggaran Kriteria Sukses 1.4.6 - Kontras Elemen \textit{Settings}]{Pelanggaran Kriteria Sukses 1.4.6 - Kontras Elemen \textit{Settings}} 
		\label{fig:kepatuhan_1_4_6_settings} 
	\end{figure}

	\item Halaman \textit{Notifications}: teks yang menunjukkan waktu notifikasi dibuat memiliki rasio kontras 1.91:1 terhadap warna latar belakangnya. Tampilan halaman web dapat dilihat pada gambar \ref{fig:kepatuhan_1_4_6_notifications}.
	\begin{figure}[H]
		\centering  
		\includegraphics[scale=0.5]{kepatuhan_1_4_6_notifications}  
		\caption[Pelanggaran Kriteria Sukses 1.4.6 - Kontras Elemen \textit{Notifications}]{Pelanggaran Kriteria Sukses 1.4.6 - Kontras Elemen \textit{Notifications}} 
		\label{fig:kepatuhan_1_4_6_notifications} 
	\end{figure}

	\item Halaman \textit{Assignment}: teks yang berwana merah pada tabel \textit{Assignment} memiliki rasio kontras 3.89:1 terhadap warna latar belakangnya. Tampilan halaman web dapat dilihat pada gambar \ref{fig:kepatuhan_1_4_6_assignments}.
	\begin{figure}[H]
		\centering  
		\includegraphics[scale=0.3]{kepatuhan_1_4_6_assignments}  
		\caption[Pelanggaran Kriteria Sukses 1.4.6 - Kontras Elemen \textit{Assignments}]{Pelanggaran Kriteria Sukses 1.4.6 - Kontras Elemen \textit{Assignments}} 
		\label{fig:kepatuhan_1_4_6_assignments} 
	\end{figure}

	\item Halaman \textit{Add Assignment}: informasi tambahan pada bidang masukkan memiliki rasio kontras 3.55:1 terhadap warna latar belakangnya. Informasi peringatan pada bidang masukkan memiliki rasio kontras 3.69:1 terhadap warna latar belakangnya. Tampilan halaman web dapat dilihat pada gambar \ref{fig:kepatuhan_1_4_6_add_assignment}.
	\begin{figure}[H]
		\centering  
		\includegraphics[scale=0.5]{kepatuhan_1_4_6_add_assignment}  
		\caption[Pelanggaran Kriteria Sukses 1.4.6 - Kontras Elemen \textit{Add Assignments}]{Pelanggaran Kriteria Sukses 1.4.6 - Kontras Elemen \textit{Add Assignments}} 
		\label{fig:kepatuhan_1_4_6_add_assignment} 
	\end{figure}

\end{itemize}

\paragraph{Kriteria Sukses 1.4.7 \textit{Low or No Background Audio}}
\label{subsubsec:kepatuhan_kriteria_1.4.7}
(Sukses) \\

Kriteria ini sukses dipatuhi karena pada aplikasi \textit{SharIF Judge} tidak terdapat konten berbasis waktu.


\paragraph{Kriteria Sukses 1.4.8 \textit{Visual Presentation}}
\label{subsubsec:kepatuhan_kriteria_1.4.8}
(Tidak Sukses) \\

Kriteria ini tidak sukses dipatuhi karena pada halaman \textit{Dashboard} dan \textit{Notifications} teks yang menampilkan isi notifikasi memiliki lebar lebih dari 80 karakter. Tampilan halaman web ditampilkan pada gambar \ref{fig:kepatuhan_1_4_8_notifications}.
\begin{figure}[H]
	\centering  
	\includegraphics[scale=0.25]{kepatuhan_1_4_8_notifications}  
	\caption[Pelanggaran Kriteria Sukses 1.4.8 - Lebar \textit{Notifications}]{Pelanggaran Kriteria Sukses 1.4.8 - \textit{Notifications}} 
	\label{fig:kepatuhan_1_4_8_notifications} 
\end{figure}

\paragraph{Kriteria Sukses 1.4.9 \textit{Images of Text (No Exception)}}
\label{subsubsec:kepatuhan_kriteria_1.4.9}
(Sukses) \\

Kriteria ini sukses dipatuhi karena semua gambar teks yang ada pada aplikasi \textit{SharIF Judge} penting untuk informasi yang disampaikan.

\paragraph{Kriteria Sukses 1.4.10 \textit{Reflow}}
\label{subsubsec:kepatuhan_kriteria_1.4.10}
(Tidak Sukses) \\

Kriteria ini tidak sukses dipatuhi karena pada setiap halaman aplikasi \textit{SharIF Judge} memerlukan \textit{scroll} secara \textit{horizontal} ketika ditampilkan pada resolusi layar 1280 \textit{pixel} dengan pembesaran 400 persen. Tampilan halaman web dapat dilihat pada gambar \ref{fig:kepatuhan_1_4_10_reflow}
\begin{figure}[H]
	\centering  
	\includegraphics[scale=0.25]{kepatuhan_1_4_10}  
	\caption[Pelanggaran Kriteria Sukses 1.4.10 - \textit{Horizontal} \textit{Scroll}]{Pelanggaran Kriteria Sukses 1.4.10 - \textit{Horizontal} \textit{Scroll}} 
	\label{fig:kepatuhan_1_4_10_reflow} 
\end{figure}

\paragraph{Kriteria Sukses 1.4.11 \textit{Non-text Contrast}}
\label{subsubsec:kepatuhan_kriteria_1.4.11}
(Diabaikan) \\

Kriteria ini diabaikan karena pengujiannya terlalu banyak dan tidak ada alat bantu untuk mempermudah pengujian ini. Tampilan halaman web dapat dilihat pada gambar \ref{fig:kepatuhan_1_4_11_non_text_contrast}.
\begin{figure}[H]
	\centering  
	\includegraphics[scale=0.3]{kepatuhan_1_4_11}  
	\caption[Pelanggaran Kriteria Sukses 1.4.11 - Ikon Pada Halaman \textit{Assignments}]{Pelanggaran Kriteria Sukses 1.4.11 - Ikon Pada Halaman \textit{Assignments}} 
	\label{fig:kepatuhan_1_4_11_non_text_contrast} 
\end{figure}

Sebagai contoh perhitungan berikut adalah gambar ikon \textit{Download Test and Description}.
\begin{figure}[H]
	\centering  
	\includegraphics[scale=0.5]{kepatuhan_1_4_11_contoh}  
	\caption[Ikon \textit{Download Test and Description}]{Ikon \textit{Download Test and Description}} 
	\label{fig:kepatuhan_1_4_11_contoh} 
\end{figure}

Ikon \textit{Download Test and Description} memiliki spesifikasi warna sebagai berikut:
\[
R_{8bit} = 3
\]
\[
G_{8bit} = 166
\]
\[
B_{8bit} = 34
\]

Sedangkan warna latar belakangnya memiliki spesifikasi warna sebagai berikut:
\[
R_{8bit} = 252
\]
\[
G_{8bit} = 252
\]
\[
B_{8bit} = 252
\]

Rumus perhitungan kontras dapat dilihat pada subbab \ref{subsec:kriteria_1.4.3}. Berikut adalah contoh perhitungan \textit{relative luminance} untuk ikon \textit{Download Test and Description} pada gambar \ref{fig:kepatuhan_1_4_11_contoh}:
\begin{align*}
	R_{sRGB} &= 3 \div 255 = 0.01176 \\
	G_{sRGB} &= 166 \div 255 = 0.65098 \\
	B_{sRGB} &= 34 \div 255 = 1.33333 \\
	\\
	R &= 0.01176 \div 12.92 = 0.00091 \\
	G &= ((0.65098 + 0.055) \div 1.055)^{2.4} = 0.38132 \\
	B &= ((1.33333 + 0.055) \div 1.055)^{2.4} = 1.93275 \\
	\\
	L &= 0.2126 \times 0.00091 + 0.7152 \times 0.38132 + 0.0722 \times 1.93275 = 0.41245
\end{align*}

Sedangkan \textit{relative luminance} latar belakang pada ikon \textit{Download Test and Description} di gambar \ref{fig:kepatuhan_1_4_11_contoh} sebagai berikut:
\begin{align*}
	R_{sRGB} &= 252 \div 255 = 0.98823 \\
	G_{sRGB} &= 252 \div 255 = 0.98823 \\
	B_{sRGB} &= 252 \div 255 = 0.98823 \\
	\\
	R &= ((0.98823 + 0.055) \div 1.055)^{2.4} = 0.97343 \\
	G &= ((0.98823 + 0.055) \div 1.055)^{2.4} = 0.97343 \\
	B &= ((0.98823 + 0.055) \div 1.055)^{2.4} = 0.97343 \\
	\\
	L &= 0.2126 \times 0.97343 + 0.7152 \times 0.97343 + 0.0722 \times 0.97343 = 0.97343
\end{align*}

Hasil perbandingan kontras ikon \textit{Download Test and Description} dengan warna latar belakangnya pada gambar \ref{fig:kepatuhan_1_4_11_contoh} sebagai berikut:
\begin{align*}
	\text{\textit{Contrast Ratio}} &= (0.97343 + 0.05) : (0.41245 +0.05) \\
	&= 2.21306 : 1
\end{align*}

Dari hasil perhitungan di atas ditemukan bahwa perbandingan rasio kontras ikon \textit{Download Test and Description} dengan warna latar belakangnya pada gambar \ref{fig:kepatuhan_1_4_11_contoh} bernilai 2.21306 : 1.

\paragraph{Kriteria Sukses 1.4.12 \textit{Text Spacing}}
\label{subsubsec:kepatuhan_kriteria_1.4.12}
(Sukses) \\

Kriteria ini sukses dipatuhi karena penulis telah melakukan uji coba dengan cara mengubah seluruh \textit{text spacing} pada salah satu halaman agar kriteria ini terpenuhi dan hasilnya tidak terjadi masalah. Oleh karena itu, penulis menyimpulkan bahwa jika perubahan dilakukan di halaman lain akan menghasilkan hal yang sama. Tampilan halaman web dapat dilihat pada gambar \ref{fig:kepatuhan_1_4_12}.
\begin{figure}[H]
	\centering  
	\includegraphics[scale=0.3]{kepatuhan_1_4_12}  
	\caption[Pelanggaran Kriteria Sukses 1.4.12 - \textit{Text Spacing} Halaman \textit{Add Assignment}]{Pelanggaran Kriteria Sukses 1.4.12 - \textit{Text Spacing} Halaman \textit{Add Assignment}} 
	\label{fig:kepatuhan_1_4_12} 
\end{figure}

\paragraph{Kriteria Sukses 1.4.13 \textit{Content on Hover or Focus}}
\label{subsubsec:kepatuhan_kriteria_1.4.13}
(Sukses) \\

Kriteria ini sukses dipatuhi karena setiap konten tambahan yang muncul sesaat ketika suatu elemen menerima penunjuk kursor atau fokus \textit{keyboard}, konten tambahan tersebut dapat disingkirkan, dapat ditunjuk, dan persisten.

\subsection{\textit{Operable}}
\label{subsec:kepatuhan_operable}

Berikut adalah kepatuhan \textit{SharIF Judge} berdasarkan prinsip \textit{operable}.

\subsubsection{\textit{Keyboard Accessible}}
\label{subsubsec:keyboard_accessible}

Pada subbab ini dibahas kriteria sukses yang termasuk kedalam kategori \textit{Keyboard Accessible}. Kategori ini mensyaratkan semua fungsionalitas dapat diakses dengan \textit{keyboard}.

\paragraph{Kriteria Sukses 2.1.1 \textit{Keyboard}}
\label{subsubsec:kepatuhan_kriteria_2.1.1}
(Tidak Sukses) \\

Kriteria ini tidak sukses dipatuhi karena masih banyak elemen yang tidak dioperasikan dengan keyboard, salah satunya tombol \textit{Tools} pada \textit{Menu}. Tampilan halaman web dapat dilihat pada gambar \ref{fig:kepatuhan_2_1_1_tools}.
\begin{figure}[H]
	\centering  
	\includegraphics[scale=0.5]{kepatuhan_2_1_1_tools}  
	\caption[Pelanggaran Kriteria Sukses 2.1.1 - Tombol \textit{Tools}]{Pelanggaran Kriteria Sukses 2.1.1 - Tombol \textit{Tools}} 
	\label{fig:kepatuhan_2_1_1_tools} 
\end{figure}

\paragraph{Kriteria Sukses 2.1.2 \textit{No Keyboard Trap}}
\label{subsubsec:kepatuhan_kriteria_2.1.2}
(Tidak Sukses) \\

Kriteria ini tidak sukses dipatuhi karena:
\begin{itemize}
	\item Dalam halaman \textit{Settings} ada \textit{input field} yang fokusnya tidak dapat dipindahkan menggunakan antarmuka \textit{keyboard}. \textit{Input field} yang dimaksud dapat dilihat pada gambar \ref{fig:kepatuhan_2_1_2_settings}.
	\begin{figure}[H]
		\centering  
		\includegraphics[scale=0.35]{kepatuhan_2_1_2_settings}  
		\caption[Pelanggaran Kriteria Sukses 2.1.2 - \textit{Keyboard Trap} Pada Halaman \textit{Settings}]{Pelanggaran Kriteria Sukses 2.1.2 - \textit{Keyboard Trap} Pada Halaman \textit{Settings}} 
		\label{fig:kepatuhan_2_1_2_settings} 
	\end{figure}

	\item Dalam halaman \textit{Add User} ada \textit{input field} yang fokusnya tidak dapat dipindahkan menggunakan antarmuka \textit{keyboard}. \textit{Input field} yang dimaksud dapat dilihat pada gambar \ref{fig:kepatuhan_2_1_2_add_user}.
	\begin{figure}[H]
		\centering  
		\includegraphics[scale=0.35]{kepatuhan_2_1_2_add_user}  
		\caption[Pelanggaran Kriteria Sukses 2.1.2 - \textit{Keyboard Trap} Pada Halaman \textit{Add User}]{Pelanggaran Kriteria Sukses 2.1.2 - \textit{Keyboard Trap} Pada Halaman \textit{Add User}} 
		\label{fig:kepatuhan_2_1_2_add_user} 
	\end{figure}

	\item Dalam halaman \textit{Add Assignment} ada \textit{input field} yang fokusnya tidak dapat dipindahkan menggunakan antarmuka \textit{keyboard}. \textit{Input field} yang dimaksud dapat dilihat pada gambar \ref{fig:kepatuhan_2_1_2_add_assignment}.
	\begin{figure}[H]
		\centering  
		\includegraphics[scale=0.35]{kepatuhan_2_1_2_add_assignment}  
		\caption[Pelanggaran Kriteria Sukses 2.1.2 - \textit{Keyboard Trap} Pada Halaman \textit{Add Assignment}]{Pelanggaran Kriteria Sukses 2.1.2 - \textit{Keyboard Trap} Pada Halaman \textit{Add Assignment}} 
		\label{fig:kepatuhan_2_1_2_add_assignment} 
	\end{figure}

	\item Dalam halaman \textit{Problems}(\textit{Edit Markdown}) ada \textit{input field} yang fokusnya tidak dapat dipindahkan menggunakan antarmuka \textit{keyboard}. \textit{Input field} yang dimaksud dapat dilihat pada gambar \ref{fig:kepatuhan_2_1_2_problems_edit_markdown}.
	\begin{figure}[H]
		\centering  
		\includegraphics[scale=0.35]{kepatuhan_2_1_2_problems_edit_markdown}  
		\caption[Pelanggaran Kriteria Sukses 2.1.2 - \textit{Keyboard Trap} Pada Halaman \textit{Problems} Bagian \textit{Edit Markdown}]{Pelanggaran Kriteria Sukses 2.1.2 - \textit{Keyboard Trap} Pada Halaman \textit{Problems} Bagian \textit{Edit Markdown}} 
		\label{fig:kepatuhan_2_1_2_problems_edit_markdown} 
	\end{figure}

\end{itemize}

\paragraph{Kriteria Sukses 2.1.3 \textit{Keyboard (No Exception)}}
\label{subsubsec:kepatuhan_kriteria_2.1.3}
(Tidak Sukses) \\

Kriteria ini tidak sukses dipatuhi karena masih banyak elemen yang tidak dioperasikan dengan keyboard, salah satunya tombol \textit{Tools} pada \textit{Menu}. Tampilan halaman web dapat dilihat pada gambar \ref{fig:kepatuhan_2_1_3_tools}.
\begin{figure}[H]
	\centering  
	\includegraphics[scale=0.5]{kepatuhan_2_1_1_tools}  
	\caption[Pelanggaran Kriteria Sukses 2.1.1 - Tombol \textit{Tools}]{Pelanggaran Kriteria Sukses 2.1.1 - Tombol \textit{Tools}} 
	\label{fig:kepatuhan_2_1_3_tools} 
\end{figure}

\paragraph{Kriteria Sukses 2.1.4 \textit{Character Key Shortcuts}}
\label{subsubsec:kepatuhan_kriteria_2.1.4}
(Sukses) \\

Kriteria ini sukses dipatuhi karena pada aplikasi \textit{SharIF Judge} tidak terdapat pintasan \textit{keyboard} untuk konten yang ditampilkan.

\subsubsection{\textit{Enough Time}}
\label{subsubsec:enough_time}

Pada subbab ini dibahas kriteria sukses yang termasuk kedalam kategori \textit{enough time}. Kategori ini mensyaratkan agar tersedianya waktu yang cukup untuk pengguna membaca dan memanfaatkan konten.

\paragraph{Kriteria Sukses 2.2.1 \textit{Timing Adjustable}}
\label{subsubsec:kepatuhan_kriteria_2.2.1}
(Sukses) \\

Kriteria ini sukses dipatuhi karena pada aplikasi \textit{SharIF Judge} batas waktu untuk mengumpulkan \textit{Assignment} merupakan hal yang esensial dan perpanjangan batas waktu menyalahi inti dari kegiatan tersebut.

\paragraph{Kriteria Sukses 2.2.2 \textit{Pause, Stop, Hide}}
\label{subsubsec:kepatuhan_kriteria_2.2.2}
(Sukses) \\

Kriteria ini sukses dipatuhi karena pada aplikasi \textit{SharIF Judge} satu-satunya informasi yang diperbarui secara otomatis adalah batas waktu pengumpulan \textit{Assignment} yang merupakan bagian dari aktivitas yang esensial.

\paragraph{Kriteria Sukses 2.2.3 \textit{No Timing}}
\label{subsubsec:kepatuhan_kriteria_2.2.3}
(Sukses) \\

Kriteria ini sukses dipatuhi karena pada aplikasi \textit{SharIF Judge} waktu pengumpulan \textit{Assignment} bukanlah bagian esensial, pengguna dapat mendapatkan nilai yang sama selama mengumpulkan \textit{Assignment} dalam batas waktu yang telah ditentukan.

\paragraph{Kriteria Sukses 2.2.4 \textit{Interruptions}}
\label{subsubsec:kepatuhan_kriteria_2.2.4}
(Sukses) \\

Kriteria ini sukses dipatuhi karena pada aplikasi \textit{SharIF Judge} tidak ada interupsi.

\paragraph{Kriteria Sukses 2.2.5 \textit{Re-authenticating}}
\label{subsubsec:kepatuhan_kriteria_2.2.5}
(Tidak Sukses) \\

Kriteria ini tidak sukses dipatuhi karena pada aplikasi \textit{SharIF Judge} ketika sesi autentikasi berakhir, data yang belum disimpan akan hilang.

\paragraph{Kriteria Sukses 2.2.6 \textit{Timeouts}}
\label{subsubsec:kepatuhan_kriteria_2.2.6}
(Tidak Sukses) \\

Kriteria ini tidak sukses dipatuhi karena pada aplikasi \textit{SharIF Judge} waktu ketidakaktifan yang dapat menyebabkan kehilangan data kurang dari 20 jam. Pada potongan kode \ref{ls_kepatuhan_2_2_6} ditunjukkan bahwa sesi autentikasi hanya berlangsung selama 2 jam ketika user tidak melakukan tindakan apa pun.
\begin{lstlisting}[basicstyle=\ttfamily, frame=single,
columns=fullflexible, keepspaces=true, breaklines=true, label=ls_kepatuhan_2_2_6, caption=Pelanggaran Kriteria Sukses 2.2.6 - Sesi Autentikasi]
...
$config['sess_cookie_name']		= 'shjsession';
$config['sess_expiration']		= 7200;
...
\end{lstlisting}

\subsubsection{\textit{Seizures and Physical Reactions}}
\label{subsubsec:seizures_and_physical_reactions}

Pada subbab ini dibahas kriteria sukses yang termasuk kedalam kategori \textit{seizures and physical reactions}. Kategori ini mensyaratkan agar desain konten dirancang dan ditampilkan tidak menyebabkan kejang atau reaksi fisik bagi pengguna.

\paragraph{Kriteria Sukses 2.3.1 \textit{Three Flashes or Below Threshold}}
\label{subsubsec:kepatuhan_kriteria_2.3.1}
(Sukses)\\

Kriteria ini sukses dipatuhi karena pada aplikasi \textit{SharIF Judge} tidak ada konten yang berkedip lebih dari tiga detik dalam periode satu detik.

\paragraph{Kriteria Sukses 2.3.2 \textit{Three Flashes}}
\label{subsubsec:kepatuhan_kriteria_2.3.2}
(Sukses) \\

Kriteria ini sukses dipatuhi karena pada aplikasi \textit{SharIF Judge} tidak ada konten yang berkedip lebih dari tiga detik dalam periode satu detik.

\paragraph{Kriteria Sukses 2.3.3 \textit{Animation from Interactions}}
\label{subsubsec:kepatuhan_kriteria_2.3.3}
(Sukses) \\

Kriteria ini sukses dipatuhi karena satu-satunya animasi yang terdapat di aplikasi \textit{SharIF Judge} adalah tampilan batas waktu pengumpulan \textit{Assignment}, animasi ini penting untuk informasi yang disampaikan.

\subsubsection{\textit{Navigable}}
\label{subsubsec:navigable}

Pada subbab ini dibahas kriteria sukses yang termasuk kedalam kategori \textit{navigable}. Kategori ini mensyaratkan agar tersedianya cara untuk membantu pengguna bernavigasi, mencari konten, dan menentukan di mana mereka berada.

\paragraph{Kriteria Sukses 2.4.1 \textit{Bypass Blocks}}
\label{subsubsec:kepatuhan_kriteria_2.4.1}
(Tidak Sukses) \\

Kriteria ini tidak sukses dipatuhi karena tidak ada mekanisme untuk meloncati menu pada \textit{sidebar}.

\paragraph{Kriteria Sukses 2.4.2 \textit{Page Titled}}
\label{subsubsec:kepatuhan_kriteria_2.4.2}
(Sukses) \\

Kriteria ini sukses dipatuhi karena semua halaman pada aplikasi \textit{SharIF Judge} memiliki judul yang menggambarkan topik atau tujuan.

\paragraph{Kriteria Sukses 2.4.3 \textit{Focus Order}}
\label{subsubsec:kepatuhan_kriteria_2.4.3}
(Sukses) \\

Kriteria ini sukses dipatuhi karena semua halaman pada aplikasi \textit{SharIF Judge} yang memiliki urutan navigasi dapat menerima fokus dalam urutan yang menjaga makna dan pengoperasiannya.

\paragraph{Kriteria Sukses 2.4.4 \textit{Link Purpose (In Context)}}
\label{subsubsec:kepatuhan_kriteria_2.4.4}
(Tidak Sukses) \\

Kriteria ini tidak sukses dipatuhi karena:
\begin{itemize}
	\item Pada halaman \textit{Assignment} ada gambar yang memiliki tautan tetapi tidak ada teks yang menjelaskan tujuan dari tautan tersebut. Pada potongan kode \ref{ls_kepatuhan_2_4_4_assignment} ditunjukkan bahwa gambar pdf memiliki tautan yang tidak memiliki teks yang menjelaskan tujuannya.
	\begin{lstlisting}[basicstyle=\ttfamily, frame=single,
	columns=fullflexible, keepspaces=true, breaklines=true, label=ls_kepatuhan_2_4_4_assignment, caption=Pelanggaran Kriteria Sukses 2.4.4 - Gambar PDF Tidak Diberi Teks Alternatif]
	...
	<td>
		<a href="{{ site_url('assignments/pdf/'~item.id) }}"><img src="{{ base_url('assets/images/pdf.svg') }}" /></a>
	</td>
	...
	\end{lstlisting}
	
	\item Pada menu \textit{Top Bar} ada gambar yang memiliki tautan tetapi tidak ada teks yang menjelaskan tujuan dari tautan tersebut. Pada potongan kode \ref{ls_kepatuhan_2_4_4_top_bar} ditunjukkan bahwa gambar logo profil memiliki tautan yang tidak memiliki teks yang menjelaskan tujuannya.
	\begin{lstlisting}[basicstyle=\ttfamily, frame=single,
	columns=fullflexible, keepspaces=true, breaklines=true, label=ls_kepatuhan_2_4_4_top_bar, caption=Pelanggaran Kriteria Sukses 2.4.4 - Gambar Logo \textit{Profile} Tidak Diberi Teks Alternatif]
	...
	<a href="{{ site_url('profile') }}" id="profile_link"><i class="fa fa-user"></i></a>
	...
	\end{lstlisting}
	
\end{itemize}

\paragraph{Kriteria Sukses 2.4.5 \textit{Multiple Ways}}
\label{subsubsec:kepatuhan_kriteria_2.4.5}
(Tidak Sukses) \\

Kriteria ini tidak sukses dipatuhi karena pada halaman aplikasi \textit{SharIF Judge} satu-satunya cara untuk menemukan halaman web yang tersedia melalui \textit{sidebar}. Tampilan halaman web dapat dilihat pada gambar \ref{fig:kepatuhan_2_4_5}.
\begin{figure}[H]
	\centering  
	\includegraphics[scale=0.5]{kepatuhan_2_4_5}  
	\caption[Pelanggaran Kriteria Sukses 2.4.5 - Menemukan Halaman Web Pada Navigasi]{Pelanggaran Kriteria Sukses 2.4.5 - Menemukan Halaman Web Pada Navigasi} 
	\label{fig:kepatuhan_2_4_5} 
\end{figure}

\paragraph{Kriteria Sukses 2.4.6 \textit{Headings and Labels}}
\label{subsubsec:kepatuhan_kriteria_2.4.6}
(Tidak Sukses) \\

Kriteria ini tidak sukses dipatuhi karena:
\begin{itemize}
	\item Pada halaman \textit{Add User} ada \textit{field input} yang tidak memiliki label yang menjelaskan tujuannya. Pada potongan kode \ref{ls_kepatuhan_2_4_6_add_user} ada bidang masukan yang tidak memiliki label.
	\begin{lstlisting}[basicstyle=\ttfamily, frame=single,
	columns=fullflexible, keepspaces=true, breaklines=true, label=ls_kepatuhan_2_4_6_add_user, caption=Pelanggaran Kriteria Sukses 2.4.6 - Halaman \textit{Add User} Tidak Memiliki Label]
	...
	<p class="input_p">
		<textarea name="new_users" id="new_users" rows="20" cols="80" class="sharif_input">
		# Lines starting with a # sign are comments.
	...
	\end{lstlisting}
	
	\item Pada halaman \textit{Add Assignment} ada \textit{field input} yang tidak memiliki label yang menjelaskan tujuannya. Pada potongan kode \ref{ls_kepatuhan_2_4_6_add_assignment} ada bidang masukan yang tidak memiliki label.
	\begin{lstlisting}[basicstyle=\ttfamily, frame=single,
	columns=fullflexible, keepspaces=true, breaklines=true, label=ls_kepatuhan_2_4_6_add_assignment, caption=Pelanggaran Kriteria Sukses 2.4.6 - Halaman \textit{Add Assignment} Tidak Memiliki Label]
	...
	<td><input type="text" name="name[]" class="sharif_input short" value="{{ problem.name }}"/></td>
	...
	\end{lstlisting}
	
	\item Pada halaman \textit{Problems} bagian \textit{Edit Markdown} dan \textit{Edit Plain HTML} memiliki \textit{field input} yang tidak memiliki label yang menjelaskan tujuannya. Pada potongan kode \ref{ls_kepatuhan_2_4_6_edit_markdown} ada bidang masukan yang tidak memiliki label.
	\begin{lstlisting}[basicstyle=\ttfamily, frame=single,
	columns=fullflexible, keepspaces=true, breaklines=true, label=ls_kepatuhan_2_4_6_edit_markdown, caption=Pelanggaran Kriteria Sukses 2.4.6 - Halaman \textit{Problems} bagian \textit{Edit Markdown} Tidak Memiliki Label]
	...
	<p class="input_p">
		<textarea name="text">{{ problem.description }}</textarea>
	</p>
	...
	\end{lstlisting}
	
\end{itemize}

\paragraph{Kriteria Sukses 2.4.7 \textit{Focus Visible}}
\label{subsubsec:kepatuhan_kriteria_2.4.7}
(Tidak Sukses) \\

Kriteria ini tidak sukses dipatuhi karena fokus tidak tampak ketika fokus sedang berada pada menu \textit{sidebar}.

\paragraph{Kriteria Sukses 2.4.8 \textit{Location}}
\label{subsubsec:kepatuhan_kriteria_2.4.8}
(Sukses) \\

Kriteria ini sukses dipatuhi karena pada setiap halaman aplikasi \textit{SharIF Judge} terdapat judul yang menjelaskan lokasi pengguna.

\paragraph{Kriteria Sukses 2.4.9 \textit{Link Purpose (Link Only)}}
\label{subsubsec:kepatuhan_kriteria_2.4.9}
(Tidak Sukses) \\

Kriteria ini tidak sukses dipatuhi karena:
\begin{itemize}
	\item Pada halaman \textit{Assignment} ada gambar yang memiliki tautan tetapi tidak ada teks yang menjelaskan tujuan dari tautan tersebut. Pada potongan kode \ref{ls_kepatuhan_2_4_9_assignment} ditunjukkan bahwa gambar pdf memiliki tautan yang tidak memiliki teks yang menjelaskan tujuannya.
	\begin{lstlisting}[basicstyle=\ttfamily, frame=single,
	columns=fullflexible, keepspaces=true, breaklines=true, label=ls_kepatuhan_2_4_9_assignment, caption=Pelanggaran Kriteria Sukses 2.4.9 - Gambar PDF Tidak Diberi Teks Alternatif]
	...
	<td>
	<a href="{{ site_url('assignments/pdf/'~item.id) }}"><img src="{{ base_url('assets/images/pdf.svg') }}" /></a>
	</td>
	...
	\end{lstlisting}
	
	\item Pada menu \textit{Top Bar} ada gambar yang memiliki tautan tetapi tidak ada teks yang menjelaskan tujuan dari tautan tersebut. Pada potongan kode \ref{ls_kepatuhan_2_4_9_top_bar} ditunjukkan bahwa gambar logo profil memiliki tautan yang tidak memiliki teks yang menjelaskan tujuannya.
	\begin{lstlisting}[basicstyle=\ttfamily, frame=single,
	columns=fullflexible, keepspaces=true, breaklines=true, label=ls_kepatuhan_2_4_9_top_bar, caption=Pelanggaran Kriteria Sukses 2.4.9 - Gambar Logo \textit{Profile} Tidak Diberi Teks Alternatif]
	...
	<a href="{{ site_url('profile') }}" id="profile_link"><i class="fa fa-user"></i></a>
	...
	\end{lstlisting}
	
\end{itemize}

\paragraph{Kriteria Sukses 2.4.10 \textit{Section Headings}}
\label{subsubsec:kepatuhan_kriteria_2.4.10}
(Tidak Sukses) \\

Kriteria ini tidak sukses dipatuhi karena pada aplikasi \textit{SharIF Judge} judul bagian tidak dipakai untuk mengatur konten. Pada potongan kode \ref{ls_kepatuhan_2_4_10_section_headings} bagian judul bagian tidak menggunakan tag \textit{heading}.
\begin{lstlisting}[basicstyle=\ttfamily, frame=single,
columns=fullflexible, keepspaces=true, breaklines=true, label=ls_kepatuhan_2_4_10_section_headings, caption=Pelanggaran Kriteria Sukses 2.4.10 - \textit{Title Heading} Tidak Dipakai]
...
<div id="page_title">
	<i class="fa "></i>
	<span dir="auto"></span>
	
</div>
...
\end{lstlisting}

\subsubsection{\textit{Input Modalities}}
\label{subsubsec:input_modalities}

Pada subbab ini dibahas kriteria sukses yang termasuk kedalam kategori \textit{input modalities}. Kategori ini mensyaratkan agar pengguna dapat mengoperasikan fungsionalitas dengan mudah melalui berbagai \textit{input} selain \textit{keyboard}.

\paragraph{Kriteria Sukses 2.5.1 \textit{Pointer Gestures}}
\label{subsubsec:kepatuhan_kriteria_2.5.1}
(Sukses) \\

Kriteria ini sukses dipatuhi karena pada aplikasi \textit{SharIF Judge} tidak ada fungsionalitas yang harus dijalankan dengan menggunakan \textit{multipoint} atau gestur berbasis \textit{path}.

\paragraph{Kriteria Sukses 2.5.2 \textit{Pointer Cancellation}}
\label{subsubsec:kepatuhan_kriteria_2.5.2}
(Sukses) \\

Kriteria ini sukses dipatuhi karena pada aplikasi \textit{SharIF Judge} tidak ada fungsionalitas yang dapat dioperasikan dengan pointer tunggal dijalankan dengan menggunakan \textit{down-event}.

\paragraph{Kriteria Sukses 2.5.3 \textit{Label in Name}}
\label{subsubsec:kepatuhan_kriteria_2.5.3}
(Sukses) \\

Kriteria ini sukses dipatuhi karena pada aplikasi \textit{SharIF Judge} komponen dengan label yang menyertakan teks atau gambar teks, nama tersebut berisi teks yang ditampilkan secara visual.

\paragraph{Kriteria Sukses 2.5.4 \textit{Motion Actuation}}
\label{subsubsec:kepatuhan_kriteria_2.5.4}
(Sukses) \\

Kriteria ini sukses dipatuhi karena pada aplikasi \textit{SharIF Judge} tidak terdapat fungsionalitas yang dapat dioperasikan oleh gerakan perangkat atau gerakan pengguna.

\paragraph{Kriteria Sukses 2.5.5 \textit{Target Size}}
\label{subsubsec:kepatuhan_kriteria_2.5.5}
(Tidak Sukses) \\

Kriteria ini tidak sukses dipatuhi karena pada aplikasi \textit{SharIF Judge} masih banyak target yang dapat difokus berukuran kurang dari 44 piksel \textit{css}. Salah satu contohnya adalah bidang masukan \textit{Upload Size Limit} yang berukuran 30 piksel css. Tampilan halaman web dapat dilihat pada gambar \ref{fig:kepatuhan_2_5_5}.

\begin{figure}[H]
	\centering  
	\includegraphics[scale=0.5]{kepatuhan_2_5_5}  
	\caption[Pelanggaran Kriteria Sukses 2.5.5 - Bidang Masukan Berukuran 30 \textit{CSS Pixel}]{Pelanggaran Kriteria Sukses 2.5.5 - Bidang Masukan Berukuran 30 \textit{CSS Pixel}} 
	\label{fig:kepatuhan_2_5_5} 
\end{figure}

\paragraph{Kriteria Sukses 2.5.6 \textit{Concurrent Input Mechanisms}}
\label{subsubsec:kepatuhan_kriteria_2.5.6}
(Sukses) \\

Kriteria ini sukses dipatuhi karena pada aplikasi \textit{SharIF Judge} tidak membatasi modalitas masukan yang tersedia pada platform.

\subsection{\textit{Understandable}}
\label{subsec:kepatuhan_understandable}

Berikut adalah kepatuhan \textit{SharIF Judge} berdasarkan prinsip \textit{understandable}.

\subsubsection{\textit{Readable}}
\label{subsubsec:readable}

Pada subbab ini dibahas kriteria sukses yang termasuk kedalam kategori \textit{readable}. Kategori ini mensyaratkan agar konten teks dapat dibaca dan dipahami.

\paragraph{Kriteria Sukses 3.1.1 \textit{Language of Page}}
\label{subsubsec:kepatuhan_kriteria_3.1.1}
(Tidak Sukses) \\

Kriteria ini tidak sukses dipatuhi karena elemen \textit{html} tidak memiliki atribut \textit{lang}. Pada potongan kode \ref{ls_kepatuhan_3_1_1} ditunjukkan bahwa elemen \textit{html} tidak memiliki atribut \textit{lang}.
\begin{lstlisting}[basicstyle=\ttfamily, frame=single,
columns=fullflexible, keepspaces=true, breaklines=true, label=ls_kepatuhan_3_1_1, caption=Pelanggaran Kriteria Sukses 3.1.1 - Elemen \textit{HTML} Tidak Memiliki Atribut \textit{lang}]
...
<html>
<head>
...
\end{lstlisting}

\paragraph{Kriteria Sukses 3.1.2 \textit{Language of Parts}}
\label{subsubsec:kepatuhan_kriteria_3.1.2}
(Tidak Sukses) \\

Kriteria ini tidak sukses dipatuhi karena elemen \textit{html} tidak memiliki atribut \textit{lang}. Semua halaman pada aplikasi \textit{SharIF Judge} memakai bahasa Inggris, jika elemen \textit{html} memiliki atribut \textit{lang} maka kriteria ini akan terpenuhi.

\paragraph{Kriteria Sukses 3.1.3 \textit{Unusual Words}}
\label{subsubsec:kepatuhan_kriteria_3.1.3}
(Sukses) \\

Kriteria ini sukses dipatuhi karena pada aplikasi \textit{SharIF Judge} tidak ada kata atau frasa tertentu yang digunakan dengan cara tidak biasa atau terbatas. 

\paragraph{Kriteria Sukses 3.1.4 \textit{Abbreviations}}
\label{subsubsec:kepatuhan_kriteria_3.1.4}
(Sukses) \\

Kriteria ini sukses dipatuhi karena singkatan yang ada pada \textit{SharIF Judge} lazim digunakan oleh kalangan informatika, singkatan yang dimaksud adalah IP, PDF.

\paragraph{Kriteria Sukses 3.1.5 \textit{Reading Level}}
\label{subsubsec:kepatuhan_kriteria_3.1.5}
(Sukses) \\

Kriteria ini sukses dipatuhi karena pada aplikasi \textit{SharIF Judge} tidak ada teks yang membutuhkan kemampuan membaca lebih maju daripada tingkat pendidikan menengah bawah.

\paragraph{Kriteria Sukses 3.1.6 \textit{Pronunciation}}
\label{subsubsec:kepatuhan_kriteria_3.1.6}
(Sukses) \\

Kriteria ini sukses dipatuhi karena pada halaman aplikasi \textit{SharIF Judge} setiap kata dapat dimengerti artinya tanpa pengguna perlu mengetahui cara mengucapkan kata tersebut.

\subsubsection{\textit{Predictable}}
\label{subsubsec:predictable}

Pada subbab ini dibahas kriteria sukses yang termasuk kedalam kategori \textit{predictable}. Kategori ini mensyaratkan agar halaman web ditampilkan dan dapat dioperasikan dengan cara yang mudah ditebak.

\paragraph{Kriteria Sukses 3.2.1 \textit{On Focus}}
\label{subsubsec:kepatuhan_kriteria_3.2.1}
(Sukses) \\

Kriteria ini sukses dipatuhi karena pada halaman aplikasi \textit{SharIF Judge} setiap komponen antarmuka yang menerima fokus tidak menyebabkan perubahan konteks.

\paragraph{Kriteria Sukses 3.2.2 \textit{On Input}}
\label{subsubsec:kepatuhan_kriteria_3.2.2}
(Sukses) \\

Kriteria ini sukses dipatuhi karena pada setiap pengguna mengubah setelan komponen antarmuka tidak menyebabkan perubahan konteks.

\paragraph{Kriteria Sukses 3.2.3 \textit{Consistent Navigation}}
\label{subsubsec:kepatuhan_kriteria_3.2.3}
(Sukses) \\

Kriteria ini sukses dipatuhi karena bagian navigasi menu yang muncul berulang pada tiap halaman aplikasi \textit{SharIF Judge}, muncul dalam urutan relatif yang sama setiap kali terlihat.

\paragraph{Kriteria Sukses 3.2.4 \textit{Consistent Identification}}
\label{subsubsec:kepatuhan_kriteria_3.2.4}
(Tidak Sukses) \\

Kriteria ini tidak sukses dipatuhi karena ada komponen yang memiliki fungsi yang sama tetapi tidak diidentifikasi secara konsisten. Komponen tersebut antara lain:

\begin{itemize}
	\item Pada halaman \textit{Assignments} istilah untuk menambah \textit{Assignment} diberi nama \textit{Add}. Tampilan halaman web dapat dilihat pada gambar \ref{fig:kepatuhan_3_2_4_assignments}.
	\begin{figure}[H]
		\centering  
		\includegraphics[scale=0.5]{kepatuhan_3_2_4_assignments}  
		\caption[Pelanggaran Kriteria Sukses 3.2.4 - \textit{Assignments}]{Pelanggaran Kriteria Sukses 3.2.4 - \textit{Assignments}} 
		\label{fig:kepatuhan_3_2_4_assignments} 
	\end{figure}

	\item Pada halaman \textit{Users} istilah untuk menambah \textit{user} diberi nama \textit{Add User}. Tampilan halaman web dapat dilihat pada gambar \ref{fig:kepatuhan_3_2_4_users}.
	\begin{figure}[H]
		\centering  
		\includegraphics[scale=0.5]{kepatuhan_3_2_4_users}  
		\caption[Pelanggaran Kriteria Sukses 3.2.4 - \textit{Users}]{Pelanggaran Kriteria Sukses 3.2.4 - \textit{Users}} 
		\label{fig:kepatuhan_3_2_4_users} 
	\end{figure}

	\item Pada halaman. \textit{Notifications} istilah untuk menambah \textit{Notification} diberi nama \textit{New}. Tampilan halaman web dapat dilihat pada gambar \ref{fig:kepatuhan_3_2_4_notifications}.
	\begin{figure}[H]
		\centering  
		\includegraphics[scale=0.5]{kepatuhan_3_2_4_notifications}  
		\caption[Pelanggaran Kriteria Sukses 3.2.4 - \textit{Notifications}]{Pelanggaran Kriteria Sukses 3.2.4 - \textit{Notifications}} 
		\label{fig:kepatuhan_3_2_4_notifications} 
	\end{figure}

\end{itemize}

\paragraph{Kriteria Sukses 3.2.5 \textit{Change on Request}}
\label{subsubsec:kepatuhan_kriteria_3.2.5}
(Sukses) \\

Kriteria ini sukses dipatuhi karena perubahan konteks hanya terjadi bila dilakukan oleh pengguna.

\subsubsection{\textit{Input Assistance}}
\label{subsubsec:input_assistance}

Pada subbab ini dibahas kriteria sukses yang termasuk kedalam kategori \textit{input assistance}. Kategori ini mensyaratkan agar tersedianya bantuan untuk pengguna menghindari kesalahan dan mengoreksi kesalahan tersebut.

\paragraph{Kriteria Sukses 3.3.1 \textit{Error Identification}}
\label{subsubsec:kepatuhan_kriteria_3.3.1}
(Sukses)\\

Kriteria ini sukses dipatuhi karena kesalahan masukan terdeteksi secara otomatis, \textit{item} yang salah diidentifikasi dan kesalahan tersebut dijelaskan kepada pengguna dalam teks.

\paragraph{Kriteria Sukses 3.3.2 \textit{Labels or Instructions}}
\label{subsubsec:kepatuhan_kriteria_3.3.2}
(Tidak Sukses) \\

Kriteria ini tidak sukses karena pada halaman problem bagian edit html, edit markdown, dan edit plain html tidak ada label atau instruksi untuk pengisiannya. Salah satu contoh tampilan halaman web dapat dilihat pada gambar \ref{fig:kepatuhan_3_3_2_edit_markdown}.
\begin{figure}[H]
	\centering  
	\includegraphics[scale=0.25]{kepatuhan_3_3_2_edit_markdown}  
	\caption[Pelanggaran Kriteria Sukses 3.3.2 - \textit{Edit Markdown} Tidak Diberi Label Atau Instruksi]{Pelanggaran Kriteria Sukses 3.3.2 - \textit{Edit Markdown} Tidak Diberi Label Atau Instruksi} 
	\label{fig:kepatuhan_3_3_2_edit_markdown} 
\end{figure}

\paragraph{Kriteria Sukses 3.3.3 \textit{Error Suggestion}}
\label{subsubsec:kepatuhan_kriteria_3.3.3}
(Sukses) \\

Kriteria ini sukses dipatuhi karena ketika kesalahan masukan terdeteksi secara otomatis dan saran untuk mengoreksi diketahui, maka saran tersebut diberikan kepada pengguna.

\paragraph{Kriteria Sukses 3.3.4 \textit{Error Prevention (Legal, Financial, Data)}}
\label{subsubsec:kepatuhan_kriteria_3.3.4}
(Sukses) \\

Kriteria ini sukses dipatuhi karena pada halaman yang mengirim tanggapan pengguna, data yang dimasukkan oleh pengguna diperiksa terkait kesalahan masukan dan pengguna dipersilakan untuk mengoreksinya.

\paragraph{Kriteria Sukses 3.3.5 \textit{Help}}
\label{subsubsec:kepatuhan_kriteria_3.3.5}
(Tidak Sukses) \\

Kriteria ini tidak sukses dipatuhi karena
pada aplikasi \textit{SharIF Judge} masih ada konten yang tidak diberi label. Pengguna harus dapat mengetahui tujuan dari konten melalui labelnya.

\paragraph{Kriteria Sukses 3.3.6 \textit{Error Prevention (All)}}
\label{subsubsec:kepatuhan_kriteria_3.3.6}
(Sukses) \\

Kriteria ini sukses dipatuhi karena pada halaman yang mengharuskan pengguna untuk mengirimkan informasi, data yang dimasukkan oleh pengguna diperiksa terkait kesalahan masukan dan pengguna dipersilakan untuk mengoreksinya.

\subsection{\textit{Robust}}
\label{subsec:kepatuhan_robust}

Berikut adalah kepatuhan \textit{SharIF Judge} berdasarkan prinsip \textit{robust}.

\subsubsection{\textit{Compatible}}
\label{subsubsec:compatible}

Pada subbab ini dibahas kriteria sukses yang termasuk kedalam kategori \textit{compatible}. Kategori ini mensyaratkan agar konten yang dibuat memiliki kompatibilitas dengan \textit{user agent} saat ini maupun saat yang akan datang, termasuk teknologi alat bantu.

\paragraph{Kriteria Sukses 4.1.1 \textit{Parsing}}
\label{subsubsec:kepatuhan_kriteria_4.1.1}
(Tidak Sukses)\\

Kriteria ini tidak sukses dipatuhi karena pada aplikasi \textit{SharIF Judge} masih banyak elemen yang tidak memiliki tag awal dan akhir yang lengkap, elemen tidak disusun berlapis sesuai dengan spesifikasinya, elemen mengandung atribut duplikat, dan ada ID yang unik. Pada potongan kode \ref{ls_kepatuhan_4_1_1} ditunjukkan bahwa masih ada elemen yang memiliki atribut yang duplikat.
\begin{lstlisting}[basicstyle=\ttfamily, frame=single,
columns=fullflexible, keepspaces=true, breaklines=true, label=ls_kepatuhan_4_1_1, caption=Pelanggaran Kriteria Sukses 4.1.1 - Elemen Memiliki Atribut Duplikat]
...
<input id="form_extra_time" type="text" name="extra_time" id="extra_time" class="sharif_input medium" value="{{ edit ? edit_assignment.extra_time|extra_time_formatter : set_value('extra_time') }}" />
...
\end{lstlisting}

\paragraph{Kriteria Sukses 4.1.2 \textit{Name, Role, Value}}
\label{subsubsec:kepatuhan_kriteria_4.1.2}
(Sukses) \\

Kriteria ini sukses dipatuhi karena program yang dibuat menggunakan standar \textit{HTML} sudah memenuhi kriteria ini jika digunakan sesuai spesifikasi.

\paragraph{Kriteria Sukses 4.1.3 \textit{Status Messages}}
\label{subsubsec:kepatuhan_kriteria_4.1.3}
(Tidak Sukses) \\

Kriteria ini tidak sukses dipatuhi karena pada aplikasi \textit{SharIF Judge} tidak ada notifikasi saat melakukan aksi.

\section{Peningkatan ke \textit{Level A}}
\label{sec:peningkatan_level_A}

Pada bagian ini dijelaskan secara singkat untuk meningkatkan kepatuhan aplikasi \textit{SharIF Judge} menjadi \textit{level A} dengan cara memenuhi 9 kriteria sukses \textit{level A} yang belum terpenuhi dari total 30 kriteria sukses \textit{level A}.

\subsection{Kriteria Sukses 1.1.1 \textit{Non-text Content}}
\label{subsec:peningkatan_A_1.1.1}

Kriteria ini dapat dipatuhi dengan cara memberikan alternatif teks untuk konten \textit{non-text} yang memberikan informasi yang sama seperti konten \textit{non-text}. Konten yang dimaksud adalah:

\begin{itemize}
	\item Logo \textit{SharIF Judge} pada bagian menu atas. Listing dapat dilihat pada \ref{ls_kepatuhan_1_1_1_logo_sharif_judge}.
	\item Gambar \textit{PDF} pada halaman \textit{Assignment}. Listing dapat dilihat pada \ref{ls_kepatuhan_1_1_1_logo_pdf}.
\end{itemize}

\subsection{Kriteria Sukses 1.3.1 \textit{Info and Relationships}}
\label{subsec:peningkatan_A_1.3.1}

Kriteria ini dapat dipatuhi dengan cara memberikan label yang sesuai pada elemen setiap form masukan. Elemen yang dimaksud adalah:

\begin{itemize}
	\item Semua elemen bagian tabel \textit{Problems} pada halaman \textit{Add Assignment}.
	\item \textit{Checkbox} pada halaman \textit{Add Users}.
	\item \textit{Textarea} pada halaman \textit{Add Users}.
	\item \textit{Textarea} pada halaman \textit{Edit Problem Markdown}.
	\item \textit{Textarea} pada halaman \textit{Edit Problem Plain HTML}.
	\item \textit{Dropdown} pada halaman \textit{Problems}.
	\item Masukan \textit{Upload File} pada halaman \textit{Problems}.
	\item Pemakaian \textit{header} pada setiap judul halaman.
\end{itemize}

\subsection{Kriteria Sukses 2.1.1 \textit{Keyboard}}
\label{subsec:peningkatan_A_2.1.1}

Kriteria ini dapat dipatuhi dengan cara memastikan semua fungsionalitas konten dapat dioperasikan dengan keyboard. Elemen yang harus diperbaiki adalah:

\begin{itemize}
	\item Menu \textit{Tools} pada bagian menu atas.
	\item Menu pada logo \textit{Profile} pada bagian menu atas.
	\item Memilih \textit{Assignment} pada bagian menu atas.
	\item Aksi \textit{Delete User} dan \textit{Delete Submissions} pada halaman \textit{Users}.
	\item \textit{Checkbox} untuk memilih \textit{Assignment} pada halaman \textit{Assignments}.
	\item \textit{Add} dan \textit{Delete Problems} pada halaman \textit{Add Assignment}.
	\item Memilih \textit{Submission} pada halaman \textit{All Submissions}.
	\item Melihat kode pada halaman \textit{Submission} dan \textit{Final Submission}.
	\item Melihat status pada halaman \textit{Submission} dan \textit{Final Submission}.
	\item Melihat \textit{log} pada halaman \textit{Submission} dan \textit{Final Submission}.
	\item Aksi \textit{rejudge} pada halaman \textit{Submission} dan \textit{Final Submission}.
\end{itemize}

\subsection{Kriteria Sukses 2.1.2 \textit{No Keyboard Trap}}
\label{subsec:peningkatan_A_2.1.2}
Kriteria ini dapat dipatuhi dengan cara memastikan tidak ada \textit{keyboard trap} di semua halaman. Elemen yang harus diperbaiki adalah:

\begin{itemize}
	\item Seluruh \textit{Textarea} pada halaman \textit{Settings}.
	\item \textit{Textarea Participants} dan \textit{Coefficient rule} pada halaman \textit{Add Assignment}.
	\item \textit{Textarea} pada halaman \textit{Edit Problem Markdown}.
\end{itemize}

\subsection{Kriteria Sukses 2.4.1 \textit{Bypass Blocks}}
\label{subsec:peningkatan_A_2.4.1}

Kriteria ini dapat dipatuhi dengan cara menambahkan mekanisme untuk meloncati menu navigasi ketika pengguna bernavigasi menggunakan \textit{keyboard}. Dengan cara ini pengguna dapat langsung menuju isi konten tanpa harus melewati navigasi menu.

\subsection{Kriteria Sukses 2.4.4 \textit{Link Purpose (In Context)}}
\label{subsec:peningkatan_A_2.4.4}

Kriteria ini dapat dipatuhi dengan cara menambahkan alternatif teks pada tautan yang menjelaskan tujuan tautan tersebut. Tautan yang dimaksud adalah:

\begin{itemize}
	\item Seluruh tautan pada \textit{sidebar}.
	\item Tautan \textit{Profile} pada bagian menu atas.
	\item Tautan \textit{PDF} pada halaman \textit{Assignment}.
	\item Tautan pada halaman \textit{Add Assignment} bagian \textit{Problems}.
\end{itemize}

\subsection{Kriteria Sukses 3.1.1 \textit{Language of Page}}
\label{subsec:peningkatan_A_3.1.1}

Kriteria ini dapat dipatuhi dengan cara memberikan atribut \textit{lang} pada setiap halaman untuk menunjukkan bahasa yang digunakan pada halaman tersebut.

\subsection{Kriteria Sukses 3.3.2 \textit{Labels or Instructions}}
\label{subsec:peningkatan_A_3.3.2}

Kriteria ini dapat dipatuhi dengan cara memberikan label yang sesuai pada setiap bidang masukan agar pengguna dapat mengerti maksud dan tujuan bidang masukan tersebut. Bidang masukan yang dimaksud adalah:

\begin{itemize}
	\item Semua elemen bagian tabel \textit{Problems} pada halaman \textit{Add Assignment}.
	\item \textit{Checkbox} pada halaman \textit{Add Users}.
	\item \textit{Textarea} pada halaman \textit{Add Users}.
	\item \textit{Textarea} pada halaman \textit{Edit Problem Markdown}.
	\item \textit{Textarea} pada halaman \textit{Edit Problem Plain HTML}.
	\item \textit{Dropdown} pada halaman \textit{Problems}.
	\item Masukan \textit{Upload File} pada halaman \textit{Problems}.
\end{itemize}

\subsection{Kriteria Sukses 4.1.1 \textit{Parsing}}
\label{subsec:peningkatan_A_4.1.1}

Kriteria ini dapat dipatuhi dengan cara memastikan setiap elemen memiliki \textit{tag} buka dan tutup yang lengkap, elemen disusun berlapis sesuai spesifikasinya, elemen tidak memiliki atribut duplikat, dan setiap elemen memiliki ID yang unik. Elemen yang belum memenuhi kriteria ini adalah:

\begin{itemize}
	\item \textit{Form} masukan \textit{Extra Time} pada halaman \textit{Add Assignment}.
\end{itemize}
