%versi 2 (8-10-2016) 
\chapter{Pendahuluan}
\label{chap:intro}
   
\section{Latar Belakang}
\label{sec:label}

-Internet sudah dapat diakses dengan mudah pada masa ini sehingga hampir setiap orang dapat menggunakan internet.

-Mahasiswa UNPAR yang beragam.

-Teknik Informatika UNPAR menggunakan aplikasi SharIF Judge untuk memeriksa tugas coding mahasiswanya khususnya yang menggunakan bahasa Java.

-Membuat aplikasi SharIF Judge dapat diakses oleh banyak kalangan.


\dtext{5-10}

\section{Rumusan Masalah}
\label{sec:rumusan}
\begin{itemize}
	\item Mengapa Web Content Accessibility Guideline pada aplikasi SharIF Judge perlu ditingkatkan ?
	\item Bagaimana tingkat kepatuhan Web Content Accessibility Guideline pada aplikasi SharIF Judge ?
\end{itemize}

\section{Tujuan}
\label{sec:tujuan}
\begin{itemize}
	\item Membuat aplikasi SharIF Judge dapat digunakan oleh banyak kalangan.
	\item Menganalisis dan meningkatkan Web Content Accessibility Guideline pada aplikasi SharIF Judge.
\end{itemize} 

\section{Batasan Masalah}
\label{sec:batasan}
Untuk mempermudah pembuatan template ini, tentu ada hal-hal yang harus dibatasi, misalnya saja bahwa template ini bukan berupa style \LaTeX{} pada umumnya (dengan alasannya karena belum mampu jika diminta membuat seperti itu)

\dtext{8}

\section{Metodologi}
\label{sec:metlit}
Tentunya akan diisi dengan metodologi yang serius sehingga templatenya terkesan lebih serius.

\dtext{9}

\section{Sistematika Pembahasan}
\label{sec:sispem}
Rencananya Bab 2 akan berisi petunjuk penggunaan template dan dasar-dasar \LaTeX.
Mungkin bab 3,4,5 dapt diisi oleh ketiga jurusan, misalnya peraturan dasar skripsi atau pedoman penulisan, tentu jika berkenan.
Bab 6 akan diisi dengan kesimpulan, bahwa membuat template ini ternyata sungguh menghabiskan banyak waktu.

\dtext{10}