%versi 2 (8-10-2016) 
\chapter{Pendahuluan}
\label{chap:intro}
   
\section{Latar Belakang}
\label{sec:label}
SharIF Judge \cite{SharIF_Judge} adalah sebuah aplikasi gratis dan \textit{open source} untuk menilai code berbahasa C , C++, Java dan Python. SharIF Judge adalah pencabangan dari Sharif Judge yang telah dibuat oleh Mohammed Javad Naderi. Versi dari pencabangan ini memuat fitur baru yang diperlukan oleh jurusan teknik informatika UNPAR. Aplikasi ini dibuat menggunakan PHP (\textit{CodeIgnitor framework}) dan bagian backendnya dibuat dengan BASH.

\textit{Web Content Accessibility Guidelines} (\textit{WCAG}) 2.1 \cite{WCAG:2.1} memuat rekomendasi untuk membuat konten web lebih mudah diakses. Pedoman-pedoman ini akan membuat konten lebih mudah diakses untuk orang disabilitas termasuk akomodasi untuk kebutaan dan penglihatan rendah, ketulian dan gangguan pendengaran, gerakan terbatas, fotosensitif, atau kombinasinya, dan beberapa akomomodasi untuk kesulitan belajar dan keterbatasan kognitif; tetapi tidak akan memenuhi setiap kebutuhan pengguna dengan disabilitas. Di dalam \textit{WCAG} 2.1 ada 78 kriteria sukses. Kriteria sukses adalah pedoman untuk membuat konten lebih mudah diakses. Ada 3 tingkat kepatuhan yaitu A (terkecil), AA, AAA (terbesar). Tingkat kepatuhan A adalah tingkat kepatuhan terkecil yang diperoleh jika seluruh kriteria sukses tingkat A terpenuhi atau versi alternatifnya tersedia. Tingkat kepatuhan AA adalah tingkat kepatuhan yang diperoleh jika seluruh kriteria sukses tingkat A dan AA terpenuhi atau versi alternatif tingkat AA tersedia. Tingkat kepatuhan AAA adalah tingkat kepatuhan yang diperoleh jika seluruh kriteria sukses tingkat A, AA, dan AAA terpenuhi atau veri alternatif tingkat AAA tersedia.

Pada skripsi ini, akan dilakukan analisis tingkat kepatuhan dan rekomendasi perbaikan aplikasi SharIF Judge berdasarkan \textit{Web Content Accessibility Guideline} 2.1. Selain itu, aplikasi SharIF Judge juga akan diuji dengan beberapa kondisi keterbatasan seperti keterbatasan visual, keterbatasan gerak, keterbatasan pendengaran. Dengan perbaikan ini diharapkan aplikasi SharIF Judge dapat diakses oleh banyak kalangan.

\section{Rumusan Masalah}
\label{sec:rumusan}
\begin{itemize}
	\item Bagaimana tingkat kepatuhan SharIF Judge terhadap \textit{WCAG} 2.1 ?
	\item Rekomendasi apa saja yang perlu dilakukan terhadap SharIF Judge untuk menaikkan level kepatuhannya ?
\end{itemize}

\section{Tujuan}
\label{sec:tujuan}
\begin{itemize}
	\item Mengetahui tingkat kepatuhan SharIF Judge terhadap WCAG 2.1.
	\item Membuat rekomendasi yang perlu dilakukan terhadap SharIF Judge untuk menaikkan level kepatuhannya.
\end{itemize} 

\section{Batasan Masalah}
\label{sec:batasan}
Batasan masalah pada skripsi ini adalah sebagai berikut :

\begin{enumerate}
	\item content
\end{enumerate}

\section{Metodologi}
\label{sec:metlit}
Metodologi yang dilakukan pada skripsi ini adalah sebagai berikut :

\begin{enumerate}
	\item Studi literatur mengenai \textit{WCAG} 2.1 dan SharIF Judge
	\item Mengukur tingkat kepatuhan SharIF Judge terhadap \textit{WCAG} 2.1
	\item Memberikan rekomendasi perbaikan pada setiap kriteria kesuksessan.
	\item Mengimplementasikan rekomendasi perbaikan.
	\item Menguji hasil perbaikan.
\end{enumerate}

\section{Sistematika Pembahasan}
\label{sec:sispem}
Setiap bab dalam skripsi ini memiliki sistematika penulisan ke dalam poin-poin sebagai berikut :

\begin{enumerate}
	\item Bab 1: Pendahuluan, akan membahas gambaran umum dari skripsi ini. Bab ini berisi latar
	belakang, rumusan masalah, tujuan, batasan masalah, metode penelitian, dan sistematika
	pembahasan.
	\item Bab 2: Landasan Teori, akan membahas dasar teori yang menjadi acuan dalam pembuatan
	skripsi ini. Dasar teori yang digunakan yaitu \textit{WCAG} 2.1 dan SharIF Judge.
	\item Bab 3: Analisis, akan membahas hasil analisis mengenai tingkat kepatuhan situs web SharIF Judge terhadap \textit{WCAG} 2.1.

	\item Bab 4: Perancangan, akan membahas mengenai perubahan-perubahan yang dapat dilakukan
	untuk meningkatkan kepatuhan situs web SharIF Judge terhadap \textit{WCAG} 2.1.
	\item Bab 5: Implementasi dan Pengujian, akan membahas hasil implementasi dan pengujian yang
	telah dilakukan pada situs web SharIF Judge.
	\item Bab 6: Kesimpulan dan saran, akan berisi kesimpulan dari hasil penelitian yang telah dilakukan
	dan saran yang dapat diberikan untuk penelitian berikutnya.
\end{enumerate}