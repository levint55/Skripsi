%versi 2 (8-10-2016) 
\chapter{Pendahuluan}
\label{chap:intro}
   
\section{Latar Belakang}
\label{sec:label}
\textit{SharIF Judge} \cite{SharIF_Judge} adalah sebuah aplikasi gratis dan \textit{open source} untuk menilai code berbahasa \textit{C} , \textit{C++}, \textit{Java} dan \textit{Python}. \textit{SharIF Judge} adalah pencabangan dari \textit{Sharif Judge} \cite{Sharif_Judge_Original} yang telah dibuat oleh Mohammed Javad Naderi. Versi dari pencabangan ini memuat fitur baru yang diperlukan oleh jurusan teknik informatika UNPAR. Aplikasi ini dibuat menggunakan PHP (\textit{CodeIgnitor framework}) dan bagian backendnya dibuat dengan BASH.

Pada zaman perkembangan teknologi saat ini, banyak sekali orang yang mengakses teknologi. Beberapa diantara mereka memiliki disabilitas. Masih banyak web yang masih belum dapat diakses oleh kalangan disabilitas. Menurut Brian Sierkowski, aksesibilitas web penting karena beberapa hal. Pertama, semua orang memiliki hak untuk mengakses internet sehingga akses web harus dapat diakses oleh berbagai kalangan. Kedua, dengan membuat sebuah web mudah diakses maka pengguna web tersebut akan bertambah.

Sejumlah individu dan organisasi dari berbagai negara bekerja sama membuat standar untuk \textit{Web content accessibility} yang memenuhi kebutuhan individu, organisasi, dan pemerintah secara internasional. Melalui perundingan tersebut terbentuklah \textit{Web Content Accessibility Guidelines} (\textit{WCAG}) 2.1 \cite{WCAG:2.1} yang memuat rekomendasi untuk membuat konten web lebih mudah diakses. Pedoman-pedoman ini akan membuat konten lebih mudah diakses untuk orang disabilitas termasuk akomodasi untuk kebutaan dan penglihatan rendah, ketulian dan gangguan pendengaran, gerakan terbatas, fotosensitif, atau kombinasinya, dan beberapa akomomodasi untuk kesulitan belajar dan keterbatasan kognitif; tetapi tidak akan memenuhi setiap kebutuhan pengguna dengan disabilitas. 

Di dalam \textit{WCAG} 2.1 ada 78 kriteria sukses. Kriteria sukses adalah pedoman untuk membuat konten lebih mudah diakses. Ada 3 tingkat kepatuhan yaitu A (terkecil), AA, AAA (terbesar). Tingkat kepatuhan A adalah tingkat kepatuhan terkecil yang diperoleh jika seluruh kriteria sukses tingkat A terpenuhi atau versi alternatifnya tersedia. Tingkat kepatuhan AA adalah tingkat kepatuhan yang diperoleh jika seluruh kriteria sukses tingkat A dan AA terpenuhi atau versi alternatif tingkat AA tersedia. Tingkat kepatuhan AAA adalah tingkat kepatuhan yang diperoleh jika seluruh kriteria sukses tingkat A, AA, dan AAA terpenuhi atau veri alternatif tingkat AAA tersedia.

Pada skripsi ini, akan dilakukan analisis tingkat kepatuhan dan rekomendasi perbaikan aplikasi \textit{SharIF Judge} berdasarkan \textit{Web Content Accessibility Guideline} 2.1. Selain itu, aplikasi \textit{SharIF Judge} juga akan diuji dengan kondisi keterbatasan visual. Dengan perbaikan ini diharapkan aplikasi \textit{SharIF Judge} dapat diakses oleh banyak kalangan.

\section{Rumusan Masalah}
\label{sec:rumusan}
\begin{itemize}
	\item Bagaimana tingkat kepatuhan \textit{SharIF Judge} terhadap \textit{WCAG} 2.1 ?
	\item Rekomendasi perbaikan apa saja yang perlu dilakukan terhadap \textit{SharIF Judge} untuk menaikkan tingkat kepatuhannya ?
\end{itemize}

\section{Tujuan}
\label{sec:tujuan}
\begin{itemize}
	\item Mengetahui tingkat kepatuhan \textit{SharIF Judge} terhadap \textit{WCAG} 2.1.
	\item Membuat rekomendasi perbaikan yang perlu dilakukan terhadap \textit{SharIF Judge} untuk menaikkan tingkat kepatuhannya.
\end{itemize} 

\section{Batasan Masalah}
\label{sec:batasan}
Batasan masalah pada skripsi ini adalah sebagai berikut:

\begin{enumerate}
	\item Pengujian hanya dilakukan dalam kondisi keterbatasan visual dan perangkat yang digunakan adalah komputer.
	\item Peningkatan tingkat kepatuhan hanya dilakukan sampai tingkat A.
	\item Untuk analisis kriteria sukses 1.4.11 tidak dapat dipenuhi karena sulit dilakukan dan tidak ada \textit{tools} yang mendukung.
\end{enumerate}

\section{Metodologi}
\label{sec:metlit}
Metodologi yang dilakukan pada skripsi ini adalah sebagai berikut:

\begin{enumerate}
	\item Studi literatur mengenai \textit{WCAG} 2.1 dan \textit{SharIF Judge}
	\item Mengukur tingkat kepatuhan \textit{SharIF Judge} terhadap \textit{WCAG} 2.1
	\item Memberikan rekomendasi perbaikan pada setiap kriteria kesuksessan.
	\item Mengimplementasikan rekomendasi perbaikan.
	\item Menguji hasil perbaikan.
\end{enumerate}

\section{Sistematika Pembahasan}
\label{sec:sispem}
Setiap bab dalam skripsi ini memiliki sistematika penulisan ke dalam poin-poin sebagai berikut:

\begin{enumerate}
	\item Bab 1: Pendahuluan, akan membahas gambaran umum dari skripsi ini. Bab ini berisi latar
	belakang, rumusan masalah, tujuan, batasan masalah, metode penelitian, dan sistematika
	pembahasan.
	\item Bab 2: Landasan Teori, akan membahas dasar teori yang menjadi acuan dalam pembuatan
	skripsi ini. Dasar teori yang digunakan yaitu \textit{WCAG} 2.1 dan \textit{SharIF Judge}.
	\item Bab 3: Analisis, akan membahas hasil analisis mengenai tingkat kepatuhan situs web \textit{SharIF Judge} terhadap \textit{WCAG} 2.1 dan perubahan yang dapat dilakukan untuk mentingkatkan kepatuhan situs web \textit{SharIF Judge} terhadap \textit{WCAG} 2.1.
	\item Bab 4: Implementasi dan Pengujian, akan membahas hasil implementasi dan pengujian yang
	telah dilakukan pada situs web \textit{SharIF Judge}.
	\item Bab 5: Kesimpulan dan saran, akan berisi kesimpulan dari hasil penelitian yang telah dilakukan dan saran yang dapat diberikan untuk penelitian berikutnya.
\end{enumerate}