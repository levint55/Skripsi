%versi 2 (8-10-2016) 
\chapter{Pendahuluan}
\label{chap:intro}
   
\section{Latar Belakang}
\label{sec:label}
SharIF Judge adalah sebuah aplikasi gratis dan \textit{open source} untuk menilai code berbahasa C , C++, Java dan Python. SharIF Judge adalah pencabangan dari Sharif Judge yang telah dibuat oleh Mohammed Javad Naderi. Versi dari pencabangan ini memuat fitur baru yang diperlukan oleh jurusan teknik informatika UNPAR. Aplikasi ini dibuat menggunakan PHP (\textit{CodeIgnitor framework}) dan bagian backendnya dibuat dengan BASH.

Pada skripsi ini, akan dilakukan analisis dan rekomendasi perbaikan aplikasi SharIF Judge berdasarkan \textit{Web Content Accessibility Guideline} 2.1. WCAG memuat rekomendasi untuk membuat konten \textit{web} lebih mudah diakses. Pedoman - pedoman ini akan membuat konten lebih mudah diakses untuk orang disabilitas termasuk akomodasi untuk kebutaan dan penglihatan rendah, ketulian dan gangguan pendengaran, gerakan terbatas, photosensivitas, atau kombinasinya, dan beberapa akomomodasi untuk kesulitan belajar dan keterbatasan kognitif; tetapi tidak akan memenuhi setiap kebutuhan pengguna dengan disabilitas. Dengan perbaikan ini diharapkan aplikasi SharIF Judge dapat diakses oleh banyak kalangan.

\section{Rumusan Masalah}
\label{sec:rumusan}
\begin{itemize}
	\item Bagaimana tingkat kepatuhan SharIF Judge terhadap WCAG 2.1 ?
	\item Rekomendasi apa saja yang perlu dilakukan terhadap SharIF Judge untuk menaikkan level kepatuhannya ?
\end{itemize}

\section{Tujuan}
\label{sec:tujuan}
\begin{itemize}
	\item Mengetahui tingkat kepatuhan SharIF Judge terhadap WCAG 2.1.
	\item Membuat rekomendasi yang perlu dilakukan terhadap SharIF Judge untuk menaikkan level kepatuhannya.
\end{itemize} 

\section{Batasan Masalah}
\label{sec:batasan}
Untuk mempermudah pembuatan template ini, tentu ada hal-hal yang harus dibatasi, misalnya saja bahwa template ini bukan berupa style \LaTeX{} pada umumnya (dengan alasannya karena belum mampu jika diminta membuat seperti itu)

\dtext{8}

\section{Metodologi}
\label{sec:metlit}
Tentunya akan diisi dengan metodologi yang serius sehingga templatenya terkesan lebih serius.

\dtext{9}

\section{Sistematika Pembahasan}
\label{sec:sispem}
Rencananya Bab 2 akan berisi petunjuk penggunaan template dan dasar-dasar \LaTeX.
Mungkin bab 3,4,5 dapt diisi oleh ketiga jurusan, misalnya peraturan dasar skripsi atau pedoman penulisan, tentu jika berkenan.
Bab 6 akan diisi dengan kesimpulan, bahwa membuat template ini ternyata sungguh menghabiskan banyak waktu.

\dtext{10}