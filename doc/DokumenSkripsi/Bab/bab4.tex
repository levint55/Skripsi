\chapter{Implementasi dan Pengujian}
\label{chap:implementasi}

Pada Bab ini akan dijelaskan bagaimana implementasi untuk menaikkan tingkat kepatuhan aplikasi \textit{SharIF Judge}.

\section{Implementasi}
\label{sec:implementasi}

Pada bagian ini akan dibahas perbaikan apa saja yang dilakukan untuk membuat kriteria sukses pada bagian 3.1
yang belum sukses dipatuhi menjadi sukses dipatuhi. Setiap perubahan akan ditampilkan dalam format \textit{diff} dengan jarak \textit{identation} sebanyak empat spasi.

\subsection{Implementasi Kriteria Sukses 1.1.1 Non-text Content}
\label{subsec:implementasi_A_1.1.1}

Berikut adalah perubahan yang perlu dilakukan untuk memenuhi Kriteria Sukses 1.1.1:

\begin{itemize}
	\item Pada bagian menu atas logo \textit{SharIF Judge} tidak memiliki alternatif teks yang menjelaskan gambar dari logo tersebut sehingga perlu ditambahkan sebuah label untuk menjelaskan gambar tersebut. Perubahan yang terjadi ada pada \textit{file} \textit{/application/views/templates/top\_bar.twig}.

	\item Pada halaman \textit{Assignments} terdapat gambar \textit{PDF} di daftar \textit{Assignment} yang tidak memiliki alternatif teks yang menjelaskan gambar tersebut sehingga perlu ditambahkan sebuah label untuk menjelaskan gambar tersebut. Perubahan yang terjadi ada pada \textit{file} \textit{/application/views/pages/assignments.twig}.
	
	Semua perubahan kode dapat dilihat pada \ref{lst_1.1.1}.
	
\begin{lstlisting}[language=diff, caption=Perubahan untuk mematuhi kriteria 1.1.1, label=lst_1.1.1, basicstyle=\ttfamily, frame=single,
columns=fullflexible, keepspaces=true, breaklines=true]
diff --git a/application/views/pages/admin/add_assignment.twig b/application/views/pages/admin/add_assignment.twig
index 13a0ddea..aa9fc435 100644
--- a/application/views/pages/admin/add_assignment.twig
+++ b/application/views/pages/admin/add_assignment.twig
@@ -187,10 +187,10 @@
<th rowspan="2">Score</th>
<th colspan="3" style="border-bottom: 1px solid #BDBDBD">Time Limit (ms)</th>
<th rowspan="2">Memory<br>Limit (kB)</th>
-    <th rowspan="2">Allowed<br>Languages (<a target="_blank" href="https://github.com/ifunpar/Sharif-Judge/blob/docs/v1.4/add_assignment.md#allowed-languages">?</a>)</th>
-    <th rowspan="2">Diff<br>Command (<a target="_blank" href="https://github.com/ifunpar/Sharif-Judge/blob/docs/v1.4/add_assignment.md#diff-command">?</a>)</th>
-    <th rowspan="2">Diff<br>Argument (<a target="_blank" href="https://github.com/ifunpar/Sharif-Judge/blob/docs/v1.4/add_assignment.md#diff-arguments">?</a>)</th>
-    <th rowspan="2">Upload<br>Only (<a target="_blank" href="https://github.com/ifunpar/Sharif-Judge/blob/docs/v1.4/add_assignment.md#upload-only">?</a>)</th>
+    <th rowspan="2">Allowed<br>Languages (<a aria-label="Link Help For Languages" target="_blank" href="https://github.com/ifunpar/Sharif-Judge/blob/docs/v1.4/add_assignment.md#allowed-languages">?</a>)</th>
+    <th rowspan="2">Diff<br>Command (<a aria-label="Link Help For Diff Command" target="_blank" href="https://github.com/ifunpar/Sharif-Judge/blob/docs/v1.4/add_assignment.md#diff-command">?</a>)</th>
+    <th rowspan="2">Diff<br>Argument (<a aria-label="Link Help For Diff Argument" target="_blank" href="https://github.com/ifunpar/Sharif-Judge/blob/docs/v1.4/add_assignment.md#diff-arguments">?</a>)</th>
+    <th rowspan="2">Upload<br>Only (<a aria-label="Link Help For Upload Only" target="_blank" href="https://github.com/ifunpar/Sharif-Judge/blob/docs/v1.4/add_assignment.md#upload-only">?</a>)</th>
<th rowspan="2"></th>
</tr>
<tr>

diff --git a/application/views/pages/assignments.twig b/application/views/pages/assignments.twig
index f4091385..5d4bafa3 100644
--- a/application/views/pages/assignments.twig
+++ b/application/views/pages/assignments.twig
@@ -69,7 +69,7 @@

</td>
<td>
-    <a href="{{ site_url('assignments/pdf/'~item.id) }}"><img src="{{ base_url('assets/images/pdf.svg') }}" /></a>
+    <a href="{{ site_url('assignments/pdf/'~item.id) }}"><img src="{{ base_url('assets/images/pdf.svg') }}" aria-label="Download PDF For Assignment {{ item.name }}"/></a>
</td>

<td>

diff --git a/application/views/templates/top_bar.twig b/application/views/templates/top_bar.twig
index ccb67df9..57ed6ca4 100644
--- a/application/views/templates/top_bar.twig
+++ b/application/views/templates/top_bar.twig
@@ -41,7 +41,7 @@
</div>
<div id="shj_logo" class="top_left">
<a href="{{ site_url('/') }}">
-    <img src="{{ base_url('assets/images/logo_small.png') }}"/>
+    <img src="{{ base_url('assets/images/logo_small.png') }}" aria-label="Logo SharIF Judge"/>
<h1 class="shjlogo-text">SharIF <span>Judge</span></h1>
</a>
</div>
\end{lstlisting}

\end{itemize}

\subsection{Implementasi Kriteria Sukses 1.3.1 Info dan Relationships}
\label{subsec:implementasi_A_1.3.1}

Berikut adalah perubahan yang perlu dilakukan untuk memenuhi Kriteria Sukses 1.3.1:

\begin{itemize}
	\item Semua elemen bagian tabel \textit{Problems} pada halaman \textit{Add Assignment} perlu diberi label untuk memberikan informasi sesuai elemennya. Perubahan yang terjadi ada pada \textit{file} \textit{/application/views/pages/admin/add\_assignment.twig}.

	\item \textit{checkbox} dan \textit{textarea} pada halaman \textit{Add Users} perlu diberi label untuk memberikan informasi elemen tersebut. Perubahan yang terjadi ada pada \textit{file} \textit{/application/views/pages/admin/add\_user.twig}.

	\item \textit{textarea} pada halaman \textit{Edit Problem Markdown} perlu diberi label untuk memberikan informasi elemen tersebut. Perubahan yang terjadi ada pada \textit{file} \textit{/application/views/pages/admin/edit\_problem\_md.twig}.

	\item \textit{textarea} pada halaman \textit{Edit Problem Plain HTML} perlu diberi label untuk memberikan informasi elemen tersebut. Perubahan yang terjadi ada pada \textit{file} \textit{/application/views/pages/admin/edit\_problem\_plain.twig}.
	
	\item \textit{Dropdown} pada halaman \textit{Problems} perlu diberi label untuk memberikan informasi elemen tersebut. Perubahan yang terjadi ada pada \textit{file} \textit{/application/views/pages/problems.twig}.

	\item Masukan \textit{Upload File} pada halaman \textit{Problems} perlu diberi label untuk memberikan informasi elemen tersebut. Perubahan yang terjadi ada pada \textit{file} \textit{/application/views/pages/problems.twig}.
	
	\item Setiap judul bagian pada halaman \textit{SharIF Judge} tidak memakai \textit{tag heading}. Perubahan yang perlu dilakukan adalah dengan cara memakai \textit{tag heading} pada setiap judul bagian.
	Perubahan yang terjadi ada pada \textit{file} \textit{/application/views/templates/base.twig}.
	
	Semua perubahan kode dapat dilihat pada \ref{lst_1.3.1}.
	
\begin{lstlisting}[language=diff, caption=Perubahan untuk mematuhi kriteria 1.3.1, label=lst_1.3.1, basicstyle=\ttfamily, frame=single,
columns=fullflexible, keepspaces=true, breaklines=true]
diff --git a/application/views/pages/admin/add_assignment.twig b/application/views/pages/admin/add_assignment.twig
index 13a0ddea..0d3d0cb3 100644
--- a/application/views/pages/admin/add_assignment.twig
+++ b/application/views/pages/admin/add_assignment.twig
@@ -22,16 +22,16 @@
<script>
shj.num_of_problems={{ problems|length }};
shj.row='<tr><td>PID</td>\
-    <td><input type="text" name="name[]" class="sharif_input short" value="Problem "/></td>\
-    <td><input type="text" name="score[]" class="sharif_input tiny2" value="100"/></td>\
-    <td><input type="text" name="c_time_limit[]" class="sharif_input tiny2" value="500"/></td>\
-    <td><input type="text" name="python_time_limit[]" class="sharif_input tiny2" value="1500"/></td>\
-    <td><input type="text" name="java_time_limit[]" class="sharif_input tiny2" value="2000"/></td>\
-    <td><input type="text" name="memory_limit[]" class="sharif_input tiny" value="50000"/></td>\
-    <td><input type="text" name="languages[]" class="sharif_input short2" value="C,C++,Python 2,Python 3,Java"/></td>\
-    <td><input type="text" name="diff_cmd[]" class="sharif_input tiny" value="diff"/></td>\
-    <td><input type="text" name="diff_arg[]" class="sharif_input tiny" value="-bB"/></td>\
-    <td><input type="checkbox" name="is_upload_only[]" class="check" value="PID"/><td><i class="fa fa-times-circle fa-lg color1 delete_problem pointer"></i></td></td>\
+    <td><input aria-label="Problem Name" type="text" name="name[]" class="sharif_input short" value="Problem "/></td>\
+    <td><input aria-label="Score" type="text" name="score[]" class="sharif_input tiny2" value="100"/></td>\
+    <td><input aria-label="Time Limit for C" type="text" name="c_time_limit[]" class="sharif_input tiny2" value="500"/></td>\
+    <td><input aria-label="Time Limit for Python" type="text" name="python_time_limit[]" class="sharif_input tiny2" value="1500"/></td>\
+    <td><input aria-label="Time Limit for Java" type="text" name="java_time_limit[]" class="sharif_input tiny2" value="2000"/></td>\
+    <td><input aria-label="Memory Limit" type="text" name="memory_limit[]" class="sharif_input tiny" value="50000"/></td>\
+    <td><input aria-label="Allowed Languages" type="text" name="languages[]" class="sharif_input short2" value="C,C++,Python 2,Python 3,Java"/></td>\
+    <td><input aria-label="Diff Command" type="text" name="diff_cmd[]" class="sharif_input tiny" value="diff"/></td>\
+    <td><input aria-label="Diff Argument" type="text" name="diff_arg[]" class="sharif_input tiny" value="-bB"/></td>\
+    <td><input aria-label="Upload Only" type="checkbox" name="is_upload_only[]" class="check" value="PID"/><td><i class="fa fa-times-circle fa-lg color1 delete_problem pointer"></i></td></td>\
</tr>';
$(document).ready(function(){
$("#add").click(function(){
@@ -201,16 +201,16 @@

<tr>
<td>{{ problem.id }}</td>
-    <td><input type="text" name="name[]" class="sharif_input short" value="{{ problem.name }}"/></td>
-    <td><input type="text" name="score[]" class="sharif_input tiny2" value="{{ problem.score }}"/></td>
-    <td><input type="text" name="c_time_limit[]" class="sharif_input tiny2" value="{{ problem.c_time_limit }}"/></td>
-    <td><input type="text" name="python_time_limit[]" class="sharif_input tiny2" value="{{ problem.python_time_limit }}"/></td>
-    <td><input type="text" name="java_time_limit[]" class="sharif_input tiny2" value="{{ problem.java_time_limit }}"/></td>
-    <td><input type="text" name="memory_limit[]" class="sharif_input tiny" value="{{ problem.memory_limit }}"/></td>
-    <td><input type="text" name="languages[]" class="sharif_input short2" value="{{ problem.allowed_languages }}"/></td>
-    <td><input type="text" name="diff_cmd[]" class="sharif_input tiny" value="{{ problem.diff_cmd }}"/></td>
-    <td><input type="text" name="diff_arg[]" class="sharif_input tiny" value="{{ problem.diff_arg }}"/></td>
-    <td><input type="checkbox" name="is_upload_only[]" class="check" value="{{ problem.id }}" {{ problem.is_upload_only ? 'checked' }}/></td>
+    <td><input aria-label="Problem Name" type="text" name="name[]" class="sharif_input short" value="{{ problem.name }}"/></td>
+    <td><input aria-label="Score" type="text" name="score[]" class="sharif_input tiny2" value="{{ problem.score }}"/></td>
+    <td><input aria-label="Time Limit for C" type="text" name="c_time_limit[]" class="sharif_input tiny2" value="{{ problem.c_time_limit }}"/></td>
+    <td><input aria-label="Time Limit for Python" type="text" name="python_time_limit[]" class="sharif_input tiny2" value="{{ problem.python_time_limit }}"/></td>
+    <td><input aria-label="Time Limit for Java" type="text" name="java_time_limit[]" class="sharif_input tiny2" value="{{ problem.java_time_limit }}"/></td>
+    <td><input aria-label="Memory Limit" type="text" name="memory_limit[]" class="sharif_input tiny" value="{{ problem.memory_limit }}"/></td>
+    <td><input aria-label="Allowed Languages" type="text" name="languages[]" class="sharif_input short2" value="{{ problem.allowed_languages }}"/></td>
+    <td><input aria-label="Diff Command" type="text" name="diff_cmd[]" class="sharif_input tiny" value="{{ problem.diff_cmd }}"/></td>
+    <td><input aria-label="Diff Argument" type="text" name="diff_arg[]" class="sharif_input tiny" value="{{ problem.diff_arg }}"/></td>
+    <td><input aria-label="Upload Only" type="checkbox" name="is_upload_only[]" class="check" value="{{ problem.id }}" {{ problem.is_upload_only ? 'checked' }}/></td>
<td><i class="fa fa-times-circle fa-lg color1 delete_problem pointer"></i></td>
</tr>


diff --git a/application/views/pages/admin/add_user.twig b/application/views/pages/admin/add_user.twig
index 7f7ef43d..0b85aeb6 100644
--- a/application/views/pages/admin/add_user.twig
+++ b/application/views/pages/admin/add_user.twig
@@ -56,10 +56,10 @@
<li>If you want to send passwords by email, do not add too many users at one time. This may result in mail delivery fail.</li>
</ul>
<p class="input_p">
-    <input type="checkbox" name="send_mail" id="send_mail" /> Send usernames and passwords by email (Waits <input type="text" name="delay" id="delay" class="sharif_input tiny" value="2"/> second(s) before sending each email, so please be patient).
+    <input type="checkbox" name="send_mail" id="send_mail" aria-label="Send mail"/> Send usernames and passwords by email (Waits <input type="text" name="delay" id="delay" class="sharif_input tiny" value="2" aria-label="Delay"/> second(s) before sending each email, so please be patient).
</p>
<p class="input_p">
-    <textarea name="new_users" id="new_users" rows="20" cols="80" class="sharif_input">
+    <textarea name="new_users" id="new_users" rows="20" cols="80" class="sharif_input" aria-label="Command for creating new user">
# Lines starting with a # sign are comments.
# Each line (except comments) represents a user.
# The syntax of each line is:

diff --git a/application/views/pages/admin/edit_problem_md.twig b/application/views/pages/admin/edit_problem_md.twig
index a9bed8fb..e6fd7923 100644
--- a/application/views/pages/admin/edit_problem_md.twig
+++ b/application/views/pages/admin/edit_problem_md.twig
@@ -177,7 +177,7 @@ Violets are blue.
</p>
{{ form_open("problems/edit/md/#{description_assignment.id}/#{problem.id}") }}
<p class="input_p">
-    <textarea dir="auto" name="text" rows="30" cols="75" class="sharif_input" id="md_editor">{{ problem.description }}</textarea>
+    <textarea dir="auto" name="text" rows="30" cols="75" class="sharif_input" id="md_editor" aria-label="Markdown Editor">{{ problem.description }}</textarea>
</p>
<p class="input_p">
<input type="submit" value="Save" class="sharif_input"/>

diff --git a/application/views/pages/admin/edit_problem_plain.twig b/application/views/pages/admin/edit_problem_plain.twig
index 2693e63d..cf8b0e76 100644
--- a/application/views/pages/admin/edit_problem_plain.twig
+++ b/application/views/pages/admin/edit_problem_plain.twig
@@ -33,7 +33,7 @@
</p>
{{ form_open("problems/edit/html/#{description_assignment.id}/#{problem.id}") }}
<p class="input_p">
-    <textarea name="text" rows="30" cols="80" class="sharif_input" id="html_editor">{{ problem.description }}</textarea>
+    <textarea name="text" rows="30" cols="80" class="sharif_input" id="html_editor" aria-label="HTML Editor">{{ problem.description }}</textarea>
</p>
<p class="input_p">
<input type="submit" value="Save" class="sharif_input"/>

diff --git a/application/views/pages/problems.twig b/application/views/pages/problems.twig
index 4f66153e..d1c119fc 100644
--- a/application/views/pages/problems.twig
+++ b/application/views/pages/problems.twig
@@ -96,7 +96,7 @@ $(document).ready(function(){
<input type="hidden" name="problem" value="{{ problem.id }}"/>

<p class="input_p">
-    <select id="languages" name="language" class="sharif_input full-width">
+    <select id="languages" name="language" class="sharif_input full-width" aria-label="Select Language">
<option value="0" selected="selected">-- Select Language --</option>

<option value="{{ l }}">{{ l }}</option>
@@ -104,7 +104,7 @@ $(document).ready(function(){
</select>
</p>
<p class="input_p">
-    <input type="file" id="file" class="sharif_input full-width" name="userfile"/>
+    <input type="file" id="file" class="sharif_input full-width" name="userfile" aria-label="Upload File"/>
</p>
<p class="input_p">
<input type="submit" value="Submit" class="sharif_input"/>

diff --git a/application/views/templates/base.twig b/application/views/templates/base.twig
index e580f5de..047dc432 100644
--- a/application/views/templates/base.twig
+++ b/application/views/templates/base.twig
@@ -58,7 +58,7 @@ shj.color_scheme = 'github';

<div id="page_title">
<i class="fa "></i>
-    <span dir="auto"></span>
+    <h1 dir="auto"></h1>

</div>

diff --git a/assets/styles/main.css b/assets/styles/main.css
index a0e61502..97fb8672 100644
--- a/assets/styles/main.css
+++ b/assets/styles/main.css
@@ -571,6 +571,14 @@ div#page_title > i + span {
margin-left: 8px;
}

+ div#page_title > h1 {
+    margin-left: 8px;
+    display: inline !important;
+    font-size: 20px !important;
+    margin-block-start: 0em !important;
+    margin-block-end: 0em !important;
+ }
+
span.title_menu_item {
font-size: 12px;
margin-left: 10px;
\end{lstlisting}

\end{itemize}

\subsection{Implementasi Kriteria Sukses 2.1.1 Keyboard}
\label{subsec:implementasi_A_2.1.1}
Berikut adalah perubahan yang perlu dilakukan untuk memenuhi Kriteria Sukses 2.1.1:

\begin{itemize}
	\item Fokus \textit{keyboard} tidak dapat terfokus pada menu \textit{Tools} pada menu bagian atas. Hal ini dapat diperbaiki dengan cara menambahkan \textit{tabindex} yang bernilai 0 sehingga fokus \textit{keyboard} dapat terfokus di elemen tersebut. Perubahan dapat dilihat pada \textit{file} \textit{/application/views/templates/top\_bar.twig}.
	
	Menu \textit{Tools} pada bagian menu atas memiliki \textit{sub-menu} yang dapat muncul ketika \textit{pointer cursor} berada diatasnya, menu ini tidak muncul ketika fokus \textit{keyboard} berada pada menu \textit{Tools}. Perlu ditambahkan fungsi untuk memunculkan \textit{sub-menu} ketika fokus \textit{keyboard} berada pada menu \textit{Tools}. \textit{Sub-menu} akan hilang jika fokus \textit{keyboard} terfokus pada \textit{sidebar}. Perubahan dapat dilihat pada \textit{file} \textit{/assets/js/shj\_functions.js}.

	\item \textit{Sub-menu} \textit{Profile} pada bagian menu atas tidak muncul ketika fokus \textit{keyboard} berada pada gambar \textit{Profile}. Gambar \textit{Profile} memiliki kelas yang sama seperti menu \textit{Tools} sehingga perubahan yang dilakukan pada kelas tersebut dapat memperbaiki masalah ini.
	
	\item \textit{Sub-menu} memilih \textit{Assignment} pada bagian menu atas tidak muncul ketika fokus \textit{keyboard} berada pada tautan \textit{Assigntment} sekarang. Tautan \textit{Assignment} memiliki kelas yang sama seperti menu \textit{Tools} sehingga perubahan yang dilakukan pada kelas tersebut dapat memperbaiki masalah ini.
	
	\textit{Sub-menu} memilih \textit{Assignment} pada bagian menu atas tidak dapat difokuskan dengan \textit{keyboard} sehingga perlu ditambahkan atribut \textit{tabindex} yang bernilai 0 agar fokus keyboard dapat terfokus pada elemen tersebut. Perubahan dapat dilihat pada \textit{file} \textit{/application/views/templates/top\_bar.twig}.

	Pada \textit{Sub-menu} memilih \textit{Assignment} pada menu atas, pengguna tidak dapat memilih \textit{Assignment} dengan menggunakan \textit{keyboard}. Hal ini dapat diperbaiki dengan cara menambahkan fungsi yang jika dijalankan akan menjalankan fungsi yang sama ketika pengguna menekan salah satu dari \textit{list} \textit{Assignment}. Pengguna harus menekan tombol \textit{''enter''} pada \textit{keyboard} ketika fokus \textit{keyboard} berada pada elemen ini untuk menjalankan fungsi tersebut. Perubahan dapat dilihat pada \textit{file} \textit{/assets/js/shj\_functions.js}.
	
	\item Aksi untuk \textit{Delete User} dan \textit{Delete Submissions} pada halaman \textit{Users} tidak dapat difokuskan dengan \textit{keyboard} sehingga perlu diubah menjadi button agar dapat difokuskan dengan keyboard. Perubahan dapat dilihat pada \textit{file} \textit{/application/views/pages/admin/users.twig}.

	\item \textit{checkbox} untuk memilih \textit{Assignment} pada halaman \textit{Assignments} tidak dapat difokuskan dengan \textit{keyboard} sehingga perlu diberi atribut \textit{tabindex} yang bernilai 0 agar fokus \textit{keyboard} dapat terfokus pada elemen tersebut. Perubahan dapat dilihat pada \textit{file} \textit{/application/views/pages/assignments.twig}.

	\item Tombol \textit{Add} dan \textit{Delete Problems} pada halaman \textit{Add Assignment} tidak dapat difokuskan \textit{keyboard} sehingga perlu dirubah menjadi \texttt{button} agar dapat difokuskan dengan keyboard. Perubahan dapat dilihat pada \textit{file} \textit{/application/views/pages/admin/add\_assignment.twig}.

	\item \textit{checkbox} pada halaman \textit{All Submissions} tidak dapat difokuskan dengan \textit{keyboard} sehingga perlu ditambahkan atribut \textit{tabindex} yang bernilai 0 agar elemen dapat difokuskan dengan \textit{keyboard}. Perubahan dapat dilihat pada \textit{file} \textit{/application/views/pages/submissions.twig}.
	
	Aksi yang ada di \textit{checkbox} pada halaman \textit{All Submissions} tidak dapat dijalankan dengan \textit{keyboard} sehingga perlu ditambahkan fungsi yang jika dijalankan akan menjalankan fungsi yang sama ketika pengguna menekan \textit{checkbox} tersebut. Fungsi ini hanya bisa dijalankan jika fokus sedang berada pada \textit{checkbox} dan pengguna menekan tombol \textit{''spacebar''} pada \textit{keyboard}. Perubahan dapat dilihat pada \textit{file} \textit{/assets/js/shj\_submissions.js}.
	
	Aksi melihat status, melihat kode, melihat \textit{log} dan aksi \textit{rejudge} pada halaman \textit{All Submission} dan \textit{Final Submissions} tidak dapat difokuskan dengan \textit{keyboard} sehingga perlu ditambahkan atribut \textit{tabindex} yang bernilai 0 agar elemen dapat difokuskan dengan \textit{keyboard}. Selain itu fungsi dari elemen tersebut tidak dapat dijalankan ketika pengguna menekan tombol \textit{''enter''} pada \textit{keyboard} sehingga perlu ditambahkan sebuah fungsi yang menjalankan fungsi yang sama ketika pengguna menekan tombol tersebut. Perubahan dapat dilihat pada \textit{file} \textit{/assets/js/shj\_submissions.js}.
	
	Semua perubahan kode dapat dilihat pada \ref{lst_2.1.1}.

\begin{lstlisting}[language=diff, caption=Perubahan untuk mematuhi kriteria 2.1.1, label=lst_2.1.1, basicstyle=\ttfamily, frame=single,
columns=fullflexible, keepspaces=true, breaklines=true]

diff --git a/application/views/pages/admin/add_assignment.twig b/application/views/pages/admin/add_assignment.twig
index 13a0ddea..57e753c4 100644
--- a/application/views/pages/admin/add_assignment.twig
+++ b/application/views/pages/admin/add_assignment.twig
@@ -31,7 +31,8 @@
<td><input type="text" name="languages[]" class="sharif_input short2" value="C,C++,Python 2,Python 3,Java"/></td>\
<td><input type="text" name="diff_cmd[]" class="sharif_input tiny" value="diff"/></td>\
<td><input type="text" name="diff_arg[]" class="sharif_input tiny" value="-bB"/></td>\
-    <td><input type="checkbox" name="is_upload_only[]" class="check" value="PID"/><td><i class="fa fa-times-circle fa-lg color1 delete_problem pointer"></i></td></td>\
+    <td><input type="checkbox" name="is_upload_only[]" class="check" value="PID"/></td>\
+    <td><button class="delete_problem" type="button" aria-label="Delete Problem"><i class="fa fa-times-circle fa-lg color1 pointer"></i></button></td>\
</tr>';
$(document).ready(function(){
$("#add").click(function(){
@@ -178,7 +179,7 @@
{{ form_error('late_rule', '<div class="shj_error">', '</div>') }}
</p>
</div>
-    <p class="input_p" id="add_problems">Problems <i class="fa fa-plus-circle fa-lg color11 pointer" id="add"></i>
+    <p class="input_p" id="add_problems">Problems <button type="button" id="add" aria-label="Add Problem"><i class="fa fa-plus-circle fa-lg color11 pointer"></i></button>
<table id="problems_table">
<thead>
<tr>
@@ -211,7 +212,7 @@
<td><input type="text" name="diff_cmd[]" class="sharif_input tiny" value="{{ problem.diff_cmd }}"/></td>
<td><input type="text" name="diff_arg[]" class="sharif_input tiny" value="{{ problem.diff_arg }}"/></td>
<td><input type="checkbox" name="is_upload_only[]" class="check" value="{{ problem.id }}" {{ problem.is_upload_only ? 'checked' }}/></td>
-    <td><i class="fa fa-times-circle fa-lg color1 delete_problem pointer"></i></td>
+    <td><button class="delete_problem" type="button" aria-label="Delete Problem"><i class="fa fa-times-circle fa-lg color1 pointer"></i></td>
</tr>

</tbody>
diff --git a/application/views/pages/admin/users.twig b/application/views/pages/admin/users.twig
index 21d51fe2..d8bf0858 100644
--- a/application/views/pages/admin/users.twig
+++ b/application/views/pages/admin/users.twig
@@ -52,10 +52,10 @@
<td>{{ user.first_login_time ? user.first_login_time : 'Never' }}</td>
<td>{{ user.last_login_time ? user.last_login_time : 'Never' }}</td>
<td>
-    <a title="Edit" href="{{ site_url('profile/'~user.id) }}"><i class="fa fa-pencil fa-lg color9"></i></a>
-    <a title="Submissions" href="{{ site_url('submissions/all/user/'~user.username) }}"><i class="fa fa-bars fa-lg color12"></i></a>
-    <span title="Delete User" class="delete_user pointer"><i title="Delete User" class="fa fa-times fa-lg color2"></i></span>
-    <span title="Delete Submissions" class="delete_submissions pointer"><i class="fa fa-times-circle fa-lg color1"></i></span>
+    <a title="Edit User {{ user.username }}" href="{{ site_url('profile/'~user.id) }}"><i class="fa fa-pencil fa-lg color9"></i></a>
+    <a title="Submissions User {{ user.username }}" href="{{ site_url('submissions/all/user/'~user.username) }}"><i class="fa fa-bars fa-lg color12"></i></a>
+    <button class="delete_user"><span title="Delete User {{ user.username }}" class="pointer"><i title="Delete User" class="fa fa-times fa-lg color2"></i></span></button>
+    <button class="delete_submissions" ><span title="Delete Submissions {{ user.username }}" class="pointer"><i class="fa fa-times-circle fa-lg color1"></i></span></button>
</td>
</tr>

diff --git a/application/views/pages/assignments.twig b/application/views/pages/assignments.twig
index f4091385..e6b13589 100644
--- a/application/views/pages/assignments.twig
+++ b/application/views/pages/assignments.twig
@@ -48,7 +48,7 @@
</thead>

<tr>
-    <td><i class="pointer select_assignment fa {{ item.id == user.selected_assignment.id ? 'fa-check-square-o color6' : 'fa-square-o' }} fa-2x" data-id="{{ item.id }}"></i></td>
+    <td><i tabindex="0" class="pointer select_assignment fa {{ item.id == user.selected_assignment.id ? 'fa-check-square-o color6' : 'fa-square-o' }} fa-2x" data-id="{{ item.id }}"></i></td>
<td dir="auto">{{ item.name }}</td>
<td><a href="{{ site_url('problems/'~item.id) }}">{{ item.problems }} problem{{ item.problems != 1 ? 's' }}</a></td>
<td>{{ item.total_submits }} submission{{ item.total_submits != 1 ? 's' }}</td>
diff --git a/application/views/pages/submissions.twig b/application/views/pages/submissions.twig
index 2f58f9cf..607c8370 100644
--- a/application/views/pages/submissions.twig
+++ b/application/views/pages/submissions.twig
@@ -108,7 +108,7 @@
<tr data-u="{{ submission.username }}" data-a="{{ submission.assignment }}" data-p="{{ submission.problem }}" data-s="{{ submission.submit_id }}" class="hl">

<td>
-    <i class="pointer set_final fa {{ submission.is_final ? 'fa-check-circle-o color11' : 'fa-circle-o' }} fa-2x"></i>
+    <i tabindex="0" aria-label="Set Final" role="checkbox" aria-checked="{{ submission.is_final ? 'true' : 'false' }}" class="pointer set_final fa {{ submission.is_final ? 'fa-check-circle-o color11' : 'fa-circle-o' }} fa-2x"></i>
</td>


@@ -148,7 +148,7 @@



-    <div class="{{ submission_class }}" data-type="result" >
+    <div tabindex="0" class="{{ submission_class }}" data-type="result" >

{{ submission.final_score }}

@@ -159,9 +159,9 @@
</td>
<td>

-    <div class="btn shj-orange" data-type="download">Download</div>
+    <div tabindex="0" class="btn shj-orange" data-type="download">Download</div>

-    <div class="btn shj-orange" data-type="code" >Code</div>
+    <div tabindex="0" class="btn shj-orange" data-type="code" >Code</div>

</td>

@@ -169,13 +169,13 @@

---

-    <div class="btn" data-type="log" >Log</div>
+    <div tabindex="0" class="btn" data-type="log" >Log</div>

</td>


<td>
-    <div class="shj_rejudge pointer"><i class="fa fa-refresh fa-lg color10"></i></div>
+    <div tabindex="0" class="shj_rejudge pointer"><i class="fa fa-refresh fa-lg color10"></i></div>
</td>

</tr>
diff --git a/application/views/templates/top_bar.twig b/application/views/templates/top_bar.twig
index ccb67df9..d6b4fc0d 100644
--- a/application/views/templates/top_bar.twig
+++ b/application/views/templates/top_bar.twig
@@ -46,7 +46,7 @@
</a>
</div>

-    <div class="top_object shj_menu top_left" id="admin_tools_top">
+    <div class="top_object shj_menu top_left" tabindex="0" id="admin_tools_top">
Tools
<ul class="top_menu">
<li><a href="{{ site_url('rejudge') }}">Rejudge</a></li>
@@ -59,7 +59,7 @@
<a href="{{ site_url('assignments') }}"><span dir="auto" class="assignment_name">{{ user.selected_assignment.name|length > 30 ? user.selected_assignment.name|slice(0, 30) ~ '...' : user.selected_assignment.name }}</span></a>
<ul class="top_menu" id="select_assignment_menu">

-    <li class="assignment_block select_assignment">
+    <li class="assignment_block select_assignment" tabindex="0">
<i class="fa {{ assignment_item.id == user.selected_assignment.id ? 'fa-check-square-o color6' : 'fa-square-o' }}" data-id="{{ assignment_item.id }}"></i>
<span class="assignment_item" dir="auto">{{ assignment_item.name }}</span>
</li>
diff --git a/assets/js/shj_functions.js b/assets/js/shj_functions.js
index c7b7140f..18c162d1 100644
--- a/assets/js/shj_functions.js
+++ b/assets/js/shj_functions.js
@@ -341,6 +341,16 @@ $(document).ready(function () {
* Top Bar
*/
$(document).ready(function () {
+    $('.shj_menu').on('focusin',function(e){
+      $(this).children(".top_menu").show();
+      $(this).addClass('shj_white');
+      $("#side_bar").on('focusin', function (e) {
+        if(!$(e.target).parent().is('top_menu')){
+          $('.shj_menu').children(".top_menu").hide();
+          $('.shj_menu').removeClass('shj_white');
+        }
+      })
+    });
$("#top_bar").hoverIntent({
over: function () {
$(this).children(".top_menu").show();
@@ -352,6 +362,11 @@ $(document).ready(function () {
},
selector: '.top_object.shj_menu'
});
+    $(".select_assignment").on('keyup', function (e) {
+      if(e.which==13){
+        $(this).trigger("click");
+      }
+    })
$(".select_assignment").click(
function () {
var id = $(this).children('i').addBack('i').data('id');
diff --git a/assets/js/shj_submissions.js b/assets/js/shj_submissions.js
index e92c3f01..ba4d15f5 100644
--- a/assets/js/shj_submissions.js
+++ b/assets/js/shj_submissions.js
@@ -12,6 +12,11 @@ $(document).ready(function () {
e.preventDefault();
$('.code-column').selectText();
});
+    $(".btn").on('keyup', function (e) {
+      if(e.which==13){
+        $(this).trigger("click");
+      }
+    })
$(".btn").click(function () {
var button = $(this);
var row = button.parents('tr');
@@ -62,6 +67,11 @@ $(document).ready(function () {

});
$(".shj_rejudge").attr('title', 'Rejudge');
+    $(".shj_rejudge").on('keyup', function (e) {
+      if(e.which==13){
+        $(this).trigger("click");
+      }
+    })
$(".shj_rejudge").click(function () {
var row = $(this).parents('tr');
$.ajax({
@@ -87,6 +97,11 @@ $(document).ready(function () {
}
});
});
+    $(".set_final").on('keyup', function (e) {
+      if(e.which==32){
+        $(this).trigger("click");
+      }
+    })
$(".set_final").click(
function () {
var row = $(this).parents('tr');
@@ -107,6 +122,8 @@ $(document).ready(function () {
error: shj.loading_error,
success: function (response) {
if (response.done) {
+    $("tr[data-u='" + username + "'][data-p='" + problem + "'] i.set_final").attr("aria-checked", false);
+    $("tr[data-u='" + username + "'][data-p='" + problem + "'][data-s='" + submit_id + "'] i.set_final").attr("aria-checked", true);
$("tr[data-u='" + username + "'][data-p='" + problem + "'] i.set_final").removeClass('fa-check-circle-o color11').addClass('fa-circle-o');
$("tr[data-u='" + username + "'][data-p='" + problem + "'][data-s='" + submit_id + "'] i.set_final").removeClass('fa-circle-o').addClass('fa-check-circle-o color11');
}
\end{lstlisting}

\end{itemize}

\subsection{Implementasi Kriteria Sukses 2.1.2 No Keyboard Trap}
\label{subsec:implementasi_A_2.1.2}

Kriteria 2.1.2 dapat dipenuhi dengan cara menambahkan mekanisme untuk mengaktifkan dan mematikan fitur \textit{indent} yang menggunakan tombol \textit{''tab''} pada setiap halaman yang memakai fitur tersebut sehingga pengguna dapat melanjutkan navigasi pada aplikasi menggunakan tombol \textit{''tab''}. Mekanisme baru ini dapat diaktifkan/dimatikan dengan cara menekan tombol \textit{''esc''}. Perubahan dapat dilihat pada listing \ref{lst_2.1.2}.

\begin{lstlisting}[language=diff, caption=Perubahan untuk mematuhi kriteria 2.1.2, label=lst_2.1.2, basicstyle=\ttfamily, frame=single,
columns=fullflexible, keepspaces=true, breaklines=true]
diff --git a/application/views/pages/admin/add_assignment.twig b/application/views/pages/admin/add_assignment.twig
index 13a0ddea..5953bd21 100644
--- a/application/views/pages/admin/add_assignment.twig
+++ b/application/views/pages/admin/add_assignment.twig
@@ -16,6 +16,13 @@
<script>
$(document).ready(function(){
tabOverride.set(document.getElementsByTagName('textarea'));
+    $switch = false;
+    $("textarea").keyup(function (e) {
+      if (e.which==27){
+        tabOverride.set(document.getElementsByTagName('textarea'),$switch);
+        $switch = !$switch;
+      }
+    });
});
</script>
<script type="text/javascript" src="{{ base_url('assets/js/jquery-ui-timepicker-addon.js') }}"></script>
@@ -126,6 +133,7 @@
<label for="form_participants">Participants<br>
<span class="form_comment">Enter username of participants here (comma separated).
Only these users are able to submit. You can use keyword "ALL".</span>
+    <span class="form_comment clear">Press "esc" to enable/disable tabindent</span>
</label>
<textarea id="form_participants" name="participants" rows="5" class="sharif_input medium">{{ edit ? edit_assignment.participants : set_value('participants', 'ALL') }}</textarea>
</p>
@@ -174,6 +182,7 @@
<p class="input_p">
<label for="form_late_rule">Coefficient rule (<a target="_blank" href="https://github.com/ifunpar/Sharif-Judge/blob/docs/v1.4/add_assignment.md#coefficient-rule">?</a>)</label><br>
<span class="form_comment medium clear" style="display: block;">PHP script without &lt;?php ?&gt; tags</span>
+    <span class="form_comment clear">Press "esc" to enable/disable tabindent</span><br>
<textarea id="form_late_rule" name="late_rule" rows="20" class="sharif_input add_text">{{ edit ? edit_assignment.late_rule : set_value('late_rule', default_late_rule) }}</textarea>
{{ form_error('late_rule', '<div class="shj_error">', '</div>') }}
</p>

diff --git a/application/views/pages/admin/add_user.twig b/application/views/pages/admin/add_user.twig
index 7f7ef43d..a0e91e60 100644
--- a/application/views/pages/admin/add_user.twig
+++ b/application/views/pages/admin/add_user.twig
@@ -22,6 +22,13 @@
<script>
$(document).ready(function(){
tabOverride.set(document.getElementsByTagName('textarea'));
+    $switch = false;
+    $("textarea").keyup(function (e) {
+        if (e.which==27){
+          tabOverride.set(document.getElementsByTagName('textarea'),$switch);
+          $switch = !$switch;
+        }
+    });
});
</script>
<script>
@@ -59,6 +66,7 @@
<input type="checkbox" name="send_mail" id="send_mail" /> Send usernames and passwords by email (Waits <input type="text" name="delay" id="delay" class="sharif_input tiny" value="2"/> second(s) before sending each email, so please be patient).
</p>
<p class="input_p">
+    <span class="form_comment clear">Press "esc" to enable/disable tabindent</span><br>
<textarea name="new_users" id="new_users" rows="20" cols="80" class="sharif_input">
# Lines starting with a # sign are comments.
# Each line (except comments) represents a user.

diff --git a/application/views/pages/admin/edit_problem_md.twig b/application/views/pages/admin/edit_problem_md.twig
index a9bed8fb..a0bc3110 100644
--- a/application/views/pages/admin/edit_problem_md.twig
+++ b/application/views/pages/admin/edit_problem_md.twig
@@ -15,8 +15,15 @@
<script type='text/javascript' src="{{ base_url('assets/js/taboverride.min.js') }}"></script>
<script>
$(document).ready(function(){
-    tabOverride.set(document.getElementById('md_editor'));
-    });
+    tabOverride.set(document.getElementById('md_editor'));
+    $switch = false;
+    $("textarea").keyup(function (e) {
+      if (e.which==27){
+        tabOverride.set(document.getElementById('md_editor'),$switch);
+        $switch = !$switch;
+      }
+    });
+    });
</script>


@@ -176,6 +183,7 @@ Violets are blue.
Problem {{ problem.id }}
</p>
{{ form_open("problems/edit/md/#{description_assignment.id}/#{problem.id}") }}
+    <span class="form_comment clear">Press "esc" to enable/disable tabindent</span><br>
<p class="input_p">
<textarea dir="auto" name="text" rows="30" cols="75" class="sharif_input" id="md_editor">{{ problem.description }}</textarea>
</p>

diff --git a/application/views/pages/admin/settings.twig b/application/views/pages/admin/settings.twig
index 56bbe8cb..827955cc 100644
--- a/application/views/pages/admin/settings.twig
+++ b/application/views/pages/admin/settings.twig
@@ -15,8 +15,15 @@
<script type='text/javascript' src="{{ base_url('assets/js/taboverride.min.js') }}"></script>
<script>
$(document).ready(function(){
-    tabOverride.set(document.getElementsByTagName('textarea'));
-    });
+    tabOverride.set(document.getElementsByTagName('textarea'));
+    $switch = false;
+    $("textarea").keyup(function (e) {
+      if (e.which==27){
+        tabOverride.set(document.getElementsByTagName('textarea'),$switch);
+        $switch = !$switch;
+      }
+    });
+    });
</script>


@@ -122,6 +129,7 @@ $(document).ready(function(){
<p class="input_p">
<label for="form_late_rule">Default Coefficient Rule</label>
<span class="form_comment clear">PHP script without &lt;?php ?&gt; tags</span><br>
+    <span class="form_comment clear">Press "esc" to enable/disable tabindent</span><br>
<textarea id="form_late_rule" name="default_late_rule" rows="15" class="sharif_input add_text clear">{{ default_late_rule }}</textarea>
</p>

@@ -140,11 +148,13 @@ $(document).ready(function(){
<p class="input_p">
<label for="form_mail_reset">Password Reset Email</label>
<span class="form_comment">You can use {SITE_URL}, {RESET_LINK} and {VALID_TIME}</span><br>
+    <span class="form_comment clear">Press "esc" to enable/disable tabindent</span><br>
<textarea id="form_mail_reset" name="reset_password_mail" rows="15" class="sharif_input add_text clear">{{ reset_password_mail }}</textarea>
</p>
<p class="input_p">
<label for="form_mail_add">Add User Email</label>
<span class="form_comment clear">You can use {SITE_URL}, {LOGIN_URL}, {ROLE}, {USERNAME} and {PASSWORD}</span><br>
+    <span class="form_comment clear">Press "esc" to enable/disable tabindent</span><br>
<textarea id="form_mail_add" name="add_user_mail" rows="15" class="sharif_input add_text clear">{{ add_user_mail }}</textarea>
</p>

@@ -203,18 +213,22 @@ $(document).ready(function(){
</p>
<p class="input_p">
<label for="form_def_c">Shield Rules (for C)</label>
+    <span class="form_comment clear">Press "esc" to enable/disable tabindent</span><br>
<textarea id="form_def_c" name="def_c" rows="15" class="sharif_input add_text clear">{{ defc }}</textarea>
</p>
<p class="input_p">
<label for="form_def_cpp">Shield Rules (for C++)</label>
+    <span class="form_comment clear">Press "esc" to enable/disable tabindent</span><br>
<textarea id="form_def_cpp" name="def_cpp" rows="15" class="sharif_input add_text clear">{{ defcpp }}</textarea>
</p>
<p class="input_p">
<label for="form_shield_py2">Shield (for Python 2)</label>
+    <span class="form_comment clear">Press "esc" to enable/disable tabindent</span><br>
<textarea id="form_shield_py2" name="shield_py2" rows="15" class="sharif_input add_text clear">{{ shield_py2 }}</textarea>
</p>
<p class="input_p">
<label for="form_shield_py3">Shield (for Python 3)</label>
+    <span class="form_comment clear">Press "esc" to enable/disable tabindent</span><br>
<textarea id="form_shield_py3" name="shield_py3" rows="15" class="sharif_input add_text clear">{{ shield_py3 }}</textarea>
</p>
<p class="input_p">
\end{lstlisting}

\subsection{Implementasi Kriteria Sukses 2.4.1 Bypass Blocks}
\label{subsec:implementasi_A_2.4.1}

Kriteria 2.4.1 dapat dipenuhi dengan cara menambahkan tautan di awal setiap halaman untuk meloncati menu navigasi. Tautan hanya dapat dijalankan jika pengguna mengoperasikannya dengan \textit{keyboard}. Perubahan dapat dilihat pada \textit{file} \textit{/application/views/templates/base.twig}. Selain itu ada juga penambahan untuk \textit{style} baru yang dapat dilihat pada \textit{file} \textit{/assets/styles/main.css}.

Tidak semua \textit{browser} dapat menangani \textit{in-page links}. Secara visual fokus \textit{keyboard} sudah berada pada lokasi target, tetapi sebenarnya fokus \textit{keyboard} belum diset ke lokasi target. Masalah ini dapat diselesaikan dengan menggunakan \textit{Javascript} untuk set fokus \textit{keyboard} ke lokasi target. Perubahan ada pada \textit{file} \textit{/assets/js/shj\_functions.js}.

Semua perubahan kode dapat dilihat pada \ref{lst_2.4.1}.

\begin{lstlisting}[language=diff, caption=Perubahan untuk mematuhi kriteria 2.4.1, label=lst_2.4.1, basicstyle=\ttfamily, frame=single,
columns=fullflexible, keepspaces=true, breaklines=true]
diff --git a/application/views/templates/base.twig b/application/views/templates/base.twig
index e580f5de..c0b6a8d0 100644
--- a/application/views/templates/base.twig
+++ b/application/views/templates/base.twig
@@ -51,6 +51,8 @@ shj.color_scheme = 'github';


<body id="body">
+    <a href="#page_title" class="skip">Skip to content</a>
+    


<div id="main_container" class="scroll-wrapper">

diff --git a/assets/js/shj_functions.js b/assets/js/shj_functions.js
index c7b7140f..9899d077 100644
--- a/assets/js/shj_functions.js
+++ b/assets/js/shj_functions.js
@@ -555,3 +555,26 @@ $(document).ready(function(){
$(document).ready(function(){
$('input').attr('dir', 'auto');
});
+    
+    /**
+    * Skip links
+    */
+  $(document).ready(function(){
+    // bind a click event to the 'skip' link
+    $(".skip").click(function(event){
+    
+    // strip the leading hash and declare
+    // the content we're skipping to
+    var skipTo="#"+this.href.split('#')[1];
+  
+    // Setting 'tabindex' to -1 takes an element out of normal 
+    // tab flow but allows it to be focused via javascript
+    $(skipTo).attr('tabindex', -1).on('blur focusout', function () {
+  
+    // when focus leaves this element, 
+    // remove the tabindex attribute
+    $(this).removeAttr('tabindex');
+  
+    }).focus(); // focus on the content container
+    });
+  });

diff --git a/assets/styles/main.css b/assets/styles/main.css
index a0e61502..333bd795 100644
--- a/assets/styles/main.css
+++ b/assets/styles/main.css
@@ -146,6 +146,26 @@ blockquote {
margin-left: 7px!important;
}

+    .skip {
+      position: absolute;
+      top: -1000px;
+      left: -1000px;
+      height: 1px;
+      width: 1px;
+      text-align: left;
+      overflow: hidden;
+    }
+  
+    a.skip:active, 
+    a.skip:focus, 
+    a.skip:hover {
+      left: 0; 
+      top: 0;
+      width: auto; 
+      height: auto; 
+      overflow: visible; 
+    }
+ 
/********************************************/
div#top_bar {
top: 0;
\end{lstlisting}

\subsection{Implementasi Kriteria Sukses 2.4.4 Link Purpose (In Context)}
\label{subsec:implementasi_A_2.4.4}

Berikut adalah perubahan yang perlu dilakukan untuk memenuhi Kriteria Sukses 2.4.4:

\begin{itemize}
	\item Seluruh link pada \textit{sidebar} perlu diberikan label yang menjelaskan link tersebut. Perubahan yang terjadi ada pada \textit{file} \textit{/application/views/templates/side\_bar.twig}.

	\item Tautan pada gambar \textit{Profile} yang ada di menu atas perlu diberi label untuk menjelaskan tujuan dari tautan tersebut. Perubahan yang terjadi ada pada \textit{/application/views/templates/top\_bar.twig}.

	\item Tautan pada gambar \textit{PDF} perlu diberi label untuk menjelaskan tujuannya. Perubahan untuk tautan pada gambar \textit{PDF} yang ada di halaman \textit{Assignments} sudah dilakukan pada listing \ref{lst_1.1.1}
	
	\item Tautan yang ada pada tabel \textit{Problems} pada halaman \textit{Add Assignment} tidak memiliki label yang menjelaskan tujuannya, untuk itu perlu diberi label yang menjelaskan tujuannya. Perubahan dapat dilihat pada \textit{file} \textit{/application/views/pages/admin/add\_assignment.twig}.
	
	Semua perubahan kode dapat dilihat pada \ref{lst_2.4.4}.
	
\begin{lstlisting}[language=diff, caption=Perubahan untuk mematuhi kriteria 2.4.4, label=lst_2.4.4, basicstyle=\ttfamily, frame=single,
columns=fullflexible, keepspaces=true, breaklines=true]
diff --git a/application/views/templates/side_bar.twig b/application/views/templates/side_bar.twig
index da069f2e..ce5a0e22 100644
--- a/application/views/templates/side_bar.twig
+++ b/application/views/templates/side_bar.twig
@@ -6,78 +6,78 @@
<div id="side_bar" class="sidebar_open">
<ul>
<li class="color-dashboard{{ selected=='dashboard' ? ' selected' }}">
-    <a href="{{ site_url('dashboard') }}">
+    <a href="{{ site_url('dashboard') }}" aria-labelledby="dashboard-label">
<i class="fa fa-dashboard fa-lg"></i>
-    <span class="sidebar_text">Dashboard</span>
+    <span class="sidebar_text" id="dashboard-label">Dashboard</span>
</a>
</li>

<li class="color-settings{{ selected=='settings' ? ' selected' }}">
-    <a href="{{ site_url('settings') }}">
+    <a href="{{ site_url('settings') }}" aria-labelledby="settings-label">
<i class="fa fa-gear fa-lg"></i>
-    <span class="sidebar_text">Settings</span>
+    <span class="sidebar_text" id="settings-label">Settings</span>
</a>
</li>
<li class="color-users{{ selected=='users' ? ' selected' }}">
-    <a href="{{ site_url('users') }}">
+    <a href="{{ site_url('users') }}" aria-labelledby="users-label">
<i class="fa fa-users fa-lg"></i>
-    <span class="sidebar_text">Users</span>
+    <span class="sidebar_text" id="users-label">Users</span>
</a>
</li>

<li class="color-notifications{{ selected=='notifications' ? ' selected' }}">
-    <a href="{{ site_url('notifications') }}">
+    <a href="{{ site_url('notifications') }}" aria-labelledby="notifications-label">
<i class="fa fa-bell fa-lg"></i>
-    <span class="sidebar_text">Notifications</span>
+    <span class="sidebar_text" id="notifications-label">Notifications</span>
</a>
</li>
<li class="color-assignments{{ selected=='assignments' ? ' selected' }}">
-    <a href="{{ site_url('assignments') }}">
+    <a href="{{ site_url('assignments') }}" aria-labelledby="assignments-label">
<i class="fa fa-folder-open fa-lg"></i>
-    <span class="sidebar_text">Assignments</span>
+    <span class="sidebar_text" id="assignments-label">Assignments</span>
</a>
</li>
<li class="color-problems{{ selected=='problems' ? ' selected' }}">
-    <a href="{{ site_url('problems') }}">
+    <a href="{{ site_url('problems') }}" aria-labelledby="problems-label">
<i class="fa fa-puzzle-piece fa-lg"></i>
-    <span class="sidebar_text">Problems</span>
+    <span class="sidebar_text" id="problems-label">Problems</span>
</a>
</li>
<li class="color-submit{{ selected=='submit' ? ' selected' }}">
-    <a href="{{ site_url('submit') }}">
+    <a href="{{ site_url('submit') }}" aria-labelledby="submit-label">
<i class="fa fa-location-arrow fa-lg"></i>
-    <span class="sidebar_text">Submit</span>
+    <span class="sidebar_text" id="submit-label">Submit</span>
</a>
</li>
<li class="color-final_submissions{{ selected=='final_submissions' ? ' selected' }}">
-    <a href="{{ site_url('submissions/final') }}">
+    <a href="{{ site_url('submissions/final') }}" aria-labelledby="final-submission-label">
<i class="fa fa-map-marker fa-lg"></i>
-    <span class="sidebar_text">Final Submissions</span>
+    <span class="sidebar_text" id="final-submission-label">Final Submissions</span>
</a>
</li>
<li class="color-all_submissions{{ selected=='all_submissions' ? ' selected' }}">
-    <a href="{{ site_url('submissions/all') }}">
+    <a href="{{ site_url('submissions/all') }}" aria-labelledby="all-submission-label">
<i class="fa fa-bars fa-lg"></i>
-    <span class="sidebar_text">All Submissions</span>
+    <span class="sidebar_text" id="all-submission-label">All Submissions</span>
</a>
</li>
<li class="color-scoreboard{{ selected=='scoreboard' ? ' selected' }}">
-    <a href="{{ site_url('scoreboard') }}">
+    <a href="{{ site_url('scoreboard') }}" aria-labelledby="scoreboard-label">
<i class="fa fa-star fa-lg"></i>
-    <span class="sidebar_text">Scoreboard</span>
+    <span class="sidebar_text" id="scoreboard-label">Scoreboard</span>
</a>
</li>
<li class="color-halloffame{{ selected=='halloffame' ? ' selected' }}">
-    <a href="{{ site_url('halloffame') }}">
+    <a href="{{ site_url('halloffame') }}" aria-labelledby="hall-of-fame-label">
<i class="fa fa-list-alt fa-lg"></i>
-    <span class="sidebar_text">Hall of Fame</span>
+    <span class="sidebar_text" id="hall-of-fame-label">Hall of Fame</span>
</a>
</li>

<li class="color-logs{{ selected=='logs' ? ' selected' }}">
-    <a href="{{ site_url('logs') }}">
+    <a href="{{ site_url('logs') }}" aria-labelledby="24-hour-log-label">
<i class="fa fa-book fa-lg"></i>
-    <span class="sidebar_text">24-hour Log</span>
+    <span class="sidebar_text" id="24-hour-log-label">24-hour Log</span>
</a>
</li>


diff --git a/application/views/templates/top_bar.twig b/application/views/templates/top_bar.twig
index ccb67df9..c49c8ab7 100644
--- a/application/views/templates/top_bar.twig
+++ b/application/views/templates/top_bar.twig
@@ -5,7 +5,7 @@
#}
<div id="top_bar" class="color-{{ selected }}">
<div class="top_object shj_menu" id="user_top">
-    <a href="{{ site_url('profile') }}" id="profile_link"><i class="fa fa-user"></i></a>
+    <a href="{{ site_url('profile') }}" id="profile_link" aria-label="Profile"><i class="fa fa-user"></i></a>
<div class="top_menu user-menu">
<div class="gravatar"><img src="http://www.gravatar.com/avatar/{{ md5(user.email) }}?s=70&d=identicon" /></div>
<div class="name"><i class="fa fa-user"></i> {{ user.username }}</div>
\end{lstlisting}

\end{itemize}

\subsection{Implementasi Kriteria Sukses 3.1.1 Language of Page}
\label{subsec:implementasi_A_3.1.1}

Seluruh halaman aplikasi \textit{SharIF Judge} menggunakan Bahasa Inggris sehingga perlu diberikan atribut \textit{lang} yang bernilai \textit{''en''} pada awal \textit{tag html} untuk menunjukkan bahwa bahasa yang dipakai di halaman tersebut adalah Bahasa Inggris. Setiap kali halaman dibuka, aplikasi \textit{SharIF Judge} akan menjalankan \textit{file} \textit{base.twig} sebagai dasarnya sehingga perubahan hanya perlu dilakukan pada \textit{file} tersebut. Perubahan dapat dilihat pada \textit{/application/views/templates/base.twig}, potongan kode dapat dilihat pada listing \ref{lst_3.1.1}.

\begin{lstlisting}[language=diff, caption=Perubahan untuk mematuhi kriteria 3.1.1, label=lst_3.1.1, basicstyle=\ttfamily, frame=single,
columns=fullflexible, keepspaces=true, breaklines=true]
diff --git a/application/views/templates/base.twig b/application/views/templates/base.twig
index e580f5de..91b720a3 100644
--- a/application/views/templates/base.twig
+++ b/application/views/templates/base.twig
@@ -4,7 +4,7 @@
# author: Mohammad Javad Naderi <mjnaderi@gmail.com>
#}
<!DOCTYPE html>
-    <html>
+    <html lang="en">
<head>
<title> - SharIF Judge</title>
<meta content="text/html" charset="UTF-8">
\end{lstlisting}

\subsection{Implementasi Kriteria Sukses 3.3.2 Labels or Instructions}
\label{subsec:implementasi_A_3.3.2}

Berikut adalah perubahan yang perlu dilakukan untuk memenuhi Kriteria Sukses 3.3.2:

\begin{itemize}
	\item Semua elemen bagian tabel \textit{Problems} pada halaman \textit{Add Assignment} perlu diberi label untuk menjelaskan tujuan dari elemen tersebut. Perubahan sudah dilakukan pada listing \ref{lst_1.3.1}.
	
	\item \textit{checkbox} dan \textit{textarea} pada halaman \textit{Add Users} perlu diberi label yang menjelaskan tujuan dari elemen tersebut. Perubahan sudah dilakukan pada listing \ref{lst_1.3.1}.
	
	\item \textit{textarea} pada halaman \textit{Edit Problem Markdown} perlu diberi label yang menjelaskan tujuan dari elemen tersebut. Perubahan sudah dilakukan pada listing \ref{lst_1.3.1}.
	
	\item \textit{textarea} pada halaman \textit{Edit Problem Plain HTML} perlu diberi label yang menjelaskan tujuan dari elemen tersebut. Perubahan sudah dilakukan pada listing \ref{lst_1.3.1}.
	
	\item \textit{Dropdown} pada halaman \textit{Problems} perlu diberi label yang menjelaskan tujuan dari elemen tersebut. Perubahan sudah dilakukan pada listing \ref{lst_1.3.1}.
	
	\item Masukan \textit{Upload File} pada halaman \textit{Problems} perlu diberi label yang menjelaskan tujuan dari elemen tersebut. Perubahan sudah dilakukan pada listing \ref{lst_1.3.1}.
\end{itemize}

\subsection{Implementasi Kriteria Sukses 4.1.1 Parsing}
\label{subsec:implementasi_A_4.1.1}

\textit{Form} masukan \textit{Extra Time} pada halaman \textit{Add Assignment} memiliki atribut id yang duplikat. Atribut yang dibuang yaitu atribut \textit{id=''extra\_time''} karena sudah dipakai pada elemen \textit{extra time} pada menu bagian atas. Perubahan dapat dilihat pada \textit{file} \textit{/application/views/pages/admin/add\_assignment.twig}, potongan kode dapat dilihat pada listing \ref{lst_4.1.1}.

\begin{lstlisting}[language=diff, caption=Perubahan untuk mematuhi kriteria 4.1.1, label=lst_4.1.1, basicstyle=\ttfamily, frame=single,
columns=fullflexible, keepspaces=true, breaklines=true]
diff --git a/application/views/pages/admin/add_assignment.twig b/application/views/pages/admin/add_assignment.twig
index 13a0ddea..a8bb55fd 100644
--- a/application/views/pages/admin/add_assignment.twig
+++ b/application/views/pages/admin/add_assignment.twig
@@ -119,7 +119,7 @@
Extra Time (minutes)<br>
<span class="form_comment">Extra time for late submissions.</span>
</label>
-    <input id="form_extra_time" type="text" name="extra_time" id="extra_time" class="sharif_input medium" value="{{ edit ? edit_assignment.extra_time|extra_time_formatter : set_value('extra_time') }}" />
+    <input id="form_extra_time" type="text" name="extra_time" class="sharif_input medium" value="{{ edit ? edit_assignment.extra_time|extra_time_formatter : set_value('extra_time') }}" />
{{ form_error('extra_time', '<div class="shj_error">', '</div>') }}
</p>
<p class="input_p clear">
\end{lstlisting}

\section{Pengujian}
\label{sec:pengujian}
Pada bagian ini akan ditulis skenario pengujian dan hasil yang didapatkan dari setiap skenario pengujian. Tujuan dari pengujian ini untuk melihat perbaikan yang dilakukan pada subbab \ref{sec:implementasi} berhasil atau tidak. Pengujian dilakukan dengan menggunakan perangkat komputer berupa laptop dengan sistem operasi Ubuntu, \textit{browser} Google Chrome, dan \textit{screen reader} ChromeVox sebagai teknologi alat bantu. Pengujian dilakukan pada server lokal milik penguji dan akun yang digunakan oleh penguji memiliki hak akses tak terbatas sehingga dapat menggunakan semua fitur yang terdapat pada halaman web \textit{SharIF Judge}. Selain itu pengujian juga dilakukan dengan kondisi seakan-akan memiliki keterbatasan visual.

\subsection{Skenario Pengujian}
\label{subsec:skenario_pengujian}
Pada bagian ini akan ditulis skenario pengujian yang dilakukan untuk menguji perbaikan yang telah dilakukan. Setiap skenario pengujian ditulis dalam bentuk poin-poin yang menjelaskan cara untuk menggunakan fitur-fitur yang ada pada aplikasi \textit{SharIF Judge}.

\subsubsection{Login}
\label{subsubsec:skenario_login}
Berikut adalah langkah-langkah yang perlu dilakukan untuk \textit{Login}:

\begin{enumerate}
	\item Memasukkan alamat \url{http://sharif-judge/login} \footnote{url yang dipakai dalam sistem milik penguji} pada \textit{address bar browser}.
	\item Mengisi \textit{Username} pada bidang masukan yang sudah disediakan.
	\item Mengisi \textit{Password} pada bidang masukan yang sudah disediakan.
	\item Menekan tombol \textit{Login}.
\end{enumerate}

\subsubsection{Settings}
\label{subsubsec:skenario_settings}
Berikut adalah langkah-langkah yang perlu dilakukan untuk sunting \textit{Settings}:

\begin{enumerate}
	\item \textit{Login} pada aplikasi \textit{SharIF Judge}.
	\item Memilih \textit{Settings} pada menu \textit{sidebar}.
	\item Mengisi bidang masukan \textit{Timezone} dengan nilai \textit{''Asia/Jakarta''}.
	\item Mengubah nilai \textit{Week Start Day} menjadi \textit{''Sunday''}.
	\item Mengubah nilai \textit{Registration} menjadi \textit{''checked''}.
	\item Mengubah isi \textit{Default Coefficient Rule} sesuai aturan yang sudah disediakan.
	\item Menekan tombol \textit{Save Changes} untuk meyimpan perubahan.
\end{enumerate}

\subsubsection{Add Users}
\label{subsubsec:skenario_add_users}
Berikut adalah langkah-langkah yang perlu dilakukan untuk menambah \textit{User}:

\begin{enumerate}
	\item \textit{Login} pada aplikasi \textit{SharIF Judge}.
	\item Memilih \textit{Users} pada menu \textit{sidebar}.
	\item Menekan tombol \textit{Add Users}.
	\item Mengubah nilai \textit{Send mail} menjadi \textit{''checked''}.
	\item Mengubah interval untuk mengirim pesan menjadi ''2''.
	\item Mengisi perintah untuk membuat \textit{User} pada bidang masukan yang sudah disediakan.
	\item Menekan tombol \textit{Add Users} untuk menambah \textit{User}.
\end{enumerate}

\subsubsection{Delete User}
\label{subsubsec:skenario_delete_user}
Berikut adalah langkah-langkah yang perlu dilakukan untuk menghapus \textit{User}:

\begin{enumerate}
	\item \textit{Login} pada aplikasi \textit{SharIF Judge}.
	\item Memilih \textit{Users} pada menu \textit{sidebar}.
	\item Mencari nama \textit{User} yang akan dihapus pada tabel \textit{Users}.
	\item Menekan tombol \textit{Delete User} pada kolom \textit{action} di baris yang sesuai.
	\item Menekan tombol \textit{Yes, Delete} pada \textit{popup} yang muncul.
\end{enumerate}

\subsubsection{Delete User Submissions}
\label{subsubsec:skenario_delete_user_submissions}
Berikut adalah langkah-langkah yang perlu dilakukan untuk menghapus \textit{User submissions}:

\begin{enumerate}
	\item \textit{Login} pada aplikasi \textit{SharIF Judge}.
	\item Memilih \textit{Users} pada menu \textit{sidebar}.
	\item Mencari nama \textit{User} yang akan dihapus \textit{Submissions} pada tabel \textit{Users}.
	\item Menekan tombol \textit{Delete Submissions} pada kolom \textit{action} di baris yang sesuai.
	\item Menekan tombol \textit{Yes, Delete} pada \textit{popup} yang muncul.
\end{enumerate}

\subsubsection{Edit User}
\label{subsubsec:skenario_edit_user}
Berikut adalah langkah-langkah yang perlu dilakukan untuk sunting \textit{User}:

\begin{enumerate}
	\item \textit{Login} pada aplikasi \textit{SharIF Judge}.
	\item Memilih \textit{Users} pada menu \textit{sidebar}.
	\item Mencari nama \textit{user} yang di \textit{edit} pada tabel \textit{Users}.
	\item Menekan tombol \textit{Edit} pada kolom \textit{action} di baris yang sesuai.
	\item Mengisi isi bidang masukan \textit{Name} menjadi \textit{''student''}.
	\item Mengubah isi \textit{User Role} menjadi \textit{''student''}. 
	\item Menekan tombol \textit{Save} untuk menyimpan perubahan.
\end{enumerate}	

\subsubsection{Add Notification}
\label{subsubsec:skenario_add_notification}
Berikut adalah langkah-langkah yang perlu dilakukan untuk menambah notifikasi:

\begin{enumerate}
	\item \textit{Login} pada aplikasi \textit{SharIF Judge}.
	\item Memilih \textit{Notifications} pada menu \textit{sidebar}.
	\item Menekan tautan \textit{New}.
	\item Mengisi judul pada bidang masukan \textit{Title}.
	\item Mengisi notifikasi pada bidang masukan \textit{Text}.
	\item Menekan tombol \textit{Add} untuk menambah notifikasi.
\end{enumerate}

\subsubsection{Add Assignment}
\label{subsubsec:skenario_add_assignment}
Berikut adalah langkah-langkah yang perlu dilakukan untuk menambah \textit{Assignment}:

\begin{enumerate}
	\item \textit{Login} pada aplikasi \textit{SharIF Judge}.
	\item Memilih \textit{Assignments} pada menu \textit{sidebar}.
	\item Menekan tautan \textit{Add}.
	\item Mengisi bidang masukan \textit{Assignment Name} dengan nilai \textit{''Assignment 1''}.
	\item Mengisi bidang masukan \textit{Participants} dengan nilai \textit{''ALL''}.
	\item Mengunggah \textit{PDF File} dengan \textit{file PDF} yang sesuai.
	\item Mengubah nilai \textit{Scoreboard} menjadi \textit{''checked''}.
	\item Menekan tombol \textit{Add Assignment} untuk menambah \textit{Assignment}.
\end{enumerate}

\subsubsection{Delete Assignment}
\label{subsubsec:skenario_delete_assignment}
Berikut adalah langkah-langkah yang perlu dilakukan untuk menghapus \textit{Assignment}:

\begin{enumerate}
	\item \textit{Login} pada aplikasi \textit{SharIF Judge}.
	\item Memilih \textit{Assignments} pada menu \textit{sidebar}.
	\item Mencari nama \textit{Assignment} yang akan di hapus pada tabel \textit{Assignments}.
	\item Menekan tombol \textit{Delete} pada kolom \textit{action} di baris yang sesuai.
	\item Menekan tombol \textit{Delete this assignment} untuk menghapus \textit{Assignment}.
\end{enumerate}

\subsubsection{Edit Assignment}
\label{subsubsec:skenario_edit_assignment}
Berikut adalah langkah-langkah yang perlu dilakukan untuk sunting \textit{Assignment}:

\begin{enumerate}
	\item \textit{Login} pada aplikasi \textit{SharIF Judge}.
	\item Memilih \textit{Assignments} pada menu \textit{sidebar}.
	\item Mencari nama \textit{Assignment} yang akan disunting pada tabel \textit{Assignments}.
	\item Menekan tombol \textit{Edit} pada kolom \textit{action} di baris yang sesuai.
	\item Mengisi bidang masukan \textit{Extra Time} dengan nilai \textit{''0*60''}.
	\item Mengisi bidang masukan \textit{Participants} dengan nilai \textit{''ALL''}.
	\item Mengunggah \textit{PDF File} dengan \textit{file PDF} yang sesuai.
	\item Mengubah nilai \textit{Open} menjadi \textit{''checked''}.
	\item Menekan tombol \textit{Edit Assignment} untuk menyimpan perubahan.
\end{enumerate}

\subsubsection{Edit Problem Description(Markdown)}
\label{subsubsec:skenario_edit_problem_description_markdown}
Berikut adalah langkah-langkah yang perlu dilakukan untuk sunting deskripsi masalah dalam format \textit{markdown}:

\begin{enumerate}
	\item \textit{Login} pada aplikasi \textit{SharIF Judge}.
	\item Memilih \textit{Assignment} pada menu bagian atas.
	\item Memilih \textit{Problems} pada menu \textit{sidebar}.
	\item Memilih \textit{Problem} pada tabel \textit{Problems}.
	\item Menekan tautan \textit{Edit Markdown}.
	\item Mengisi deskripsi \textit{Problem} pada \textit{textarea}.
	\item Menekan tombol \textit{Save} untuk menyimpan perubahan.
\end{enumerate}

\subsubsection{Edit Problem Description(HTML)}
\label{subsubsec:skenario_edit_problem_description_html}
Berikut adalah langkah-langkah yang perlu dilakukan untuk sunting deskripsi masalah dalam format \textit{HTML}:

\begin{enumerate}
	\item \textit{Login} pada aplikasi \textit{SharIF Judge}.
	\item Memilih \textit{Assignment} pada menu bagian atas.
	\item Memilih \textit{Problems} pada menu \textit{sidebar}.
	\item Memilih \textit{Problem} pada tabel \textit{Problems}.
	\item Menekan tautan \textit{Edit HTML}.
	\item Mengisi deskripsi \textit{Problem} pada \textit{text editor}.
	\item Menekan tombol \textit{Save} untuk menyimpan perubahan.
\end{enumerate}

\subsubsection{Edit Problem Description(HTML)}
\label{subsubsec:skenario_edit_problem_description_plain_html}
Berikut adalah langkah-langkah yang perlu dilakukan untuk sunting deskripsi masalah dalam format \textit{plain HTML}:

\begin{enumerate}
	\item \textit{Login} pada aplikasi \textit{SharIF Judge}.
	\item Memilih \textit{Assignment} pada menu bagian atas.
	\item Memilih \textit{Problems} pada menu \textit{sidebar}.
	\item Memilih \textit{Problem} pada tabel \textit{Problems}.
	\item Menekan tautan \textit{Edit Plain HTML}.
	\item Mengisi deskripsi \textit{Problem} pada \textit{textarea}.
	\item Menekan tombol \textit{Save} untuk menyimpan perubahan.
\end{enumerate}

\subsubsection{Submit File}
\label{subsubsec:skenario_submit}
Berikut adalah langkah-langkah yang perlu dilakukan untuk mengumpulkan \textit{file}:

\begin{enumerate}
	\item \textit{Login} pada aplikasi \textit{SharIF Judge}.
	\item Memilih \textit{Assignment} pada menu bagian atas.
	\item Memilih \textit{Submit} pada menu \textit{sidebar}.
	\item Memilih \textit{Problem} pada \textit{dropdown Problem}.
	\item Memilih bahasa pemrograman pada \textit{dropdown Language}.
	\item Memilih \textit{file} yang akan dikumpulkan.
	\item Menekan tombol \textit{Submit} untuk mengumpulkan \textit{file}.
\end{enumerate}

\subsubsection{Final Submission}
\label{subsubsec:skenario_final_submission}
Berikut adalah langkah-langkah yang perlu dilakukan untuk memilih \textit{Final Submission}:

\begin{enumerate}
	\item \textit{Login} pada aplikasi \textit{SharIF Judge}.
	\item Memilih \textit{Assignment} pada menu bagian atas.
	\item Memilih \textit{All Submission} pada menu \textit{sidebar}.
	\item Memilih \textit{Submission} yang akan dijadikan \textit{Final Submission} pada tabel \textit{All Submissions}
	\item Menekan tombol pada kolom \textit{Final} untuk menjadikan \textit{Submission} menjadi \textit{Final Submission}.
\end{enumerate}

\subsection{Hasil Pengujian}
\label{subsec:hasil_pengujian}
Pada bagian ini akan ditulis hasil dari pengujian yang telah dilakukan berdasarkan skenario pengujian pada bagian \ref{subsec:skenario_pengujian}. Hasil pengujian akan dituliskan dalam bentuk tabel yang berisi langkah skenario pengujian, hasil pengujian, dan aksi yang dilakukan dalam pengujian.

\subsubsection{Login}
\label{subsubsec:hasil_login}
Berikut adalah hasil pengujian \textit{Login}:

\begin{table}[H]
	\centering
	\caption{Tabel hasil pengujian \textit{Login}}
	\label{tab:hasil_login}
	\begin{tabular}{|c|c|p{12cm}|}
		\toprule
		Langkah & Hasil & Aksi\\
		\midrule
		1 & Sukses & Menekan tombol \textit{Tab} sampai ChromeVox membacakan \textit{''address and search bar''} kemudian pengguna memasukkan alamat \url{http://sharif-judge/login}.\\
		2 & Sukses & Menekan tombol \textit{Tab} sampai ChromeVox membacakan bidang masukan \textit{Username}. Setelah itu pengguna memasukkan \textit{Username}.\\
		3 & Sukses & Menekan tombol \textit{Tab} sampai ChromeVox membacakan bidang masukan \textit{Password}. Setelah itu pengguna memasukkan \textit{Password}. \\
		4 & Sukses & Menekan tombol \textit{Tab} sampai ChromeVox membacakan tombol \textit{Login}. Setelah itu pengguna menekan tombol \textit{Enter} pada \textit{keyboard}.\\
		\bottomrule
	\end{tabular}
\end{table}

\subsubsection{Settings}
\label{subsubsec:hasil_settings}
Berikut adalah hasil pengujian \textit{Settings}:

\begin{table}[H]
	\centering
	\caption{Tabel hasil pengujian \textit{Settings}}
	\label{tab:hasil_settings}
	\begin{tabular}{|c|c|p{12cm}|}
		\toprule
		Langkah & Hasil & Aksi\\
		\midrule
		1 & Sukses & Langkah \textit{Login} dapat dilihat pada \ref{subsubsec:hasil_login}.\\
		2 & Sukses & Menekan tombol \textit{Tab} untuk navigasi ke \textit{sidebar} sampai ChromeVox membacakan tautan untuk menu \textit{Settings}. Setelah itu pengguna menekan tombol \textit{Enter} pada keyboard.\\
		3 & Sukses & Menekan tombol \textit{Tab} sampai ChromeVox membacakan bidang masukan \textit{Timezone}. Setelah itu pengguna mengisi dengan nilai \textit{''Asia/Jakarta''}.\\
		4 & Sukses & Menekan tombol \textit{Tab} sampai ChromeVox membacakan bidang masukan \textit{Week Start Day}. Setelah itu pengguna mengubah isi \textit{combobox} dengan menekan tombol panah atas atau bawah pada \textit{keyboard} sampai ChromeVox membacakan \textit{''Sunday''}.\\
		5 & Sukses & Menekan tombol \textit{Tab} sampai ChromeVox membacakan bidang masukan \textit{Default Coefficient Rule}. Setelah itu pengguna mengisi sesuai spesifikasi yang sudah ditentukan. Pengguna dapat melanjutkan navigasi dengan menekan tombol \textit{Tab} setelah mematikan fitur \textit{tab indent} pada \textit{textarea} dengan menekan tombol \textit{Esc} pada \textit{keyboard}.\\
		6 & Sukses & Menekan tombol \textit{Tab} sampai ChromeVox membacakan tombol \textit{Save Changes} kemudian menekan tombol \textit{Enter} pada \textit{keyboard}\\
		\bottomrule
	\end{tabular}
\end{table}

\subsubsection{Add Users}
\label{subsubsec:hasil_add_users}
Berikut adalah hasil pengujian \textit{Add Users}:

\begin{table}[H]
	\centering
	\caption{Tabel hasil pengujian \textit{Add Users}}
	\label{tab:hasil_add_users}
	\begin{tabular}{|c|c|p{12cm}|}
		\toprule
		Langkah & Hasil & Aksi\\
		\midrule
		1 & Sukses & Langkah \textit{Login} dapat dilihat pada \ref{subsubsec:hasil_login}.\\
		2 & Sukses & Menekan tombol \textit{Tab} untuk navigasi ke \textit{sidebar} sampai ChromeVox membacakan tautan untuk menu \textit{Users}. Setelah itu pengguna menekan tombol \textit{Enter} pada keyboard.\\
		3 & Sukses & Menekan tombol \textit{Tab} sampai ChromeVox membacakan tautan \textit{Add Users} kemudian pengguna menekan tombol \textit{Enter} pada \textit{keyboard}.\\
		4 & Sukses & Menekan tombol \textit{Tab} sampai ChromeVox membacakan \textit{checkbox} untuk \textit{Send mail}. Kemudian pengguna menekan \textit{space bar} pada \textit{keyboard} untuk mengubah nilainya menjadi \textit{''checked''}.\\
		5 & Sukses & Menekan tombol \textit{Tab} sampai ChromeVox membacakan \textit{''Delay''}. Setelah itu pengguna mengisi dengan nilai ''2''.\\
		6 & Sukses & Menekan tombol \textit{Tab} sampai ChromeVox membacakan \textit{''Command for creating new user''}. Setelah itu pengguna mengisi sesuai spesifikasi yang sudah ditentukan. Pengguna dapat melanjutkan navigasi dengan menekan tombol \textit{Tab} setelah mematikan fitur \textit{tab indent} pada \textit{textarea} dengan menekan tombol \textit{Esc} pada \textit{keyboard}.\\
		7 & Sukses & Menekan tombol \textit{Tab} sampai ChromeVox membacakan tombol \textit{Add Users} kemudian menekan tombol \textit{Enter} pada \textit{keyboard}.\\
		\bottomrule
	\end{tabular}
\end{table}

\subsubsection{Delete User}
\label{subsubsec:hasil_delete_user}
Berikut adalah hasil pengujian \textit{Delete User}:

\begin{table}[H]
	\centering
	\caption{Tabel hasil pengujian \textit{Delete User}}
	\label{tab:hasil_delete_user}
	\begin{tabular}{|c|c|p{12cm}|}
		\toprule
		Langkah & Hasil & Aksi\\
		\midrule
		1 & Sukses & Langkah \textit{Login} dapat dilihat pada \ref{subsubsec:hasil_login}.\\
		2 & Sukses & Menekan tombol \textit{Tab} untuk navigasi ke \textit{sidebar} sampai ChromeVox membacakan tautan untuk menu \textit{Users}. Setelah itu pengguna menekan tombol \textit{Enter} pada keyboard.\\
		3 & Sukses & Menavigasikan ChromeVox sampai membacakan baris \textit{User} yang akan dihapus pada tabel.\\
		4 & Sukses & Menekan tombol navigasi maju ChromeVox sampai membacakan \textit{Delete User} kemudian pengguna menekan tombol \textit{Enter} pada \textit{keyboard}.\\
		5 & Sukses & Menekan tombol navigasi maju ChromeVox sampai membacakan tombol \textit{Yes, Delete} kemudian pengguna menekan tombol \textit{Enter} pada \textit{keyboard}.\\
		\bottomrule
	\end{tabular}
\end{table}

\subsubsection{Delete User Submissions}
\label{subsubsec:hasil_delete_user_submissions}
Berikut adalah hasil pengujian \textit{Delete User Submissions}:

\begin{table}[H]
	\centering
	\caption{Tabel hasil pengujian \textit{Delete User Submissions}}
	\label{tab:hasil_delete_user_submissions}
	\begin{tabular}{|c|c|p{12cm}|}
		\toprule
		Langkah & Hasil & Aksi\\
		\midrule
		1 & Sukses & Langkah \textit{Login} dapat dilihat pada \ref{subsubsec:hasil_login}.\\
		2 & Sukses & Menekan tombol \textit{Tab} untuk navigasi ke \textit{sidebar} sampai ChromeVox membacakan tautan untuk menu \textit{Users}. Setelah itu pengguna menekan tombol \textit{Enter} pada keyboard.\\
		3 & Sukses & Menavigasikan ChromeVox sampai membacakan baris \textit{User} yang akan dihapus \textit{Submissions} pada tabel.\\
		4 & Sukses & Menekan tombol navigasi maju ChromeVox sampai membacakan \textit{Delete User Submissions} kemudian pengguna menekan tombol \textit{Enter} pada \textit{keyboard}.\\
		5 & Sukses & Menekan tombol navigasi maju ChromeVox sampai membacakan tombol \textit{Yes, Delete} kemudian pengguna menekan tombol \textit{Enter} pada \textit{keyboard}.\\
		\bottomrule
	\end{tabular}
\end{table}

\subsubsection{Edit User}
\label{subsubsec:hasil_edit_user}
Berikut adalah hasil pengujian \textit{Edit User}:

\begin{table}[H]
	\centering
	\caption{Tabel hasil pengujian \textit{Edit User}}
	\label{tab:hasil_edit_user}
	\begin{tabular}{|c|c|p{12cm}|}
		\toprule
		Langkah & Hasil & Aksi\\
		\midrule
		1 & Sukses & Langkah \textit{Login} dapat dilihat pada \ref{subsubsec:hasil_login}.\\
		2 & Sukses & Menekan tombol \textit{Tab} untuk navigasi ke \textit{sidebar} sampai ChromeVox membacakan tautan untuk menu \textit{Users}. Setelah itu pengguna menekan tombol \textit{Enter} pada keyboard.\\
		3 & Sukses & Menavigasikan ChromeVox sampai membacakan baris \textit{User} yang akan dihapus \textit{Submissions} pada tabel.\\
		4 & Sukses & Menekan tombol navigasi maju ChromeVox sampai membacakan \textit{Edit} kemudian pengguna menekan tombol \textit{Enter} pada \textit{keyboard}.\\
		5 & Sukses & Menekan tombol \textit{Tab} sampai ChromeVox membacakan bidang masukan \textit{Name}. Setelah itu pengguna mengisi dengan nilai \textit{''student''}.\\
		6 & Sukses & Menekan tombol \textit{Tab} sampai ChromeVox membacakan bidang masukan \textit{User Role}. Setelah itu pengguna mengubah isi \textit{combobox} dengan menekan tombol panah atas atau bawah pada \textit{keyboard} sampai ChromeVox membacakan \textit{''student''}.\\
		7 & Sukses & Menekan tombol \textit{Tab} sampai ChromeVox membacakan tombol \textit{Save} kemudian pengguna menekan tombol \textit{Enter} pada \textit{keyboard}.\\
		\bottomrule
	\end{tabular}
\end{table}

\subsubsection{Add Notification}
\label{subsubsec:hasil_add_notification}
Berikut adalah hasil pengujian \textit{Add Notification}:

\begin{table}[H]
	\centering
	\caption{Tabel hasil pengujian \textit{Add Notification}}
	\label{tab:hasil_add_notification}
	\begin{tabular}{|c|c|p{12cm}|}
		\toprule
		Langkah & Hasil & Aksi\\
		\midrule
		1 & Sukses & Langkah \textit{Login} dapat dilihat pada \ref{subsubsec:hasil_login}.\\
		2 & Sukses & Menekan tombol \textit{Tab} untuk navigasi ke \textit{sidebar} sampai ChromeVox membacakan tautan untuk menu \textit{Notifications}. Setelah itu pengguna menekan tombol \textit{Enter} pada keyboard.\\
		3 & Sukses & Menekan tombol \textit{Tab} sampai ChromeVox membacakan tautan \textit{New} kemudian pengguna menekan tombol \textit{Enter} pada keyboard.\\
		4 & Sukses & Menekan tombol \textit{Tab} sampai ChromeVox membacakan bidang masukan \textit{Title}. Setelah itu pengguna mengisi dengan nilai \textit{''Judul''}.\\
		5 & Sukses & Menekan tombol \textit{Tab} sampai ChromeVox membacakan bidang masukan \textit{Text}. Setelah itu pengguna mengisi dengan nilai \textit{''Isi Notifikasi''}.\\
		6 & Sukses & Menekan tombol \textit{Tab} sampai ChromeVox membacakan tombol \textit{Add} kemudian pengguna menekan tombol \textit{Enter} pada \textit{keyboard}.\\
		\bottomrule
	\end{tabular}
\end{table}

\subsubsection{Add Assignment}
\label{subsubsec:hasil_add_assignment}
Berikut adalah hasil pengujian \textit{Add Assignment}:

\begin{table}[H]
	\centering
	\caption{Tabel hasil pengujian \textit{Add Assignment}}
	\label{tab:hasil_add_assignment}
	\begin{tabular}{|c|c|p{12cm}|}
		\toprule
		Langkah & Hasil & Aksi\\
		\midrule
		1 & Sukses & Langkah \textit{Login} dapat dilihat pada \ref{subsubsec:hasil_login}.\\
		2 & Sukses & Menekan tombol \textit{Tab} untuk navigasi ke \textit{sidebar} sampai ChromeVox membacakan tautan untuk menu \textit{Assignments}. Setelah itu pengguna menekan tombol \textit{Enter} pada keyboard.\\
		3 & Sukses & Menekan tombol \textit{Tab} sampai ChromeVox membacakan tautan \textit{Add} kemudian pengguna menekan tombol \textit{Enter} pada keyboard.\\
		4 & Sukses & Menekan tombol \textit{Tab} sampai ChromeVox membacakan bidang masukan \textit{Assignment Name}. Setelah itu pengguna mengisi dengan nilai \textit{''Assignment 1''}.\\
		5 & Sukses & Menekan tombol \textit{Tab} sampai ChromeVox membacakan bidang masukan \textit{Participants}. Setelah itu pengguna mengisi dengan nilai \textit{''All''}. Pengguna dapat melanjutkan navigasi dengan menekan tombol \textit{Tab} setelah mematikan fitur \textit{tab indent} pada \textit{textarea} dengan menekan tombol \textit{Esc} pada \textit{keyboard}.\\
		6 & Sukses & Menekan tombol \textit{Tab} sampai ChromeVox membacakan \textit{PDF File} kemudian pengguna menekan tombol \textit{Enter} pada \textit{keyboard}. Setelah itu pengguna memilih \textit{file PDF} yang akan diunggah.\\
		7 & Sukses & Menekan tombol \textit{Tab} sampai ChromeVox membacakan \textit{checkbox} untuk \textit{Scoreboard}. Kemudian pengguna menekan \textit{space bar} pada \textit{keyboard} untuk mengubah nilainya menjadi \textit{''checked''}.\\
		8 & Sukses & Menekan tombol \textit{Tab} sampai ChromeVox membacakan tombol \textit{Add ASsignment} kemudian pengguna menekan tombol \textit{Enter} pada \textit{keyboard}.\\
		\bottomrule
	\end{tabular}
\end{table}

\subsubsection{Delete Assignment}
\label{subsubsec:hasil_delete_assignment}
Berikut adalah hasil pengujian \textit{Delete Assignment}:

\begin{table}[H]
	\centering
	\caption{Tabel hasil pengujian \textit{Delete Assignment}}
	\label{tab:hasil_delete_assignment}
	\begin{tabular}{|c|c|p{12cm}|}
		\toprule
		Langkah & Hasil & Aksi\\
		\midrule
		1 & Sukses & Langkah \textit{Login} dapat dilihat pada \ref{subsubsec:hasil_login}.\\
		2 & Sukses & Menekan tombol \textit{Tab} untuk navigasi ke \textit{sidebar} sampai ChromeVox membacakan tautan untuk menu \textit{Assignments}. Setelah itu pengguna menekan tombol \textit{Enter} pada keyboard.\\
		3 & Sukses & Menavigasikan ChromeVox sampai membacakan baris \textit{Assignment} yang akan dihapus pada tabel.\\
		4 & Sukses & Menekan tombol navigasi maju ChromeVox sampai membacakan \textit{Delete} kemudian pengguna menekan tombol \textit{Enter} pada \textit{keyboard}.\\
		5 & Sukses & Menekan tombol \textit{Tab} sampai ChromeVox membacakan tombol \textit{Delete this assignment} kemudian pengguna menekan tombol \textit{Enter} pada \textit{keyboard}.\\
		\bottomrule
	\end{tabular}
\end{table}

\subsubsection{Edit Assignment}
\label{subsubsec:hasil_edit_assignment}
Berikut adalah hasil pengujian \textit{Edit Assignment}:

\begin{table}[H]
	\centering
	\caption{Tabel hasil pengujian \textit{Edit Assignment}}
	\label{tab:hasil_edit_assignment}
	\begin{tabular}{|c|c|p{12cm}|}
		\toprule
		Langkah & Hasil & Aksi\\
		\midrule
		1 & Sukses & Langkah \textit{Login} dapat dilihat pada \ref{subsubsec:hasil_login}.\\
		2 & Sukses & Menekan tombol \textit{Tab} untuk navigasi ke \textit{sidebar} sampai ChromeVox membacakan tautan untuk menu \textit{Assignments}. Setelah itu pengguna menekan tombol \textit{Enter} pada keyboard.\\
		3 & Sukses & Menavigasikan ChromeVox sampai membacakan baris \textit{Assignment} yang akan disunting pada tabel.\\
		4 & Sukses & Menekan tombol navigasi maju ChromeVox sampai membacakan \textit{Edit} kemudian pengguna menekan tombol \textit{Enter} pada \textit{keyboard}.\\
		5 & Sukses & Menekan tombol \textit{Tab} sampai ChromeVox membacakan bidang masukan \textit{Extra Time}. Setelah itu pengguna mengisi dengan nilai \textit{''0*60''}.\\
		6 & Sukses & Menekan tombol \textit{Tab} sampai ChromeVox membacakan bidang masukan \textit{Participants}. Setelah itu pengguna mengisi dengan nilai \textit{''All''}. Pengguna dapat melanjutkan navigasi dengan menekan tombol \textit{Tab} setelah mematikan fitur \textit{tab indent} pada \textit{textarea} dengan menekan tombol \textit{Esc} pada \textit{keyboard}.\\
		7 & Sukses & Menekan tombol \textit{Tab} sampai ChromeVox membacakan \textit{PDF File} kemudian pengguna menekan tombol \textit{Enter} pada \textit{keyboard}. Setelah itu pengguna memilih \textit{file PDF} yang akan diunggah.\\
		8 & Sukses & Menekan tombol \textit{Tab} sampai ChromeVox membacakan \textit{checkbox} untuk \textit{Open}. Kemudian pengguna menekan \textit{space bar} pada \textit{keyboard} untuk mengubah nilainya menjadi \textit{''checked''}.\\
		9 & Sukses & Menekan tombol \textit{Tab} sampai ChromeVox membacakan tombol \textit{Edit Assignment} kemudian pengguna menekan tombol \textit{Enter} pada \textit{keyboard}.\\
		\bottomrule
	\end{tabular}
\end{table}

\subsubsection{Edit Problem Description(Markdown)}
\label{subsubsec:hasil_edit_problem_description_markdown}
Berikut adalah hasil pengujian \textit{Edit Problem Description(Markdown)}:

\begin{table}[H]
	\centering
	\caption{Tabel hasil pengujian \textit{Edit Problem Description(Markdown)}}
	\label{tab:hasil_edit_problem_description_markdown}
	\begin{tabular}{|c|c|p{12cm}|}
		\toprule
		Langkah & Hasil & Aksi\\
		\midrule
		1 & Sukses & Langkah \textit{Login} dapat dilihat pada \ref{subsubsec:hasil_login}.\\
		2 & Sukses & Menekan tombol \textit{Tab} untuk navigasi ke \textit{topbar} sampai ChromeVox membacakan \textit{Assignment} yang akan disunting deskripsinya. Setelah itu pengguna menekan tombol \textit{Enter} pada keyboard.\\
		3 & Sukses & Menekan tombol \textit{Tab} untuk navigasi ke \textit{sidebar} sampai ChromeVox membacakan tautan untuk menu \textit{Problems}. Setelah itu pengguna menekan tombol \textit{Enter} pada keyboard.\\
		4 & Sukses & Menekan tombol \textit{Tab} sampai ChromeVox membacakan \textit{Problem} pada tabel yang akan disunting deskripsinya kemudian pengguna menekan tombol \textit{Enter} pada \textit{keyboard}.\\
		5 & Sukses & Menekan tombol \textit{Tab} sampai ChromeVox membacakan tautan \textit{Edit Markdown} kemudian pengguna menekan tombol \textit{Enter} pada \textit{keyboard}.\\
		6 & Sukses & Menekan tombol \textit{Tab} sampai ChromeVox membacakan bidang masukan \textit{Markdown Editor}. Setelah itu pengguna mengisi deskripsi sesuai dengan spesifikasinya. Pengguna dapat melanjutkan navigasi dengan menekan tombol \textit{Tab} setelah mematikan fitur \textit{tab indent} pada \textit{textarea} dengan menekan tombol \textit{Esc} pada \textit{keyboard}.\\
		7 & Sukses & Menekan tombol \textit{Tab} sampai ChromeVox membacakan tombol \textit{Save} kemudian pengguna menekan tombol \textit{Enter} pada \textit{keyboard}.\\
		\bottomrule
	\end{tabular}
\end{table}

\subsubsection{Edit Problem Description(HTML)}
\label{subsubsec:hasil_edit_problem_description_html}
Berikut adalah hasil pengujian \textit{Edit Problem Description(HTML)}:

\begin{table}[H]
	\centering
	\caption{Tabel hasil pengujian \textit{Edit Problem Description(HTML)}}
	\label{tab:hasil_edit_problem_description_html}
	\begin{tabular}{|c|c|p{12cm}|}
		\toprule
		Langkah & Hasil & Aksi\\
		\midrule
		1 & Sukses & Langkah \textit{Login} dapat dilihat pada \ref{subsubsec:hasil_login}.\\
		2 & Sukses & Menekan tombol \textit{Tab} untuk navigasi ke \textit{topbar} sampai ChromeVox membacakan \textit{Assignment} yang akan disunting deskripsinya. Setelah itu pengguna menekan tombol \textit{Enter} pada keyboard.\\
		3 & Sukses & Menekan tombol \textit{Tab} untuk navigasi ke \textit{sidebar} sampai ChromeVox membacakan tautan untuk menu \textit{Problems}. Setelah itu pengguna menekan tombol \textit{Enter} pada keyboard.\\
		4 & Sukses & Menekan tombol \textit{Tab} sampai ChromeVox membacakan \textit{Problem} pada tabel yang akan disunting deskripsinya kemudian pengguna menekan tombol \textit{Enter} pada \textit{keyboard}.\\
		5 & Sukses & Menekan tombol \textit{Tab} sampai ChromeVox membacakan tautan \textit{Edit HTML} kemudian pengguna menekan tombol \textit{Enter} pada \textit{keyboard}.\\
		6 & Sukses & Menekan tombol \textit{Tab} sampai ChromeVox membacakan bidang masukan \textit{Edit Text}.\\
		7 & Sukses & Menekan tombol \textit{Tab} sampai ChromeVox membacakan tombol \textit{Save} kemudian pengguna menekan tombol \textit{Enter} pada \textit{keyboard}.\\
		\bottomrule
	\end{tabular}
\end{table}

\subsubsection{Edit Problem Description(Plain HTML)}
\label{subsubsec:hasil_edit_problem_description_plain_html}
Berikut adalah hasil pengujian \textit{Edit Problem Description(Plain HTML)}:

\begin{table}[H]
	\centering
	\caption{Tabel hasil pengujian \textit{Edit Problem Description(Plain HTML)}}
	\label{tab:hasil_edit_problem_description_plain_html}
	\begin{tabular}{|c|c|p{12cm}|}
		\toprule
		Langkah & Hasil & Aksi\\
		\midrule
		1 & Sukses & Langkah \textit{Login} dapat dilihat pada \ref{subsubsec:hasil_login}.\\
		2 & Sukses & Menekan tombol \textit{Tab} untuk navigasi ke \textit{topbar} sampai ChromeVox membacakan \textit{Assignment} yang akan disunting deskripsinya. Setelah itu pengguna menekan tombol \textit{Enter} pada keyboard.\\
		3 & Sukses & Menekan tombol \textit{Tab} untuk navigasi ke \textit{sidebar} sampai ChromeVox membacakan tautan untuk menu \textit{Problems}. Setelah itu pengguna menekan tombol \textit{Enter} pada keyboard.\\
		4 & Sukses & Menekan tombol \textit{Tab} sampai ChromeVox membacakan \textit{Problem} pada tabel yang akan disunting deskripsinya kemudian pengguna menekan tombol \textit{Enter} pada \textit{keyboard}.\\
		5 & Sukses & Menekan tombol \textit{Tab} sampai ChromeVox membacakan tautan \textit{Edit Plain HTML} kemudian pengguna menekan tombol \textit{Enter} pada \textit{keyboard}.\\
		6 & Sukses & Menekan tombol \textit{Tab} sampai ChromeVox membacakan bidang masukan \textit{HTML Editor}.\\
		7 & Sukses & Menekan tombol \textit{Tab} sampai ChromeVox membacakan tombol \textit{Save} kemudian pengguna menekan tombol \textit{Enter} pada \textit{keyboard}.\\
		\bottomrule
	\end{tabular}
\end{table}

\subsubsection{Submit File}
\label{subsubsec:hasil_submit}
Berikut adalah hasil pengujian untuk mengumpulkan \textit{file}:

\begin{table}[H]
	\centering
	\caption{Tabel hasil pengujian \textit{Submit}}
	\label{tab:hasil_submit}
	\begin{tabular}{|c|c|p{12cm}|}
		\toprule
		Langkah & Hasil & Aksi\\
		\midrule
		1 & Sukses & Langkah \textit{Login} dapat dilihat pada \ref{subsubsec:hasil_login}.\\
		2 & Sukses & Menekan tombol \textit{Tab} untuk navigasi ke \textit{topbar} sampai ChromeVox membacakan \textit{Assignment} yang sesuai. Setelah itu pengguna menekan tombol \textit{Enter} pada keyboard.\\
		3 & Sukses & Menekan tombol \textit{Tab} untuk navigasi ke \textit{sidebar} sampai ChromeVox membacakan tautan untuk menu \textit{Submit}. Setelah itu pengguna menekan tombol \textit{Enter} pada keyboard.\\
		4 & Sukses & Menekan tombol \textit{Tab} sampai ChromeVox membacakan bidang masukan \textit{Problem}. Setelah itu pengguna mengubah isi \textit{combobox} dengan menekan tombol panah atas atau bawah pada \textit{keyboard} sampai ChromeVox membacakan \textit{''Problem 1''}.\\
		5 & Sukses & Menekan tombol \textit{Tab} sampai ChromeVox membacakan bidang masukan \textit{Language}. Setelah itu pengguna mengubah isi \textit{combobox} dengan menekan tombol panah atas atau bawah pada \textit{keyboard} sampai ChromeVox membacakan \textit{''Java''}.\\
		6 & Sukses & Menekan tombol \textit{Tab} sampai ChromeVox membacakan bidang masukan \textit{File}. Setelah itu pengguna menekan tombol \textit{Enter} pada keyboard. Pengguna harus memilih \textit{file} yang akan diunggah dengan benar.\\
		7 & Sukses & Menekan tombol \textit{Tab} sampai ChromeVox membacakan tombol \textit{Submit} kemudian pengguna menekan tombol \textit{Enter} pada \textit{keyboard}.\\
		\bottomrule
	\end{tabular}
\end{table}

\subsubsection{Final Submission}
\label{subsubsec:hasil_final_submission}
Berikut adalah hasil pengujian untuk memilih \textit{Final Submission}:

\begin{table}[H]
	\centering
	\caption{Tabel hasil pengujian \textit{Final Submission}}
	\label{tab:hasil_final_submission}
	\begin{tabular}{|c|c|p{12cm}|}
		\toprule
		Langkah & Hasil & Aksi\\
		\midrule
		1 & Sukses & Langkah \textit{Login} dapat dilihat pada \ref{subsubsec:hasil_login}.\\
		2 & Sukses & Menekan tombol \textit{Tab} untuk navigasi ke \textit{topbar} sampai ChromeVox membacakan \textit{Assignment} yang sesuai. Setelah itu pengguna menekan tombol \textit{Enter} pada keyboard.\\
		3 & Sukses & Menekan tombol \textit{Tab} untuk navigasi ke \textit{sidebar} sampai ChromeVox membacakan tautan untuk menu \textit{All Submissions}. Setelah itu pengguna menekan tombol \textit{Enter} pada keyboard.\\
		4 & Sukses & Menavigasikan ChromeVox sampai membacakan baris \textit{Submission} yang akan menjadi \textit{Final Submission} pada tabel.\\
		5 & Sukses & Menavigasikan ChromeVox sampai membacakan tombol \textit{Set Final} pada \textit{Submission} yang akan menjadi \textit{Final Submission} pada tabel. Kemudian pengguna menekan tombol \textit{Spacebar} pada \textit{keyboard}.\\
		\bottomrule
	\end{tabular}
\end{table}