\chapter{Implementasi}
\label{chap:implementasi}

Pada Bab ini akan dijelaskan bagaimana implementasi untuk menaikkan tingkat kepatuhan aplikasi \textit{SharIF Judge}. Bagian ini akan dibagi menjadi tiga bagian yaitu implementasi ke \textit{level A, AA, AAA}.

\subsection{Implementasi Ke \textit{Level A}}
\label{subsec:implementasi_A}
TODO

\subsubsection{Implementasi Kriteria Sukses 1.1.1 Non-text Content}
\label{subsubsec:implementasi_A_1.1.1}

\subsubsection{Implementasi Kriteria Sukses 1.3.1 Info dan Relationships}
\label{subsubsec:implementasi_A_1.3.1}

\subsubsection{Implementasi Kriteria Sukses 2.1.1 Keyboard}
\label{subsubsec:implementasi_A_2.1.1}
TODO

\subsubsection{Implementasi Kriteria Sukses 2.1.2 No Keyboard Trap}
\label{subsubsec:implementasi_A_2.1.2}
TODO

\subsubsection{Implementasi Kriteria Sukses 2.4.1 Bypass Blocks}
\label{subsubsec:implementasi_A_2.4.1}
TODO

\subsubsection{Implementasi Kriteria Sukses 2.4.4 Link Purpose (In Context)}
\label{subsubsec:implementasi_A_2.4.4}

- logo pdf sudah diberi label pada kriteria 1.1.1

\subsubsection{Implementasi Kriteria Sukses 3.1.1 Language of Page}
\label{subsubsec:implementasi_A_3.1.1}

\subsubsection{Implementasi Kriteria Sukses 3.3.2 Labels or Instructions}
\label{subsubsec:implementasi_A_3.3.2}

- pada halaman add assignment bagian problems sudah diberi label pada kriteria 1.3.1

- pada halaman problems bagian edit markdown dan plain html, serta halaman add user sudah diberi label pada kriteria 1.3.1

\subsubsection{Implementasi Kriteria Sukses 4.1.1 Parsing}
\label{subsubsec:implementasi_A_4.1.1}

- tanya pada halaman add notification bagian textarea