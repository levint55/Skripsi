\chapter{Implementasi}
\label{chap:implementasi}

Pada Bab ini akan dijelaskan bagaimana implementasi untuk menaikkan tingkat kepatuhan aplikasi \textit{SharIF Judge}. Bagian ini akan dibagi menjadi tiga bagian yaitu implementasi ke \textit{level A, AA, AAA}.

\section{Implementasi Ke \textit{Level A}}
\label{sec:implementasi_A}
TODO

\subsection{Implementasi Kriteria Sukses 1.1.1 Non-text Content}
\label{subsec:implementasi_A_1.1.1}

Berikut adalah perubahan yang perlu dilakukan untuk memenuhi Kriteria Sukses 1.1.1:

\begin{itemize}
	\item Pada bagian menu atas logo \textit{SharIF Judge} tidak memiliki teks alternatif yang menjelaskan gambar dari logo tersebut sehingga perlu ditambahkan sebuah label untuk menjelaskan gambar tersebut. Perubahan yang terjadi ada pada \textit{file} \textit{/application/views/templates/top\_bar.twig}, potongan kode dapat dilihat pada \ref{lst_1.1.1:1}.
	
\begin{lstlisting}[language=diff, caption=Perubahan pada \textit{file} \textit{top\_bar.twig}, label=lst_1.1.1:1, basicstyle=\ttfamily, frame=single,
columns=fullflexible, keepspaces=true, breaklines=true]
@@ -41,7 +41,7 @@
</div>
<div id="shj_logo" class="top_left">
<a href="{{ site_url('/') }}">
- <img src="{{ base_url('assets/images/logo_small.png') }}"/>
+ <img src="{{ base_url('assets/images/logo_small.png') }}" aria-label="Logo SharIF Judge"/>
<h1 class="shjlogo-text">SharIF <span>Judge</span></h1>
</a>
</div>
\end{lstlisting}

	\item Pada halaman \textit{Assignments} terdapat gambar \textit{PDF} di daftar \textit{Assignment} yang tidak memiliki teks alternatif yang menjelaskan gambar tersebut sehingga perlu ditambahkan sebuah label untuk menjelaskan gambar tersebut. Perubahan yang terjadi ada pada \textit{file} \textit{/application/views/pages/assignments.twig}, potongan kode dapat dilihat pada \ref{lst_1.1.1:2}.
	
\begin{lstlisting}[language=diff, caption=Perubahan pada \textit{file} \textit{assignments.twig}, label=lst_1.1.1:2, basicstyle=\ttfamily, frame=single,
columns=fullflexible, keepspaces=true, breaklines=true]
@@ -69,7 +69,7 @@

</td>
<td>
- <a href="{{ site_url('assignments/pdf/'~item.id) }}"><img src="{{ base_url('assets/images/pdf.svg') }}" /></a>
+ <a href="{{ site_url('assignments/pdf/'~item.id) }}"><img src="{{ base_url('assets/images/pdf.svg') }}" aria-label="Download PDF For Assignment {{ item.name }}"/></a>
</td>

<td>
\end{lstlisting}
\end{itemize}

\subsection{Implementasi Kriteria Sukses 1.3.1 Info dan Relationships}
\label{subsec:implementasi_A_1.3.1}

Berikut adalah perubahan yang perlu dilakukan untuk memenuhi Kriteria Sukses 1.3.1:

\begin{itemize}
	\item Semua elemen bagian tabel \textit{Problems} pada halaman \textit{Add Assignment} perlu diberi label untuk memberikan informasi sesuai elemennya. Perubahan yang terjadi ada pada \textit{file} \textit{/application/views/pages/admin/add\_assignment.twig}, potongan kode dapat dilihat pada \ref{lst_1.3.1:1}.
	
\begin{lstlisting}[language=diff, caption=Perubahan pada \textit{file} \textit{add\_assignment.twig}, label=lst_1.3.1:1, basicstyle=\ttfamily, frame=single,
columns=fullflexible, keepspaces=true, breaklines=true]
@@ -201,16 +201,16 @@

<tr>
<td>{{ problem.id }}</td>
- 	<td><input type="text" name="name[]" class="sharif_input short" value="{{ problem.name }}"/></td>
- 	<td><input type="text" name="score[]" class="sharif_input tiny2" value="{{ problem.score }}"/></td>
- 	<td><input type="text" name="c_time_limit[]" class="sharif_input tiny2" value="{{ problem.c_time_limit }}"/></td>
- 	<td><input type="text" name="python_time_limit[]" class="sharif_input tiny2" value="{{ problem.python_time_limit }}"/></td>
- 	<td><input type="text" name="java_time_limit[]" class="sharif_input tiny2" value="{{ problem.java_time_limit }}"/></td>
- 	<td><input type="text" name="memory_limit[]" class="sharif_input tiny" value="{{ problem.memory_limit }}"/></td>
- 	<td><input type="text" name="languages[]" class="sharif_input short2" value="{{ problem.allowed_languages }}"/></td>
- 	<td><input type="text" name="diff_cmd[]" class="sharif_input tiny" value="{{ problem.diff_cmd }}"/></td>
- 	<td><input type="text" name="diff_arg[]" class="sharif_input tiny" value="{{ problem.diff_arg }}"/></td>
- 	<td><input type="checkbox" name="is_upload_only[]" class="check" value="{{ problem.id }}" {{ problem.is_upload_only ? 'checked' }}/></td>
+ 	<td><input type="text" name="name[]" class="sharif_input short" value="{{ problem.name }}" aria-label="Problem Name"/></td>
+ 	<td><input type="text" name="score[]" class="sharif_input tiny2" value="{{ problem.score }}" aria-label="Score"/></td>
+ 	<td><input type="text" name="c_time_limit[]" class="sharif_input tiny2" value="{{ problem.c_time_limit }}" aria-label="Time Limit for C"/></td>
+ 	<td><input type="text" name="python_time_limit[]" class="sharif_input tiny2" value="{{ problem.python_time_limit }}" aria-label="Time Limit for Python"/></td>
+ 	<td><input type="text" name="java_time_limit[]" class="sharif_input tiny2" value="{{ problem.java_time_limit }}" aria-label="Time Limit for Java"/></td>
+ 	<td><input type="text" name="memory_limit[]" class="sharif_input tiny" value="{{ problem.memory_limit }}" aria-label="Memory Limit"/></td>
+ 	<td><input type="text" name="languages[]" class="sharif_input short2" value="{{ problem.allowed_languages }}" aria-label="Allowed Languages"/></td>
+ 	<td><input type="text" name="diff_cmd[]" class="sharif_input tiny" value="{{ problem.diff_cmd }}" aria-label="Diff Command"/></td>
+ 	<td><input type="text" name="diff_arg[]" class="sharif_input tiny" value="{{ problem.diff_arg }}" aria-label="Diff Argument"/></td>
+ 	<td><input type="checkbox" name="is_upload_only[]" class="check" value="{{ problem.id }}" {{ problem.is_upload_only ? 'checked' }} aria-label="Upload Only"/></td>
<td><i class="fa fa-times-circle fa-lg color1 delete_problem pointer"></i></td>
</tr>

\end{lstlisting}

	\item \textit{Checkbox} dan \textit{Textarea} pada halaman \textit{Add Users} perlu diberi label untuk memberikan informasi elemen tersebut. Perubahan yang terjadi ada pada \textit{file} \textit{/application/views/pages/admin/add\_user.twig}, potongan kode dapat dilihat pada \ref{lst_1.3.1:2}.
	
\begin{lstlisting}[language=diff, caption=Perubahan pada \textit{file} \textit{add\_user.twig}, label=lst_1.3.1:2, basicstyle=\ttfamily, frame=single,
columns=fullflexible, keepspaces=true, breaklines=true]
@@ -56,10 +56,10 @@
<li>If you want to send passwords by email, do not add too many users at one time. This may result in mail delivery fail.</li>
</ul>
<p class="input_p">
- 	<input type="checkbox" name="send_mail" id="send_mail" /> Send usernames and passwords by email (Waits <input type="text" name="delay" id="delay" class="sharif_input tiny" value="2"/> second(s) before sending each email, so please be patient).
+ 	<input type="checkbox" name="send_mail" id="send_mail" aria-label="Send mail"/> Send usernames and passwords by email (Waits <input type="text" name="delay" id="delay" class="sharif_input tiny" value="2" aria-label="Delay"/> second(s) before sending each email, so please be patient).
</p>
<p class="input_p">
- <textarea name="new_users" id="new_users" rows="20" cols="80" class="sharif_input">
+ <textarea name="new_users" id="new_users" rows="20" cols="80" class="sharif_input" aria-label="Command for creating new user">
# Lines starting with a # sign are comments.
# Each line (except comments) represents a user.
# The syntax of each line is:
\end{lstlisting}

	\item \textit{Textarea} pada halaman \textit{Edit Problem Markdown} perlu diberi label untuk memberikan informasi elemen tersebut. Perubahan yang terjadi ada pada \textit{file} \textit{/application/views/pages/admin/edit\_problem\_md.twig}, potongan kode dapat dilihat pada \ref{lst_1.3.1:3}.

\begin{lstlisting}[language=diff, caption=Perubahan pada \textit{file} \textit{edit\_problem\_md.twig}, label=lst_1.3.1:3, basicstyle=\ttfamily, frame=single,
columns=fullflexible, keepspaces=true, breaklines=true]
@@ -177,7 +177,7 @@ Violets are blue.
</p>
{{ form_open("problems/edit/md/#{description_assignment.id}/#{problem.id}") }}
<p class="input_p">
- 	<textarea dir="auto" name="text" rows="30" cols="75" class="sharif_input" id="md_editor">{{ problem.description }}</textarea>
+ 	<textarea dir="auto" name="text" rows="30" cols="75" class="sharif_input" id="md_editor" aria-label="Markdown Editor">{{ problem.description }}</textarea>
</p>
<p class="input_p">
<input type="submit" value="Save" class="sharif_input"/>
\end{lstlisting}

	\item \textit{Textarea} pada halaman \textit{Edit Problem Plain HTML} perlu diberi label untuk memberikan informasi elemen tersebut. Perubahan yang terjadi ada pada \textit{file} \textit{/application/views/pages/admin/edit\_problem\_plain.twig}, potongan kode dapat dilihat pada \ref{lst_1.3.1:4}.
	
\begin{lstlisting}[language=diff, caption=Perubahan pada \textit{file} \textit{edit\_problem\_plain.twig}, label=lst_1.3.1:4, basicstyle=\ttfamily, frame=single,
columns=fullflexible, keepspaces=true, breaklines=true]
@@ -33,7 +33,7 @@
</p>
{{ form_open("problems/edit/html/#{description_assignment.id}/#{problem.id}") }}
<p class="input_p">
- 	<textarea name="text" rows="30" cols="80" class="sharif_input" id="html_editor">{{ problem.description }}</textarea>
+ 	<textarea name="text" rows="30" cols="80" class="sharif_input" id="html_editor" aria-label="HTML Editor">{{ problem.description }}</textarea>
</p>
<p class="input_p">
<input type="submit" value="Save" class="sharif_input"/>
\end{lstlisting}
	
	\item \textit{Dropdown} pada halaman \textit{Problems} perlu diberi label untuk memberikan informasi elemen tersebut. Perubahan yang terjadi ada pada \textit{file} \textit{/application/views/pages/problems.twig}, potongan kode dapat dilihat pada \ref{lst_1.3.1:5}.
	
\begin{lstlisting}[language=diff, caption=Perubahan pada \textit{file} \textit{problems.twig} untuk \textit{Dropdown Select Language}, label=lst_1.3.1:5, basicstyle=\ttfamily, frame=single,
columns=fullflexible, keepspaces=true, breaklines=true]
@@ -96,7 +96,7 @@ 
$(document).ready(function(){
<input type="hidden" name="problem" value="{{ problem.id }}"/>

<p class="input_p">
- 	<select id="languages" name="language" class="sharif_input full-width">
+ 	<select id="languages" name="language" class="sharif_input full-width" aria-label="Select Language">
<option value="0" selected="selected">-- Select Language --</option>

<option value="{{ l }}">{{ l }}</option>
\end{lstlisting}

	\item Masukan \textit{Upload File} pada halaman \textit{Problems} perlu diberi label untuk memberikan informasi elemen tersebut. Perubahan yang terjadi ada pada \textit{file} \textit{/application/views/pages/problems.twig}, potongan kode dapat dilihat pada \ref{lst_1.3.1:6}.
	
\begin{lstlisting}[language=diff, caption=Perubahan pada \textit{file} \textit{problems.twig} untuk masukan \textit{Upload \textit{file}}, label=lst_1.3.1:6, basicstyle=\ttfamily, frame=single,
columns=fullflexible, keepspaces=true, breaklines=true]
@@ -104,7 +104,7 @@ 
$(document).ready(function(){
</select>
</p>
<p class="input_p">
- 	<input type="file" id="file" class="sharif_input full-width" name="userfile"/>
+ 	<input type="file" id="file" class="sharif_input full-width" name="userfile" aria-label="Upload File"/>
</p>
<p class="input_p">
<input type="submit" value="Submit" class="sharif_input"/>
\end{lstlisting}
\end{itemize}

\subsection{Implementasi Kriteria Sukses 2.1.1 Keyboard}
\label{subsec:implementasi_A_2.1.1}
TODO

\subsection{Implementasi Kriteria Sukses 2.1.2 No Keyboard Trap}
\label{subsec:implementasi_A_2.1.2}
TODO

\subsection{Implementasi Kriteria Sukses 2.4.1 Bypass Blocks}
\label{subsec:implementasi_A_2.4.1}
TODO

\subsection{Implementasi Kriteria Sukses 2.4.4 Link Purpose (In Context)}
\label{subsec:implementasi_A_2.4.4}

Berikut adalah perubahan yang perlu dilakukan untuk memenuhi Kriteria Sukses 2.4.4:

\begin{itemize}
	\item Seluruh link pada \textit{sidebar} perlu diberikan label yang menjelaskan link tersebut. Perubahan yang terjadi ada pada \textit{file} \textit{/application/views/templates/side\_bar.twig}, potongan kode dapat dilihat pada \ref{lst_2.4.4:1}.
	
\begin{lstlisting}[language=diff, caption=Perubahan pada \textit{file} \textit{side\_bar.twig}, label=lst_2.4.4:1, basicstyle=\ttfamily, frame=single,
columns=fullflexible, keepspaces=true, breaklines=true]
@@ -6,78 +6,78 @@
<div id="side_bar" class="sidebar_open">
<ul>
<li class="color-dashboard{{ selected=='dashboard' ? ' selected' }}">
- 	<a href="{{ site_url('dashboard') }}">
+ 	<a href="{{ site_url('dashboard') }}" aria-labelledby="dashboard-label">
<i class="fa fa-dashboard fa-lg"></i>
- 		<span class="sidebar_text">Dashboard</span>
+ 		<span class="sidebar_text" id="dashboard-label">Dashboard</span>
</a>
</li>

<li class="color-settings{{ selected=='settings' ? ' selected' }}">
- 	<a href="{{ site_url('settings') }}">
+ 	<a href="{{ site_url('settings') }}" aria-labelledby="settings-label">
<i class="fa fa-gear fa-lg"></i>
- 		<span class="sidebar_text">Settings</span>
+ 		<span class="sidebar_text" id="settings-label">Settings</span>
</a>
</li>
<li class="color-users{{ selected=='users' ? ' selected' }}">
- 	<a href="{{ site_url('users') }}">
+ 	<a href="{{ site_url('users') }}" aria-labelledby="users-label">
<i class="fa fa-users fa-lg"></i>
- 		<span class="sidebar_text">Users</span>
+ 		<span class="sidebar_text" id="users-label">Users</span>
</a>
</li>

<li class="color-notifications{{ selected=='notifications' ? ' selected' }}">
- 	<a href="{{ site_url('notifications') }}">
+ 	<a href="{{ site_url('notifications') }}" aria-labelledby="notifications-label">
<i class="fa fa-bell fa-lg"></i>
- 		<span class="sidebar_text">Notifications</span>
+ 		<span class="sidebar_text" id="notifications-label">Notifications</span>
</a>
</li>
<li class="color-assignments{{ selected=='assignments' ? ' selected' }}">
- 	<a href="{{ site_url('assignments') }}">
+ 	<a href="{{ site_url('assignments') }}" aria-labelledby="assignments-label">
<i class="fa fa-folder-open fa-lg"></i>
- 		<span class="sidebar_text">Assignments</span>
+ 		<span class="sidebar_text" id="assignments-label">Assignments</span>
</a>
</li>
<li class="color-problems{{ selected=='problems' ? ' selected' }}">
- 	<a href="{{ site_url('problems') }}">
+ 	<a href="{{ site_url('problems') }}" aria-labelledby="problems-label">
<i class="fa fa-puzzle-piece fa-lg"></i>
- 		<span class="sidebar_text">Problems</span>
+ 		<span class="sidebar_text" id="problems-label">Problems</span>
</a>
</li>
<li class="color-submit{{ selected=='submit' ? ' selected' }}">
- 	<a href="{{ site_url('submit') }}">
+ 	<a href="{{ site_url('submit') }}" aria-labelledby="submit-label">
<i class="fa fa-location-arrow fa-lg"></i>
- 		<span class="sidebar_text">Submit</span>
+ 		<span class="sidebar_text" id="submit-label">Submit</span>
</a>
</li>
<li class="color-final_submissions{{ selected=='final_submissions' ? ' selected' }}">
- 	<a href="{{ site_url('submissions/final') }}">
+ 	<a href="{{ site_url('submissions/final') }}" aria-labelledby="final-submission-label">
<i class="fa fa-map-marker fa-lg"></i>
- 		<span class="sidebar_text">Final Submissions</span>
+ 		<span class="sidebar_text" id="final-submission-label">Final Submissions</span>
</a>
</li>
<li class="color-all_submissions{{ selected=='all_submissions' ? ' selected' }}">
- 	<a href="{{ site_url('submissions/all') }}">
+ 	<a href="{{ site_url('submissions/all') }}" aria-labelledby="all-submission-label">
<i class="fa fa-bars fa-lg"></i>
- 		<span class="sidebar_text">All Submissions</span>
+ 		<span class="sidebar_text" id="all-submission-label">All Submissions</span>
</a>
</li>
<li class="color-scoreboard{{ selected=='scoreboard' ? ' selected' }}">
- 	<a href="{{ site_url('scoreboard') }}">
+ 	<a href="{{ site_url('scoreboard') }}" aria-labelledby="scoreboard-label">
<i class="fa fa-star fa-lg"></i>
- 		<span class="sidebar_text">Scoreboard</span>
+ 		<span class="sidebar_text" id="scoreboard-label">Scoreboard</span>
</a>
</li>
<li class="color-halloffame{{ selected=='halloffame' ? ' selected' }}">
- 	<a href="{{ site_url('halloffame') }}">
+ 	<a href="{{ site_url('halloffame') }}" aria-labelledby="hall-of-fame-label">
<i class="fa fa-list-alt fa-lg"></i>
- 		<span class="sidebar_text">Hall of Fame</span>
+ 		<span class="sidebar_text" id="hall-of-fame-label">Hall of Fame</span>
</a>
</li>

<li class="color-logs{{ selected=='logs' ? ' selected' }}">
- 	<a href="{{ site_url('logs') }}">
+ 	<a href="{{ site_url('logs') }}" aria-labelledby="24-hour-log-label">
<i class="fa fa-book fa-lg"></i>
- 		<span class="sidebar_text">24-hour Log</span>
+ 		<span class="sidebar_text" id="24-hour-log-label">24-hour Log</span>
</a>
</li>

\end{lstlisting}

	\item Tautan pada gambar \textit{Profile} yang ada di menu atas perlu diberi label untuk menjelaskan tujuan dari tautan tersebut. Perubahan yang terjadi ada pada \textit{/application/views/templates/top\_bar.twig}, potongan kode dapat dilihat pada \ref{lst_2.4.4:2}.
	
\begin{lstlisting}[language=diff, caption=Perubahan pada \textit{file} \textit{top\_bar.twig}, label=lst_2.4.4:2, basicstyle=\ttfamily, frame=single,
columns=fullflexible, keepspaces=true, breaklines=true]
@@ -5,7 +5,7 @@
#}
<div id="top_bar" class="color-{{ selected }}">
<div class="top_object shj_menu" id="user_top">
- 	<a href="{{ site_url('profile') }}" id="profile_link"><i class="fa fa-user"></i></a>
+ 	<a href="{{ site_url('profile') }}" id="profile_link" aria-label="Profile"><i class="fa fa-user"></i></a>
<div class="top_menu user-menu">
<div class="gravatar"><img src="http://www.gravatar.com/avatar/{{ md5(user.email) }}?s=70&d=identicon" /></div>
<div class="name"><i class="fa fa-user"></i> {{ user.username }}</div>
\end{lstlisting}

	\item Tautan pada gambar \textit{PDF} perlu diberi label untuk menjelaskan tujuannya. Perubahan untuk tautan pada gambar \textit{PDF} yang ada di halaman \textit{Assignments} sudah dilakukan pada \ref{lst_1.1.1:2}
\end{itemize}

\subsection{Implementasi Kriteria Sukses 3.1.1 Language of Page}
\label{subsec:implementasi_A_3.1.1}

Seluruh halaman aplikasi \textit{SharIF Judge} menggunakan Bahasa Inggris sehingga perlu diberikan atribut \textit{lang} yang bernilai \textit{''en''} pada awal \textit{tag html} untuk menunjukkan bahwa bahasa yang dipakai di halaman tersebut adalah Bahasa Inggris. Setiap kali halaman dibuka, aplikasi \textit{SharIF Judge} akan menjalankan \textit{file} \textit{base.twig} sebagai dasarnya sehingga perubahan hanya perlu dilakukan pada \textit{file} tersebut. Perubahan dapat dilihat pada \textit{/application/views/templates/base.twig}, potongan kode dapat dilihat pada \ref{lst_3.1.1:1}.

\begin{lstlisting}[language=diff, caption=Perubahan pada \textit{file} \textit{base.twig}, label=lst_3.1.1:1, basicstyle=\ttfamily, frame=single,
columns=fullflexible, keepspaces=true, breaklines=true]
@@ -4,7 +4,7 @@
# author: Mohammad Javad Naderi <mjnaderi@gmail.com>
#}
<!DOCTYPE html>
- <html>
+ <html lang="en">
<head>
<title> - SharIF Judge</title>
<meta content="text/html" charset="UTF-8">
\end{lstlisting}

\subsection{Implementasi Kriteria Sukses 3.3.2 Labels or Instructions}
\label{subsec:implementasi_A_3.3.2}

Berikut adalah perubahan yang perlu dilakukan untuk memenuhi Kriteria Sukses 3.3.2:

\begin{itemize}
	\item Semua elemen bagian tabel \textit{Problems} pada halaman \textit{Add Assignment} perlu diberi label untuk menjelaskan tujuan dari elemen tersebut. Perubahan sudah dilakukan pada \ref{lst_1.3.1:1}.
	
	\item \textit{Checkbox} dan \textit{Textarea} pada halaman \textit{Add Users} perlu diberi label yang menjelaskan tujuan dari elemen tersebut. Perubahan sudah dilakukan pada \ref{lst_1.3.1:2}.
	
	\item \textit{Textarea} pada halaman \textit{Edit Problem Markdown} perlu diberi label yang menjelaskan tujuan dari elemen tersebut. Perubahan sudah dilakukan pada \ref{lst_1.3.1:3}.
	
	\item \textit{Textarea} pada halaman \textit{Edit Problem Plain HTML} perlu diberi label yang menjelaskan tujuan dari elemen tersebut. Perubahan sudah dilakukan pada \ref{lst_1.3.1:4}.
	
	\item \textit{Dropdown} pada halaman \textit{Problems} perlu diberi label yang menjelaskan tujuan dari elemen tersebut. Perubahan sudah dilakukan pada \ref{lst_1.3.1:5}.
	
	\item Masukan \textit{Upload File} pada halaman \textit{Problems} perlu diberi label yang menjelaskan tujuan dari elemen tersebut. Perubahan sudah dilakukan pada \ref{lst_1.3.1:6}.
\end{itemize}

\subsection{Implementasi Kriteria Sukses 4.1.1 Parsing}
\label{subsec:implementasi_A_4.1.1}

\textit{Form} masukan \textit{Extra Time} pada halaman \textit{Add Assignment} memiliki atribut id yang duplikat. Atribut yang dibuang yaitu atribut \textit{id=''extra\_time''} karena sudah dipakai pada elemen \textit{extra time} pada menu bagian atas. Perubahan dapat dilihat pada \textit{file} \textit{/application/views/pages/admin/add\_assignment.twig}, potongan kode dapat dilihat pada \ref{lst_4.1.1:1}.

\begin{lstlisting}[language=diff, caption=Perubahan pada \textit{file} \textit{add\_assignment.twig}, label=lst_4.1.1:1, basicstyle=\ttfamily, frame=single,
columns=fullflexible, keepspaces=true, breaklines=true]
@@ -119,7 +119,7 @@
Extra Time (minutes)<br>
<span class="form_comment">Extra time for late submissions.</span>
</label>
- 	<input id="form_extra_time" type="text" name="extra_time" id="extra_time" class="sharif_input medium" value="{{ edit ? edit_assignment.extra_time|extra_time_formatter : set_value('extra_time') }}" />
+ 	<input id="form_extra_time" type="text" name="extra_time" class="sharif_input medium" value="{{ edit ? edit_assignment.extra_time|extra_time_formatter : set_value('extra_time') }}" />
{{ form_error('extra_time', '<div class="shj_error">', '</div>') }}
</p>
<p class="input_p clear">
\end{lstlisting}