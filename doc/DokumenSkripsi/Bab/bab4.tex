\chapter{Implementasi dan Pengujian}
\label{chap:implementasi}

Bab ini membahas bagaimana implementasi untuk menaikkan tingkat kepatuhan aplikasi \textit{SharIF Judge}.

\section{Implementasi}
\label{sec:implementasi}

Pada bagian ini akan dibahas perbaikan apa saja yang dilakukan untuk membuat kriteria sukses pada subbab \ref{sec:peningkatan_level_A}
yang belum sukses dipatuhi menjadi sukses dipatuhi. Setiap perubahan akan ditampilkan dalam format \textit{diff} dengan jarak \textit{identation} sudah dimodifikasi untuk menghemat ruang.

\subsection{Implementasi Kriteria Sukses 1.1.1 \textit{Non-text Content}}
\label{subsec:implementasi_A_1.1.1}

Berikut adalah perubahan yang perlu dilakukan untuk memenuhi kriteria sukses 1.1.1:

\begin{itemize}
	\item Pada bagian menu atas logo \textit{SharIF Judge} tidak memiliki alternatif teks yang menjelaskan gambar dari logo tersebut sehingga perlu ditambahkan sebuah label untuk menjelaskan gambar tersebut. Perubahan yang terjadi ada pada \textit{file} \textit{/application/views/templates/top\_bar.twig}.

	\item Pada halaman \textit{Assignments} terdapat gambar \textit{PDF} di daftar \textit{Assignment} yang tidak memiliki alternatif teks yang menjelaskan gambar tersebut sehingga perlu ditambahkan sebuah label untuk menjelaskan gambar tersebut. Perubahan yang terjadi ada pada \textit{file} \textit{/application/views/pages/assignments.twig}.
	
\end{itemize}

Semua perubahan kode dapat dilihat pada listing \ref{lst_1.1.1}.

\subsection{Implementasi Kriteria Sukses 1.3.1 \textit{Info and Relationships}}
\label{subsec:implementasi_A_1.3.1}

Berikut adalah perubahan yang perlu dilakukan untuk memenuhi kriteria sukses 1.3.1:

\begin{itemize}
	\item Semua elemen bagian tabel \textit{Problems} pada halaman \textit{Add Assignment} perlu diberi label untuk memberikan informasi sesuai elemennya. Perubahan yang terjadi ada pada \textit{file} \textit{/application/views/pages/admin/add\_assignment.twig}.

	\item \textit{checkbox} dan \textit{textarea} pada halaman \textit{Add Users} perlu diberi label untuk memberikan informasi elemen tersebut. Perubahan yang terjadi ada pada \textit{file} \textit{/application/views/pages/admin/add\_user.twig}.

	\item \textit{textarea} pada halaman \textit{Edit Problem Markdown} perlu diberi label untuk memberikan informasi elemen tersebut. Perubahan yang terjadi ada pada \textit{file} \textit{/application/views/pages/admin/edit\_problem\_md.twig}.

	\item \textit{textarea} pada halaman \textit{Edit Problem Plain HTML} perlu diberi label untuk memberikan informasi elemen tersebut. Perubahan yang terjadi ada pada \textit{file} \textit{/application/views/pages/admin/edit\_problem\_plain.twig}.
	
	\item \textit{Dropdown} pada halaman \textit{Problems} perlu diberi label untuk memberikan informasi elemen tersebut. Perubahan yang terjadi ada pada \textit{file} \textit{/application/views/pages/problems.twig}.

	\item Masukan \textit{Upload File} pada halaman \textit{Problems} perlu diberi label untuk memberikan informasi elemen tersebut. Perubahan yang terjadi ada pada \textit{file} \textit{/application/views/pages/problems.twig}.
	
	\item Setiap judul bagian pada halaman \textit{SharIF Judge} tidak memakai \textit{tag heading}. Perubahan yang perlu dilakukan adalah dengan cara memakai \textit{tag heading} pada setiap judul bagian.
	Perubahan yang terjadi ada pada \textit{file} \textit{/application/views/templates/base.twig}.

\end{itemize}

Semua perubahan kode dapat dilihat pada listing \ref{lst_1.3.1}.

\subsection{Implementasi Kriteria Sukses 2.1.1 \textit{Keyboard}}
\label{subsec:implementasi_A_2.1.1}
Berikut adalah perubahan yang perlu dilakukan untuk memenuhi kriteria sukses 2.1.1:

\begin{itemize}
	\item Fokus \textit{keyboard} tidak dapat terfokus pada menu \textit{Tools} pada menu bagian atas. Hal ini dapat diperbaiki dengan cara menambahkan \textit{tabindex} yang bernilai 0 sehingga fokus \textit{keyboard} dapat terfokus di elemen tersebut. Perubahan dapat dilihat pada \textit{file} \textit{/application/views/templates/top\_bar.twig}.
	
	Menu \textit{Tools} pada bagian menu atas memiliki \textit{sub-menu} yang dapat muncul ketika \textit{pointer cursor} berada diatasnya, menu ini tidak muncul ketika fokus \textit{keyboard} berada pada menu \textit{Tools}. Perlu ditambahkan fungsi untuk memunculkan \textit{sub-menu} ketika fokus \textit{keyboard} berada pada menu \textit{Tools}. \textit{Sub-menu} akan hilang jika fokus \textit{keyboard} terfokus pada \textit{sidebar}. Perubahan dapat dilihat pada \textit{file} \textit{/assets/js/shj\_functions.js}.

	\item \textit{Sub-menu} \textit{Profile} pada bagian menu atas tidak muncul ketika fokus \textit{keyboard} berada pada gambar \textit{Profile}. Gambar \textit{Profile} memiliki kelas yang sama seperti menu \textit{Tools} sehingga perubahan yang dilakukan pada kelas tersebut dapat memperbaiki masalah ini.
	
	\item \textit{Sub-menu} memilih \textit{Assignment} pada bagian menu atas tidak muncul ketika fokus \textit{keyboard} berada pada tautan \textit{Assigntment} sekarang. Tautan \textit{Assignment} memiliki kelas yang sama seperti menu \textit{Tools} sehingga perubahan yang dilakukan pada kelas tersebut dapat memperbaiki masalah ini.
	
	\textit{Sub-menu} memilih \textit{Assignment} pada bagian menu atas tidak dapat difokuskan dengan \textit{keyboard} sehingga perlu ditambahkan atribut \textit{tabindex} yang bernilai 0 agar fokus \textit{keyboard} dapat terfokus pada elemen tersebut. Perubahan dapat dilihat pada \textit{file} \textit{/application/views/templates/top\_bar.twig}.

	Pada \textit{Sub-menu} memilih \textit{Assignment} pada menu atas, pengguna tidak dapat memilih \textit{Assignment} dengan menggunakan \textit{keyboard}. Hal ini dapat diperbaiki dengan cara menambahkan fungsi yang jika dijalankan akan menjalankan fungsi yang sama ketika pengguna menekan salah satu dari \textit{list} \textit{Assignment}. Pengguna harus menekan tombol \textit{''enter''} pada \textit{keyboard} ketika fokus \textit{keyboard} berada pada elemen ini untuk menjalankan fungsi tersebut. Perubahan dapat dilihat pada \textit{file} \textit{/assets/js/shj\_functions.js}.
	
	\item Aksi untuk \textit{Delete User} dan \textit{Delete Submissions} pada halaman \textit{Users} tidak dapat difokuskan dengan \textit{keyboard} sehingga perlu diubah menjadi button agar dapat difokuskan dengan \textit{keyboard}. Perubahan dapat dilihat pada \textit{file} \textit{/application/views/pages/admin/users.twig}.

	\item \textit{checkbox} untuk memilih \textit{Assignment} pada halaman \textit{Assignments} tidak dapat difokuskan dengan \textit{keyboard} sehingga perlu diberi atribut \textit{tabindex} yang bernilai 0 agar fokus \textit{keyboard} dapat terfokus pada elemen tersebut. Perubahan dapat dilihat pada \textit{file} \textit{/application/views/pages/assignments.twig}.

	\item Tombol \textit{Add} dan \textit{Delete Problems} pada halaman \textit{Add Assignment} tidak dapat difokuskan \textit{keyboard} sehingga perlu dirubah menjadi \texttt{button} agar dapat difokuskan dengan \textit{keyboard}. Perubahan dapat dilihat pada \textit{file} \textit{/application/views/pages/admin/add\_assignment.twig}.

	\item \textit{checkbox} pada halaman \textit{All Submissions} tidak dapat difokuskan dengan \textit{keyboard} sehingga perlu ditambahkan atribut \textit{tabindex} yang bernilai 0 agar elemen dapat difokuskan dengan \textit{keyboard}. Perubahan dapat dilihat pada \textit{file} \textit{/application/views/pages/submissions.twig}.
	
	Aksi yang ada di \textit{checkbox} pada halaman \textit{All Submissions} tidak dapat dijalankan dengan \textit{keyboard} sehingga perlu ditambahkan fungsi yang jika dijalankan akan menjalankan fungsi yang sama ketika pengguna menekan \textit{checkbox} tersebut. Fungsi ini hanya bisa dijalankan jika fokus sedang berada pada \textit{checkbox} dan pengguna menekan tombol \textit{''spacebar''} pada \textit{keyboard}. Perubahan dapat dilihat pada \textit{file} \textit{/assets/js/shj\_submissions.js}.
	
	Aksi melihat status, melihat kode, melihat \textit{log} dan aksi \textit{rejudge} pada halaman \textit{All Submission} dan \textit{Final Submissions} tidak dapat difokuskan dengan \textit{keyboard} sehingga perlu ditambahkan atribut \textit{tabindex} yang bernilai 0 agar elemen dapat difokuskan dengan \textit{keyboard}. Selain itu fungsi dari elemen tersebut tidak dapat dijalankan ketika pengguna menekan tombol \textit{''enter''} pada \textit{keyboard} sehingga perlu ditambahkan sebuah fungsi yang menjalankan fungsi yang sama ketika pengguna menekan tombol tersebut. Perubahan dapat dilihat pada \textit{file} \textit{/assets/js/shj\_submissions.js}.

\end{itemize}

Semua perubahan kode dapat dilihat pada listing \ref{lst_2.1.1}.

\subsection{Implementasi Kriteria Sukses 2.1.2 \textit{No \textit{Keyboard} Trap}}
\label{subsec:implementasi_A_2.1.2}

Kriteria 2.1.2 dapat dipenuhi dengan cara menambahkan mekanisme untuk mengaktifkan dan mematikan fitur \textit{indent} yang menggunakan tombol \textit{''tab''} pada setiap halaman yang memakai fitur tersebut sehingga pengguna dapat melanjutkan navigasi pada aplikasi menggunakan tombol \textit{''tab''}. Mekanisme baru ini dapat diaktifkan/dimatikan dengan cara menekan tombol \textit{''esc''}. Perubahan kode dapat dilihat pada listing \ref{lst_2.1.2}.

\subsection{Implementasi Kriteria Sukses 2.4.1 \textit{Bypass Blocks}}
\label{subsec:implementasi_A_2.4.1}

Kriteria 2.4.1 dapat dipenuhi dengan cara menambahkan tautan di awal setiap halaman untuk meloncati menu navigasi. Tautan hanya dapat dijalankan jika pengguna mengoperasikannya dengan \textit{keyboard}. Perubahan dapat dilihat pada \textit{file} \textit{/application/views/templates/base.twig}. Selain itu ada juga penambahan untuk \textit{style} baru yang dapat dilihat pada \textit{file} \textit{/assets/styles/main.css}.

Tidak semua \textit{browser} dapat menangani \textit{in-page links}. Secara visual fokus \textit{keyboard} sudah berada pada lokasi target, tetapi sebenarnya fokus \textit{keyboard} belum diset ke lokasi target. Masalah ini dapat diselesaikan dengan menggunakan \textit{Javascript} untuk set fokus \textit{keyboard} ke lokasi target. Perubahan ada pada \textit{file} \textit{/assets/js/shj\_functions.js}.

Semua perubahan kode dapat dilihat pada listing \ref{lst_2.4.1}.

\subsection{Implementasi Kriteria Sukses 2.4.4 \textit{Link Purpose (In Context)}}
\label{subsec:implementasi_A_2.4.4}

Berikut adalah perubahan yang perlu dilakukan untuk memenuhi kriteria sukses 2.4.4:

\begin{itemize}
	\item Seluruh link pada \textit{sidebar} perlu diberikan label yang menjelaskan link tersebut. Perubahan yang terjadi ada pada \textit{file} \textit{/application/views/templates/side\_bar.twig}.

	\item Tautan pada gambar \textit{Profile} yang ada di menu atas perlu diberi label untuk menjelaskan tujuan dari tautan tersebut. Perubahan yang terjadi ada pada \textit{/application/views/templates/top\_bar.twig}.

	\item Tautan pada gambar \textit{PDF} perlu diberi label untuk menjelaskan tujuannya. Perubahan untuk tautan pada gambar \textit{PDF} yang ada di halaman \textit{Assignments} sudah dilakukan pada listing \ref{lst_1.1.1}
	
	\item Tautan yang ada pada tabel \textit{Problems} pada halaman \textit{Add Assignment} tidak memiliki label yang menjelaskan tujuannya, untuk itu perlu diberi label yang menjelaskan tujuannya. Perubahan dapat dilihat pada \textit{file} \textit{/application/views/pages/admin/add\_assignment.twig}.

\end{itemize}

Semua perubahan kode dapat dilihat pada listing \ref{lst_2.4.4}.

\subsection{Implementasi Kriteria Sukses 3.1.1 \textit{Language of Page}}
\label{subsec:implementasi_A_3.1.1}

Seluruh halaman aplikasi \textit{SharIF Judge} menggunakan Bahasa Inggris sehingga perlu diberikan atribut \textit{lang} yang bernilai \textit{''en''} pada awal \textit{tag html} untuk menunjukkan bahwa bahasa yang dipakai di halaman tersebut adalah Bahasa Inggris. Setiap kali halaman dibuka, aplikasi \textit{SharIF Judge} akan menjalankan \textit{file} \textit{base.twig} sebagai dasarnya sehingga perubahan hanya perlu dilakukan pada \textit{file} tersebut. Perubahan dapat dilihat pada \textit{/application/views/templates/base.twig}, potongan kode dapat dilihat pada listing \ref{lst_3.1.1}.

\subsection{Implementasi Kriteria Sukses 3.3.2 \textit{Labels or Instructions}}
\label{subsec:implementasi_A_3.3.2}

Berikut adalah perubahan yang perlu dilakukan untuk memenuhi kriteria sukses 3.3.2:

\begin{itemize}
	\item Semua elemen bagian tabel \textit{Problems} pada halaman \textit{Add Assignment} perlu diberi label untuk menjelaskan tujuan dari elemen tersebut. Perubahan sudah dilakukan pada listing \ref{lst_1.3.1}.
	
	\item \textit{checkbox} dan \textit{textarea} pada halaman \textit{Add Users} perlu diberi label yang menjelaskan tujuan dari elemen tersebut. Perubahan sudah dilakukan pada listing \ref{lst_1.3.1}.
	
	\item \textit{textarea} pada halaman \textit{Edit Problem Markdown} perlu diberi label yang menjelaskan tujuan dari elemen tersebut. Perubahan sudah dilakukan pada listing \ref{lst_1.3.1}.
	
	\item \textit{textarea} pada halaman \textit{Edit Problem Plain HTML} perlu diberi label yang menjelaskan tujuan dari elemen tersebut. Perubahan sudah dilakukan pada listing \ref{lst_1.3.1}.
	
	\item \textit{Dropdown} pada halaman \textit{Problems} perlu diberi label yang menjelaskan tujuan dari elemen tersebut. Perubahan sudah dilakukan pada listing \ref{lst_1.3.1}.
	
	\item Masukan \textit{Upload File} pada halaman \textit{Problems} perlu diberi label yang menjelaskan tujuan dari elemen tersebut. Perubahan sudah dilakukan pada listing \ref{lst_1.3.1}.
\end{itemize}

\subsection{Implementasi Kriteria Sukses 4.1.1 \textit{Parsing}}
\label{subsec:implementasi_A_4.1.1}

\textit{Form} masukan \textit{Extra Time} pada halaman \textit{Add Assignment} memiliki atribut id yang duplikat. Atribut yang dibuang yaitu atribut \textit{id=''extra\_time''} karena sudah dipakai pada elemen \textit{extra time} pada menu bagian atas. Perubahan dapat dilihat pada \textit{file} \textit{/application/views/pages/admin/add\_assignment.twig}, potongan kode dapat dilihat pada listing \ref{lst_4.1.1}.

\section{Pengujian}
\label{sec:pengujian}
Pada bagian ini ditulis skenario pengujian dan hasil yang didapatkan dari setiap skenario pengujian. Tujuan dari pengujian ini untuk melihat perbaikan yang dilakukan pada subbab \ref{sec:implementasi} berhasil atau tidak. Pengujian dilakukan dengan menggunakan perangkat komputer berupa laptop dengan sistem operasi Ubuntu, \textit{browser} \textit{Google Chrome}, dan \textit{screen reader} \textit{ChromeVox} sebagai teknologi alat bantu. Pengujian dilakukan pada server lokal milik penguji dan akun yang digunakan oleh penguji memiliki hak akses tak terbatas sehingga dapat menggunakan semua fitur yang terdapat pada halaman web \textit{SharIF Judge}. Selain itu pengujian juga dilakukan dengan kondisi seakan-akan memiliki keterbatasan visual.

\subsection{Skenario dan Hasil Pengujian}
\label{subsec:skenario_pengujian}
Pada bagian ini akan ditulis skenario dan hasil pengujian yang dilakukan untuk menguji perbaikan yang telah dilakukan. Setiap skenario pengujian ditulis dalam bentuk poin-poin yang menjelaskan cara untuk menggunakan fitur-fitur yang ada pada aplikasi \textit{SharIF Judge} dan hasil pengujian dituliskan dalam bentuk tabel yang berisi langkah skenario pengujian, hasil pengujian, dan aksi yang dilakukan dalam pengujian.

\subsubsection{\textit{Login}}
\label{subsubsec:skenario_login}
Berikut adalah langkah-langkah yang perlu dilakukan untuk \textit{Login}:

\begin{enumerate}
	\item Memasukkan alamat \url{http://sharif-judge/login} \footnote{url yang dipakai dalam sistem milik penguji} pada \textit{address bar browser}.
	\item Mengisi \textit{Username} pada bidang masukan yang sudah disediakan.
	\item Mengisi \textit{Password} pada bidang masukan yang sudah disediakan.
	\item Menekan tombol \textit{Login}.
\end{enumerate}

Hasil pengujian \textit{Login} dapat dilihat pada tabel \ref{tab:hasil_login}.

\begin{table}[H]
	\centering
	\caption{Hasil pengujian \textit{Login}}
	\label{tab:hasil_login}
	\begin{tabular}{|c|c|p{12cm}|}
		\toprule
		Langkah & Hasil & Aksi\\
		\midrule
		1 & Sukses & Menekan tombol \textit{Tab} sampai \textit{ChromeVox} membacakan \textit{''address and search bar''} kemudian pengguna memasukkan alamat \url{http://sharif-judge/login}.\\
		2 & Sukses & Menekan tombol \textit{Tab} sampai \textit{ChromeVox} membacakan bidang masukan \textit{Username}. Setelah itu pengguna memasukkan \textit{Username}.\\
		3 & Sukses & Menekan tombol \textit{Tab} sampai \textit{ChromeVox} membacakan bidang masukan \textit{Password}. Setelah itu pengguna memasukkan \textit{Password}. \\
		4 & Sukses & Menekan tombol \textit{Tab} sampai \textit{ChromeVox} membacakan tombol \textit{Login}. Setelah itu pengguna menekan tombol \textit{Enter} pada \textit{keyboard}.\\
		\bottomrule
	\end{tabular}
\end{table}

\subsubsection{\textit{Settings}}
\label{subsubsec:skenario_settings}
Berikut adalah langkah-langkah yang perlu dilakukan untuk sunting \textit{Settings}:

\begin{enumerate}
	\item \textit{Login} pada aplikasi \textit{SharIF Judge}.
	\item Memilih \textit{Settings} pada menu \textit{sidebar}.
	\item Mengisi bidang masukan \textit{Timezone} dengan nilai \textit{''Asia/Jakarta''}.
	\item Mengubah nilai \textit{Week Start Day} menjadi \textit{''Sunday''}.
	\item Mengubah nilai \textit{Registration} menjadi \textit{''checked''}.
	\item Mengubah isi \textit{Default Coefficient Rule} sesuai aturan yang sudah disediakan.
	\item Menekan tombol \textit{Save Changes} untuk meyimpan perubahan.
\end{enumerate}

Hasil pengujian \textit{Settings} dapat dilihat pada tabel \ref{tab:hasil_settings}.

\begin{table}[H]
	\centering
	\caption{Hasil pengujian \textit{Settings}}
	\label{tab:hasil_settings}
	\begin{tabular}{|c|c|p{12cm}|}
		\toprule
		Langkah & Hasil & Aksi\\
		\midrule
		1 & Sukses & Langkah \textit{Login} dapat dilihat pada subbab \ref{subsubsec:skenario_login}.\\
		2 & Sukses & Menekan tombol \textit{Tab} untuk navigasi ke \textit{sidebar} sampai \textit{ChromeVox} membacakan tautan untuk menu \textit{Settings}. Setelah itu pengguna menekan tombol \textit{Enter} pada \textit{keyboard}.\\
		3 & Sukses & Menekan tombol \textit{Tab} sampai \textit{ChromeVox} membacakan bidang masukan \textit{Timezone}. Setelah itu pengguna mengisi dengan nilai \textit{''Asia/Jakarta''}.\\
		4 & Sukses & Menekan tombol \textit{Tab} sampai \textit{ChromeVox} membacakan bidang masukan \textit{Week Start Day}. Setelah itu pengguna mengubah isi \textit{combobox} dengan menekan tombol panah atas atau bawah pada \textit{keyboard} sampai ChromeVox membacakan \textit{''Sunday''}.\\
		5 & Sukses & Menekan tombol \textit{Tab} sampai \textit{ChromeVox} membacakan bidang masukan \textit{Default Coefficient Rule}. Setelah itu pengguna mengisi sesuai spesifikasi yang sudah ditentukan. Pengguna dapat melanjutkan navigasi dengan menekan tombol \textit{Tab} setelah mematikan fitur \textit{tab indent} pada \textit{textarea} dengan menekan tombol \textit{Esc} pada \textit{keyboard}.\\
		6 & Sukses & Menekan tombol \textit{Tab} sampai \textit{ChromeVox} membacakan tombol \textit{Save Changes} kemudian menekan tombol \textit{Enter} pada \textit{keyboard}\\
		\bottomrule
	\end{tabular}
\end{table}

\subsubsection{\textit{Add Users}}
\label{subsubsec:skenario_add_users}
Berikut adalah langkah-langkah yang perlu dilakukan untuk menambah \textit{User}:

\begin{enumerate}
	\item \textit{Login} pada aplikasi \textit{SharIF Judge}.
	\item Memilih \textit{Users} pada menu \textit{sidebar}.
	\item Menekan tombol \textit{Add Users}.
	\item Mengubah nilai \textit{Send mail} menjadi \textit{''checked''}.
	\item Mengubah interval untuk mengirim pesan menjadi ''2''.
	\item Mengisi perintah untuk membuat \textit{User} pada bidang masukan yang sudah disediakan.
	\item Menekan tombol \textit{Add Users} untuk menambah \textit{User}.
\end{enumerate}

Hasil pengujian \textit{Add Users} dapat dilihat pada tabel \ref{tab:hasil_add_users}.

\begin{table}[H]
	\centering
	\caption{Hasil pengujian \textit{Add Users}}
	\label{tab:hasil_add_users}
	\begin{tabular}{|c|c|p{12cm}|}
		\toprule
		Langkah & Hasil & Aksi\\
		\midrule
		1 & Sukses & Langkah \textit{Login} dapat dilihat pada subbab \ref{subsubsec:skenario_login}.\\
		2 & Sukses & Menekan tombol \textit{Tab} untuk navigasi ke \textit{sidebar} sampai \textit{ChromeVox} membacakan tautan untuk menu \textit{Users}. Setelah itu pengguna menekan tombol \textit{Enter} pada \textit{keyboard}.\\
		3 & Sukses & Menekan tombol \textit{Tab} sampai \textit{ChromeVox} membacakan tautan \textit{Add Users} kemudian pengguna menekan tombol \textit{Enter} pada \textit{keyboard}.\\
		4 & Sukses & Menekan tombol \textit{Tab} sampai \textit{ChromeVox} membacakan \textit{checkbox} untuk \textit{Send mail}. Kemudian pengguna menekan \textit{space bar} pada \textit{keyboard} untuk mengubah nilainya menjadi \textit{''checked''}.\\
		5 & Sukses & Menekan tombol \textit{Tab} sampai \textit{ChromeVox} membacakan \textit{''Delay''}. Setelah itu pengguna mengisi dengan nilai ''2''.\\
		6 & Sukses & Menekan tombol \textit{Tab} sampai \textit{ChromeVox} membacakan \textit{''Command for creating new user''}. Setelah itu pengguna mengisi sesuai spesifikasi yang sudah ditentukan. Pengguna dapat melanjutkan navigasi dengan menekan tombol \textit{Tab} setelah mematikan fitur \textit{tab indent} pada \textit{textarea} dengan menekan tombol \textit{Esc} pada \textit{keyboard}.\\
		7 & Sukses & Menekan tombol \textit{Tab} sampai \textit{ChromeVox} membacakan tombol \textit{Add Users} kemudian menekan tombol \textit{Enter} pada \textit{keyboard}.\\
		\bottomrule
	\end{tabular}
\end{table}

\subsubsection{\textit{Delete User}}
\label{subsubsec:skenario_delete_user}
Berikut adalah langkah-langkah yang perlu dilakukan untuk menghapus \textit{User}:

\begin{enumerate}
	\item \textit{Login} pada aplikasi \textit{SharIF Judge}.
	\item Memilih \textit{Users} pada menu \textit{sidebar}.
	\item Mencari nama \textit{User} yang akan dihapus pada tabel \textit{Users}.
	\item Menekan tombol \textit{Delete User} pada kolom \textit{action} di baris yang sesuai.
	\item Menekan tombol \textit{Yes, Delete} pada \textit{popup} yang muncul.
\end{enumerate}

Hasil pengujian \textit{Delete User} dapat dilihat pada tabel \ref{tab:hasil_delete_user}.

\begin{table}[H]
	\centering
	\caption{Hasil pengujian \textit{Delete User}}
	\label{tab:hasil_delete_user}
	\begin{tabular}{|c|c|p{12cm}|}
		\toprule
		Langkah & Hasil & Aksi\\
		\midrule
		1 & Sukses & Langkah \textit{Login} dapat dilihat pada subbab \ref{subsubsec:skenario_login}.\\
		2 & Sukses & Menekan tombol \textit{Tab} untuk navigasi ke \textit{sidebar} sampai \textit{ChromeVox} membacakan tautan untuk menu \textit{Users}. Setelah itu pengguna menekan tombol \textit{Enter} pada \textit{keyboard}.\\
		3 & Sukses & Menavigasikan \textit{ChromeVox} sampai membacakan baris \textit{User} yang akan dihapus pada tabel.\\
		4 & Sukses & Menekan tombol navigasi maju \textit{ChromeVox} sampai membacakan \textit{Delete User} kemudian pengguna menekan tombol \textit{Enter} pada \textit{keyboard}.\\
		5 & Sukses & Menekan tombol navigasi maju \textit{ChromeVox} sampai membacakan tombol \textit{Yes, Delete} kemudian pengguna menekan tombol \textit{Enter} pada \textit{keyboard}.\\
		\bottomrule
	\end{tabular}
\end{table}

\subsubsection{\textit{Delete User Submissions}}
\label{subsubsec:skenario_delete_user_submissions}
Berikut adalah langkah-langkah yang perlu dilakukan untuk menghapus \textit{User submissions}:

\begin{enumerate}
	\item \textit{Login} pada aplikasi \textit{SharIF Judge}.
	\item Memilih \textit{Users} pada menu \textit{sidebar}.
	\item Mencari nama \textit{User} yang akan dihapus \textit{Submissions} pada tabel \textit{Users}.
	\item Menekan tombol \textit{Delete Submissions} pada kolom \textit{action} di baris yang sesuai.
	\item Menekan tombol \textit{Yes, Delete} pada \textit{popup} yang muncul.
\end{enumerate}

Hasil pengujian \textit{Delete User Submissions} dapat dilihat pada tabel \ref{tab:hasil_delete_user_submissions}.

\begin{table}[H]
	\centering
	\caption{Hasil pengujian \textit{Delete User Submissions}}
	\label{tab:hasil_delete_user_submissions}
	\begin{tabular}{|c|c|p{12cm}|}
		\toprule
		Langkah & Hasil & Aksi\\
		\midrule
		1 & Sukses & Langkah \textit{Login} dapat dilihat pada subbab \ref{subsubsec:skenario_login}.\\
		2 & Sukses & Menekan tombol \textit{Tab} untuk navigasi ke \textit{sidebar} sampai \textit{ChromeVox} membacakan tautan untuk menu \textit{Users}. Setelah itu pengguna menekan tombol \textit{Enter} pada \textit{keyboard}.\\
		3 & Sukses & Menavigasikan \textit{ChromeVox} sampai membacakan baris \textit{User} yang akan dihapus \textit{Submissions} pada tabel.\\
		4 & Sukses & Menekan tombol navigasi maju \textit{ChromeVox} sampai membacakan \textit{Delete User Submissions} kemudian pengguna menekan tombol \textit{Enter} pada \textit{keyboard}.\\
		5 & Sukses & Menekan tombol navigasi maju \textit{ChromeVox} sampai membacakan tombol \textit{Yes, Delete} kemudian pengguna menekan tombol \textit{Enter} pada \textit{keyboard}.\\
		\bottomrule
	\end{tabular}
\end{table}

\subsubsection{\textit{Edit User}}
\label{subsubsec:skenario_edit_user}
Berikut adalah langkah-langkah yang perlu dilakukan untuk sunting \textit{User}:

\begin{enumerate}
	\item \textit{Login} pada aplikasi \textit{SharIF Judge}.
	\item Memilih \textit{Users} pada menu \textit{sidebar}.
	\item Mencari nama \textit{user} yang di \textit{edit} pada tabel \textit{Users}.
	\item Menekan tombol \textit{Edit} pada kolom \textit{action} di baris yang sesuai.
	\item Mengisi isi bidang masukan \textit{Name} menjadi \textit{''student''}.
	\item Mengubah isi \textit{User Role} menjadi \textit{''student''}. 
	\item Menekan tombol \textit{Save} untuk menyimpan perubahan.
\end{enumerate}	

Hasil pengujian \textit{Edit User} dapat dilihat pada tabel \ref{tab:hasil_edit_user}.

\begin{table}[H]
	\centering
	\caption{Hasil pengujian \textit{Edit User}}
	\label{tab:hasil_edit_user}
	\begin{tabular}{|c|c|p{12cm}|}
		\toprule
		Langkah & Hasil & Aksi\\
		\midrule
		1 & Sukses & Langkah \textit{Login} dapat dilihat pada subbab \ref{subsubsec:skenario_login}.\\
		2 & Sukses & Menekan tombol \textit{Tab} untuk navigasi ke \textit{sidebar} sampai \textit{ChromeVox} membacakan tautan untuk menu \textit{Users}. Setelah itu pengguna menekan tombol \textit{Enter} pada \textit{keyboard}.\\
		3 & Sukses & Menavigasikan \textit{ChromeVox} sampai membacakan baris \textit{User} yang akan dihapus \textit{Submissions} pada tabel.\\
		4 & Sukses & Menekan tombol navigasi maju \textit{ChromeVox} sampai membacakan \textit{Edit} kemudian pengguna menekan tombol \textit{Enter} pada \textit{keyboard}.\\
		5 & Sukses & Menekan tombol \textit{Tab} sampai \textit{ChromeVox} membacakan bidang masukan \textit{Name}. Setelah itu pengguna mengisi dengan nilai \textit{''student''}.\\
		6 & Sukses & Menekan tombol \textit{Tab} sampai \textit{ChromeVox} membacakan bidang masukan \textit{User Role}. Setelah itu pengguna mengubah isi \textit{combobox} dengan menekan tombol panah atas atau bawah pada \textit{keyboard} sampai \textit{ChromeVox} membacakan \textit{''student''}.\\
		7 & Sukses & Menekan tombol \textit{Tab} sampai \textit{ChromeVox} membacakan tombol \textit{Save} kemudian pengguna menekan tombol \textit{Enter} pada \textit{keyboard}.\\
		\bottomrule
	\end{tabular}
\end{table}

\subsubsection{\textit{Add Notification}}
\label{subsubsec:skenario_add_notification}
Berikut adalah langkah-langkah yang perlu dilakukan untuk menambah notifikasi:

\begin{enumerate}
	\item \textit{Login} pada aplikasi \textit{SharIF Judge}.
	\item Memilih \textit{Notifications} pada menu \textit{sidebar}.
	\item Menekan tautan \textit{New}.
	\item Mengisi judul pada bidang masukan \textit{Title}.
	\item Mengisi notifikasi pada bidang masukan \textit{Text}.
	\item Menekan tombol \textit{Add} untuk menambah notifikasi.
\end{enumerate}

Hasil pengujian \textit{Add Notification} dapat dilihat pada tabel \ref{tab:hasil_add_notification}.

\begin{table}[H]
	\centering
	\caption{Hasil pengujian \textit{Add Notification}}
	\label{tab:hasil_add_notification}
	\begin{tabular}{|c|c|p{12cm}|}
		\toprule
		Langkah & Hasil & Aksi\\
		\midrule
		1 & Sukses & Langkah \textit{Login} dapat dilihat pada subbab \ref{subsubsec:skenario_login}.\\
		2 & Sukses & Menekan tombol \textit{Tab} untuk navigasi ke \textit{sidebar} sampai \textit{ChromeVox} membacakan tautan untuk menu \textit{Notifications}. Setelah itu pengguna menekan tombol \textit{Enter} pada \textit{keyboard}.\\
		3 & Sukses & Menekan tombol \textit{Tab} sampai \textit{ChromeVox} membacakan tautan \textit{New} kemudian pengguna menekan tombol \textit{Enter} pada \textit{keyboard}.\\
		4 & Sukses & Menekan tombol \textit{Tab} sampai \textit{ChromeVox} membacakan bidang masukan \textit{Title}. Setelah itu pengguna mengisi dengan nilai \textit{''Judul''}.\\
		5 & Sukses & Menekan tombol \textit{Tab} sampai \textit{ChromeVox} membacakan bidang masukan \textit{Text}. Setelah itu pengguna mengisi dengan nilai \textit{''Isi Notifikasi''}.\\
		6 & Sukses & Menekan tombol \textit{Tab} sampai \textit{ChromeVox} membacakan tombol \textit{Add} kemudian pengguna menekan tombol \textit{Enter} pada \textit{keyboard}.\\
		\bottomrule
	\end{tabular}
\end{table}

\subsubsection{\textit{Add Assignment}}
\label{subsubsec:skenario_add_assignment}
Berikut adalah langkah-langkah yang perlu dilakukan untuk menambah \textit{Assignment}:

\begin{enumerate}
	\item \textit{Login} pada aplikasi \textit{SharIF Judge}.
	\item Memilih \textit{Assignments} pada menu \textit{sidebar}.
	\item Menekan tautan \textit{Add}.
	\item Mengisi bidang masukan \textit{Assignment Name} dengan nilai \textit{''Assignment 1''}.
	\item Mengisi bidang masukan \textit{Participants} dengan nilai \textit{''ALL''}.
	\item Mengunggah \textit{PDF File} dengan \textit{file PDF} yang sesuai.
	\item Mengubah nilai \textit{Scoreboard} menjadi \textit{''checked''}.
	\item Menekan tombol \textit{Add Assignment} untuk menambah \textit{Assignment}.
\end{enumerate}

Hasil pengujian \textit{Add Assignment} dapat dilihat pada tabel \ref{tab:hasil_add_assignment}.

\begin{table}[H]
	\centering
	\caption{Hasil pengujian \textit{Add Assignment}}
	\label{tab:hasil_add_assignment}
	\begin{tabular}{|c|c|p{12cm}|}
		\toprule
		Langkah & Hasil & Aksi\\
		\midrule
		1 & Sukses & Langkah \textit{Login} dapat dilihat pada subbab \ref{subsubsec:skenario_login}.\\
		2 & Sukses & Menekan tombol \textit{Tab} untuk navigasi ke \textit{sidebar} sampai \textit{ChromeVox} membacakan tautan untuk menu \textit{Assignments}. Setelah itu pengguna menekan tombol \textit{Enter} pada \textit{keyboard}.\\
		3 & Sukses & Menekan tombol \textit{Tab} sampai \textit{ChromeVox} membacakan tautan \textit{Add} kemudian pengguna menekan tombol \textit{Enter} pada \textit{keyboard}.\\
		4 & Sukses & Menekan tombol \textit{Tab} sampai \textit{ChromeVox} membacakan bidang masukan \textit{Assignment Name}. Setelah itu pengguna mengisi dengan nilai \textit{''Assignment 1''}.\\
		5 & Sukses & Menekan tombol \textit{Tab} sampai \textit{ChromeVox} membacakan bidang masukan \textit{Participants}. Setelah itu pengguna mengisi dengan nilai \textit{''All''}. Pengguna dapat melanjutkan navigasi dengan menekan tombol \textit{Tab} setelah mematikan fitur \textit{tab indent} pada \textit{textarea} dengan menekan tombol \textit{Esc} pada \textit{keyboard}.\\
		6 & Sukses & Menekan tombol \textit{Tab} sampai \textit{ChromeVox} membacakan \textit{PDF File} kemudian pengguna menekan tombol \textit{Enter} pada \textit{keyboard}. Setelah itu pengguna memilih \textit{file PDF} yang akan diunggah.\\
		7 & Sukses & Menekan tombol \textit{Tab} sampai \textit{ChromeVox} membacakan \textit{checkbox} untuk \textit{Scoreboard}. Kemudian pengguna menekan \textit{space bar} pada \textit{keyboard} untuk mengubah nilainya menjadi \textit{''checked''}.\\
		8 & Sukses & Menekan tombol \textit{Tab} sampai \textit{ChromeVox} membacakan tombol \textit{Add Assignment} kemudian pengguna menekan tombol \textit{Enter} pada \textit{keyboard}.\\
		\bottomrule
	\end{tabular}
\end{table}

\subsubsection{\textit{Delete Assignment}}
\label{subsubsec:skenario_delete_assignment}
Berikut adalah langkah-langkah yang perlu dilakukan untuk menghapus \textit{Assignment}:

\begin{enumerate}
	\item \textit{Login} pada aplikasi \textit{SharIF Judge}.
	\item Memilih \textit{Assignments} pada menu \textit{sidebar}.
	\item Mencari nama \textit{Assignment} yang akan di hapus pada tabel \textit{Assignments}.
	\item Menekan tombol \textit{Delete} pada kolom \textit{action} di baris yang sesuai.
	\item Menekan tombol \textit{Delete this assignment} untuk menghapus \textit{Assignment}.
\end{enumerate}

Hasil pengujian \textit{Delete Assignment} dapat dilihat pada tabel \ref{tab:hasil_delete_assignment}.

\begin{table}[H]
	\centering
	\caption{Hasil pengujian \textit{Delete Assignment}}
	\label{tab:hasil_delete_assignment}
	\begin{tabular}{|c|c|p{12cm}|}
		\toprule
		Langkah & Hasil & Aksi\\
		\midrule
		1 & Sukses & Langkah \textit{Login} dapat dilihat pada subbab \ref{subsubsec:skenario_login}.\\
		2 & Sukses & Menekan tombol \textit{Tab} untuk navigasi ke \textit{sidebar} sampai \textit{ChromeVox} membacakan tautan untuk menu \textit{Assignments}. Setelah itu pengguna menekan tombol \textit{Enter} pada \textit{keyboard}.\\
		3 & Sukses & Menavigasikan \textit{ChromeVox} sampai membacakan baris \textit{Assignment} yang akan dihapus pada tabel.\\
		4 & Sukses & Menekan tombol navigasi maju \textit{ChromeVox} sampai membacakan \textit{Delete} kemudian pengguna menekan tombol \textit{Enter} pada \textit{keyboard}.\\
		5 & Sukses & Menekan tombol \textit{Tab} sampai \textit{ChromeVox} membacakan tombol \textit{Delete this assignment} kemudian pengguna menekan tombol \textit{Enter} pada \textit{keyboard}.\\
		\bottomrule
	\end{tabular}
\end{table}

\subsubsection{\textit{Edit Assignment}}
\label{subsubsec:skenario_edit_assignment}
Berikut adalah langkah-langkah yang perlu dilakukan untuk sunting \textit{Assignment}:

\begin{enumerate}
	\item \textit{Login} pada aplikasi \textit{SharIF Judge}.
	\item Memilih \textit{Assignments} pada menu \textit{sidebar}.
	\item Mencari nama \textit{Assignment} yang akan disunting pada tabel \textit{Assignments}.
	\item Menekan tombol \textit{Edit} pada kolom \textit{action} di baris yang sesuai.
	\item Mengisi bidang masukan \textit{Extra Time} dengan nilai \textit{''0*60''}.
	\item Mengisi bidang masukan \textit{Participants} dengan nilai \textit{''ALL''}.
	\item Mengunggah \textit{PDF File} dengan \textit{file PDF} yang sesuai.
	\item Mengubah nilai \textit{Open} menjadi \textit{''checked''}.
	\item Menekan tombol \textit{Edit Assignment} untuk menyimpan perubahan.
\end{enumerate}

Hasil pengujian \textit{Edit Assignment} dapat dilihat pada tabel \ref{tab:hasil_edit_assignment}.

\begin{table}[H]
	\centering
	\caption{Hasil pengujian \textit{Edit Assignment}}
	\label{tab:hasil_edit_assignment}
	\begin{tabular}{|c|c|p{12cm}|}
		\toprule
		Langkah & Hasil & Aksi\\
		\midrule
		1 & Sukses & Langkah \textit{Login} dapat dilihat pada subbab \ref{subsubsec:skenario_login}.\\
		2 & Sukses & Menekan tombol \textit{Tab} untuk navigasi ke \textit{sidebar} sampai \textit{ChromeVox} membacakan tautan untuk menu \textit{Assignments}. Setelah itu pengguna menekan tombol \textit{Enter} pada \textit{keyboard}.\\
		3 & Sukses & Menavigasikan \textit{ChromeVox} sampai membacakan baris \textit{Assignment} yang akan disunting pada tabel.\\
		4 & Sukses & Menekan tombol navigasi maju \textit{ChromeVox} sampai membacakan \textit{Edit} kemudian pengguna menekan tombol \textit{Enter} pada \textit{keyboard}.\\
		5 & Sukses & Menekan tombol \textit{Tab} sampai \textit{ChromeVox} membacakan bidang masukan \textit{Extra Time}. Setelah itu pengguna mengisi dengan nilai \textit{''0*60''}.\\
		6 & Sukses & Menekan tombol \textit{Tab} sampai \textit{ChromeVox} membacakan bidang masukan \textit{Participants}. Setelah itu pengguna mengisi dengan nilai \textit{''All''}. Pengguna dapat melanjutkan navigasi dengan menekan tombol \textit{Tab} setelah mematikan fitur \textit{tab indent} pada \textit{textarea} dengan menekan tombol \textit{Esc} pada \textit{keyboard}.\\
		7 & Sukses & Menekan tombol \textit{Tab} sampai \textit{ChromeVox} membacakan \textit{PDF File} kemudian pengguna menekan tombol \textit{Enter} pada \textit{keyboard}. Setelah itu pengguna memilih \textit{file PDF} yang akan diunggah.\\
		8 & Sukses & Menekan tombol \textit{Tab} sampai \textit{ChromeVox} membacakan \textit{checkbox} untuk \textit{Open}. Kemudian pengguna menekan \textit{space bar} pada \textit{keyboard} untuk mengubah nilainya menjadi \textit{''checked''}.\\
		9 & Sukses & Menekan tombol \textit{Tab} sampai \textit{ChromeVox} membacakan tombol \textit{Edit Assignment} kemudian pengguna menekan tombol \textit{Enter} pada \textit{keyboard}.\\
		\bottomrule
	\end{tabular}
\end{table}

\subsubsection{\textit{Edit Problem Description(Markdown)}}
\label{subsubsec:skenario_edit_problem_description_markdown}
Berikut adalah langkah-langkah yang perlu dilakukan untuk sunting deskripsi masalah dalam format \textit{markdown}:

\begin{enumerate}
	\item \textit{Login} pada aplikasi \textit{SharIF Judge}.
	\item Memilih \textit{Assignment} pada menu bagian atas.
	\item Memilih \textit{Problems} pada menu \textit{sidebar}.
	\item Memilih \textit{Problem} pada tabel \textit{Problems}.
	\item Menekan tautan \textit{Edit Markdown}.
	\item Mengisi deskripsi \textit{Problem} pada \textit{textarea}.
	\item Menekan tombol \textit{Save} untuk menyimpan perubahan.
\end{enumerate}

Hasil pengujian \textit{Edit Problem Description (Markdown)} dapat dilihat pada tabel \ref{tab:hasil_edit_problem_description_markdown}.

\begin{table}[H]
	\centering
	\caption{Hasil pengujian \textit{Edit Problem Description (Markdown)}}
	\label{tab:hasil_edit_problem_description_markdown}
	\begin{tabular}{|c|c|p{12cm}|}
		\toprule
		Langkah & Hasil & Aksi\\
		\midrule
		1 & Sukses & Langkah \textit{Login} dapat dilihat pada subbab \ref{subsubsec:skenario_login}.\\
		2 & Sukses & Menekan tombol \textit{Tab} untuk navigasi ke \textit{topbar} sampai \textit{ChromeVox} membacakan \textit{Assignment} yang akan disunting deskripsinya. Setelah itu pengguna menekan tombol \textit{Enter} pada \textit{keyboard}.\\
		3 & Sukses & Menekan tombol \textit{Tab} untuk navigasi ke \textit{sidebar} sampai \textit{ChromeVox} membacakan tautan untuk menu \textit{Problems}. Setelah itu pengguna menekan tombol \textit{Enter} pada \textit{keyboard}.\\
		4 & Sukses & Menekan tombol \textit{Tab} sampai \textit{ChromeVox} membacakan \textit{Problem} pada tabel yang akan disunting deskripsinya kemudian pengguna menekan tombol \textit{Enter} pada \textit{keyboard}.\\
		5 & Sukses & Menekan tombol \textit{Tab} sampai \textit{ChromeVox} membacakan tautan \textit{Edit Markdown} kemudian pengguna menekan tombol \textit{Enter} pada \textit{keyboard}.\\
		6 & Sukses & Menekan tombol \textit{Tab} sampai \textit{ChromeVox} membacakan bidang masukan \textit{Markdown Editor}. Setelah itu pengguna mengisi deskripsi sesuai dengan spesifikasinya. Pengguna dapat melanjutkan navigasi dengan menekan tombol \textit{Tab} setelah mematikan fitur \textit{tab indent} pada \textit{textarea} dengan menekan tombol \textit{Esc} pada \textit{keyboard}.\\
		7 & Sukses & Menekan tombol \textit{Tab} sampai \textit{ChromeVox} membacakan tombol \textit{Save} kemudian pengguna menekan tombol \textit{Enter} pada \textit{keyboard}.\\
		\bottomrule
	\end{tabular}
\end{table}

\subsubsection{\textit{Edit Problem Description(HTML)}}
\label{subsubsec:skenario_edit_problem_description_html}
Berikut adalah langkah-langkah yang perlu dilakukan untuk sunting deskripsi masalah dalam format \textit{HTML}:

\begin{enumerate}
	\item \textit{Login} pada aplikasi \textit{SharIF Judge}.
	\item Memilih \textit{Assignment} pada menu bagian atas.
	\item Memilih \textit{Problems} pada menu \textit{sidebar}.
	\item Memilih \textit{Problem} pada tabel \textit{Problems}.
	\item Menekan tautan \textit{Edit HTML}.
	\item Mengisi deskripsi \textit{Problem} pada \textit{text editor}.
	\item Menekan tombol \textit{Save} untuk menyimpan perubahan.
\end{enumerate}

Hasil pengujian \textit{Edit Problem Description (HTML)} dapat dilihat pada tabel \ref{tab:hasil_edit_problem_description_html}.

\begin{table}[H]
	\centering
	\caption{Hasil pengujian \textit{Edit Problem Description (HTML)}}
	\label{tab:hasil_edit_problem_description_html}
	\begin{tabular}{|c|c|p{12cm}|}
		\toprule
		Langkah & Hasil & Aksi\\
		\midrule
		1 & Sukses & Langkah \textit{Login} dapat dilihat pada subbab \ref{subsubsec:skenario_login}.\\
		2 & Sukses & Menekan tombol \textit{Tab} untuk navigasi ke \textit{topbar} sampai \textit{ChromeVox} membacakan \textit{Assignment} yang akan disunting deskripsinya. Setelah itu pengguna menekan tombol \textit{Enter} pada \textit{keyboard}.\\
		3 & Sukses & Menekan tombol \textit{Tab} untuk navigasi ke \textit{sidebar} sampai \textit{ChromeVox} membacakan tautan untuk menu \textit{Problems}. Setelah itu pengguna menekan tombol \textit{Enter} pada \textit{keyboard}.\\
		4 & Sukses & Menekan tombol \textit{Tab} sampai \textit{ChromeVox} membacakan \textit{Problem} pada tabel yang akan disunting deskripsinya kemudian pengguna menekan tombol \textit{Enter} pada \textit{keyboard}.\\
		5 & Sukses & Menekan tombol \textit{Tab} sampai \textit{ChromeVox} membacakan tautan \textit{Edit HTML} kemudian pengguna menekan tombol \textit{Enter} pada \textit{keyboard}.\\
		6 & Sukses & Menekan tombol \textit{Tab} sampai \textit{ChromeVox} membacakan bidang masukan \textit{Edit Text}.\\
		7 & Sukses & Menekan tombol \textit{Tab} sampai \textit{ChromeVox} membacakan tombol \textit{Save} kemudian pengguna menekan tombol \textit{Enter} pada \textit{keyboard}.\\
		\bottomrule
	\end{tabular}
\end{table}

\subsubsection{\textit{Edit Problem Description(HTML)}}
\label{subsubsec:skenario_edit_problem_description_plain_html}
Berikut adalah langkah-langkah yang perlu dilakukan untuk sunting deskripsi masalah dalam format \textit{plain HTML}:

\begin{enumerate}
	\item \textit{Login} pada aplikasi \textit{SharIF Judge}.
	\item Memilih \textit{Assignment} pada menu bagian atas.
	\item Memilih \textit{Problems} pada menu \textit{sidebar}.
	\item Memilih \textit{Problem} pada tabel \textit{Problems}.
	\item Menekan tautan \textit{Edit Plain HTML}.
	\item Mengisi deskripsi \textit{Problem} pada \textit{textarea}.
	\item Menekan tombol \textit{Save} untuk menyimpan perubahan.
\end{enumerate}

Hasil pengujian \textit{Edit Problem Description (Plain HTML)} dapat dilihat pada tabel \ref{tab:hasil_edit_problem_description_plain_html}.

\begin{table}[H]
	\centering
	\caption{Hasil pengujian \textit{Edit Problem Description (Plain HTML)}}
	\label{tab:hasil_edit_problem_description_plain_html}
	\begin{tabular}{|c|c|p{12cm}|}
		\toprule
		Langkah & Hasil & Aksi\\
		\midrule
		1 & Sukses & Langkah \textit{Login} dapat dilihat pada subbab \ref{subsubsec:skenario_login}.\\
		2 & Sukses & Menekan tombol \textit{Tab} untuk navigasi ke \textit{topbar} sampai \textit{ChromeVox} membacakan \textit{Assignment} yang akan disunting deskripsinya. Setelah itu pengguna menekan tombol \textit{Enter} pada \textit{keyboard}.\\
		3 & Sukses & Menekan tombol \textit{Tab} untuk navigasi ke \textit{sidebar} sampai \textit{ChromeVox} membacakan tautan untuk menu \textit{Problems}. Setelah itu pengguna menekan tombol \textit{Enter} pada \textit{keyboard}.\\
		4 & Sukses & Menekan tombol \textit{Tab} sampai \textit{ChromeVox} membacakan \textit{Problem} pada tabel yang akan disunting deskripsinya kemudian pengguna menekan tombol \textit{Enter} pada \textit{keyboard}.\\
		5 & Sukses & Menekan tombol \textit{Tab} sampai \textit{ChromeVox} membacakan tautan \textit{Edit Plain HTML} kemudian pengguna menekan tombol \textit{Enter} pada \textit{keyboard}.\\
		6 & Sukses & Menekan tombol \textit{Tab} sampai \textit{ChromeVox} membacakan bidang masukan \textit{HTML Editor}.\\
		7 & Sukses & Menekan tombol \textit{Tab} sampai \textit{ChromeVox} membacakan tombol \textit{Save} kemudian pengguna menekan tombol \textit{Enter} pada \textit{keyboard}.\\
		\bottomrule
	\end{tabular}
\end{table}

\subsubsection{\textit{Submit File}}
\label{subsubsec:skenario_submit}
Berikut adalah langkah-langkah yang perlu dilakukan untuk mengumpulkan \textit{file}:

\begin{enumerate}
	\item \textit{Login} pada aplikasi \textit{SharIF Judge}.
	\item Memilih \textit{Assignment} pada menu bagian atas.
	\item Memilih \textit{Submit} pada menu \textit{sidebar}.
	\item Memilih \textit{Problem} pada \textit{dropdown Problem}.
	\item Memilih bahasa pemrograman pada \textit{dropdown Language}.
	\item Memilih \textit{file} yang akan dikumpulkan.
	\item Menekan tombol \textit{Submit} untuk mengumpulkan \textit{file}.
\end{enumerate}

Hasil pengujian untuk mengumpulkan \textit{file} dapat dilihat pada tabel \ref{tab:hasil_submit}.

\begin{table}[H]
	\centering
	\caption{Hasil pengujian \textit{Submit}}
	\label{tab:hasil_submit}
	\begin{tabular}{|c|c|p{12cm}|}
		\toprule
		Langkah & Hasil & Aksi\\
		\midrule
		1 & Sukses & Langkah \textit{Login} dapat dilihat pada subbab \ref{subsubsec:skenario_login}.\\
		2 & Sukses & Menekan tombol \textit{Tab} untuk navigasi ke \textit{topbar} sampai \textit{ChromeVox} membacakan \textit{Assignment} yang sesuai. Setelah itu pengguna menekan tombol \textit{Enter} pada \textit{keyboard}.\\
		3 & Sukses & Menekan tombol \textit{Tab} untuk navigasi ke \textit{sidebar} sampai \textit{ChromeVox} membacakan tautan untuk menu \textit{Submit}. Setelah itu pengguna menekan tombol \textit{Enter} pada \textit{keyboard}.\\
		4 & Sukses & Menekan tombol \textit{Tab} sampai \textit{ChromeVox} membacakan bidang masukan \textit{Problem}. Setelah itu pengguna mengubah isi \textit{combobox} dengan menekan tombol panah atas atau bawah pada \textit{keyboard} sampai \textit{ChromeVox} membacakan \textit{''Problem 1''}.\\
		5 & Sukses & Menekan tombol \textit{Tab} sampai \textit{ChromeVox} membacakan bidang masukan \textit{Language}. Setelah itu pengguna mengubah isi \textit{combobox} dengan menekan tombol panah atas atau bawah pada \textit{keyboard} sampai \textit{ChromeVox} membacakan \textit{''Java''}.\\
		6 & Sukses & Menekan tombol \textit{Tab} sampai \textit{ChromeVox} membacakan bidang masukan \textit{File}. Setelah itu pengguna menekan tombol \textit{Enter} pada \textit{keyboard}. Pengguna harus memilih \textit{file} yang akan diunggah dengan benar.\\
		7 & Sukses & Menekan tombol \textit{Tab} sampai \textit{ChromeVox} membacakan tombol \textit{Submit} kemudian pengguna menekan tombol \textit{Enter} pada \textit{keyboard}.\\
		\bottomrule
	\end{tabular}
\end{table}

\subsubsection{\textit{Final Submission}}
\label{subsubsec:skenario_final_submission}
Berikut adalah langkah-langkah yang perlu dilakukan untuk memilih \textit{Final Submission}:

\begin{enumerate}
	\item \textit{Login} pada aplikasi \textit{SharIF Judge}.
	\item Memilih \textit{Assignment} pada menu bagian atas.
	\item Memilih \textit{All Submission} pada menu \textit{sidebar}.
	\item Memilih \textit{Submission} yang akan dijadikan \textit{Final Submission} pada tabel \textit{All Submissions}
	\item Menekan tombol pada kolom \textit{Final} untuk menjadikan \textit{Submission} menjadi \textit{Final Submission}.
\end{enumerate}

Hasil pengujian untuk memilih \textit{Final Submission} dapat dilihat pada tabel \ref{tab:hasil_final_submission}.

\begin{table}[H]
	\centering
	\caption{Hasil pengujian \textit{Final Submission}}
	\label{tab:hasil_final_submission}
	\begin{tabular}{|c|c|p{12cm}|}
		\toprule
		Langkah & Hasil & Aksi\\
		\midrule
		1 & Sukses & Langkah \textit{Login} dapat dilihat pada subbab \ref{subsubsec:skenario_login}.\\
		2 & Sukses & Menekan tombol \textit{Tab} untuk navigasi ke \textit{topbar} sampai \textit{ChromeVox} membacakan \textit{Assignment} yang sesuai. Setelah itu pengguna menekan tombol \textit{Enter} pada \textit{keyboard}.\\
		3 & Sukses & Menekan tombol \textit{Tab} untuk navigasi ke \textit{sidebar} sampai \textit{ChromeVox} membacakan tautan untuk menu \textit{All Submissions}. Setelah itu pengguna menekan tombol \textit{Enter} pada \textit{keyboard}.\\
		4 & Sukses & Menavigasikan \textit{ChromeVox} sampai membacakan baris \textit{Submission} yang akan menjadi \textit{Final Submission} pada tabel.\\
		5 & Sukses & Menavigasikan \textit{ChromeVox} sampai membacakan tombol \textit{Set Final} pada \textit{Submission} yang akan menjadi \textit{Final Submission} pada tabel. Kemudian pengguna menekan tombol \textit{Spacebar} pada \textit{keyboard}.\\
		\bottomrule
	\end{tabular}
\end{table}

\subsection{Kesimpulan}
\label{subsec:kesimpulan_pengujian}

Semua perubahan yang dilakukan pada aplikasi \textit{SharIF Judge} telah meningkatkan tingkat kepatuhannya sampai tingkat \textit{A}. Perubahan ini memungkinkan pengguna untuk memakai aplikasi \textit{SharIF Judge} menggunakan \textit{keyboard} dalam kondisi tidak dapat melihat dengan bantuan \textit{screen reader ChromeVox}. Dengan perubahan yang dilakukan pada aplikasi \textit{SharIF Judge} diharapkan pengguna yang memiliki disabilitas keterbatasan penglihatan dapat memakai aplikasi \textit{SharIF Judge} tanpa menghadapi kesulitan.