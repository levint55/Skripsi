%versi 2 (8-10-2016)
\chapter{Landasan Teori}
\label{chap:teori}


\section{WCAG 2.1}
\label{sec:WCAG2.1} 
WCAG 2.1 merupakan pembaruan dari WCAG 2.0 yang dibuat pada desember 2008. WCAG memuat rekomendasi untuk membuat konten \textit{web} lebih mudah diakses. Pedoman - pedoman ini akan membuat konten lebih mudah diakses untuk orang disabilitas termasuk akomodasi untuk kebutaan dan penglihatan rendah, ketulian dan gangguan pendengaran, gerakan terbatas, photosensivitas, atau kombinasinya, dan beberapa akomomodasi untuk kesulitan belajar dan keterbatasan kognitif; tetapi tidak akan memenuhi setiap kebutuhan pengguna dengan disabilitas. Pedoman ini mencakup aksesibilitas konten \textit{web} di desktop, laptop, tablet, dan perangkat bergerak. Dengan mengikuti pedoman ini juga akan sering membuat konten \textit{web} lebih bermanfaat bagi pengguna secara umum. Kriteria Sukses WCAG 2.1 ditulis sebagai pernyataan yang dapat diuji yang tidak teknologi spesifik.

Ada beberapa kondisi yang harus dipenuhi untuk sebuah Kriteria Sukses yaitu :
\begin{enumerate}
	\item Semua Kriteria Sukses harus menjadi masalah akses penting bagi orang disabilitas yang mengatasi masalah di luar masalah kegunaan yang dihadapi oleh semua pengguna. Dengan kata lain, masalah akses harus menyebabkan masalah yang lebih besar bagi orang disabilitas daripada orang yang tidak disabilitas agar dianggap sebagai masalah aksesibilitas.
	\item Semua Kriteria Sukses harus dapat diuji. Hal ini penting karena jika tidak, maka tidak mungkin untuk menentukan apakah suatu halaman memenuhi Kriteria Sukses. Kriteria Sukses dapat diuji dengan kombinasi evaluasi mesin dan manusia selama pengujian dapat menentukan apakah sebuah Kriteria Sukses terpenuhi dengan tingkat kepercayaan yang tinggi.
\end{enumerate}

Kriteria Sukses memiliki tiga tingkat kesesuaian yaitu \textit{Level} A, AA, AAA. Ada beberapa faktor yang menentukan tingat kesesuaian. Faktor tersebut termasuk :
\begin{enumerate}
	\item Apakah Kriteria Sukses esensiil (dalam kata lain, jika Kriteria Sukses tidak terpenuhi maka teknologi bantuan juga tidak dapat membuat konten dapat diakses).
	\item Apakah mungkin untuk memenuhi Kriteria Sukses untuk semua situs \textit{Web} dan jenis konten yang akan diterapkan Kriteria Sukses.
	\item Apakah Kriteria Sukses membutuhkan keterampilan yang dapat dicapai secara wajar oleh pembuat konten (Pengetahuan dan keterampilan untuk memenuhi Kriteria Sukses dapat diperoleh dalam pelatihan seminggu atau kurang).
	\item Apakah Kriteria Sukses dapat memaksakan batasan tampilan dan fungsi dari halaman \textit{Web} (batasan dari fungsi, presentasi, kebebasan berekspresi, desain atau estetika)
	\item Apakah tidak ada solusi jika Kriteria Sukses tidak terpenuhi
\end{enumerate}

\subsection{Kriteria Sukses 1.1.1 Non-text Content}
\label{sec:kriteria_1.1.1}
Semua konten yang bukan text harus dipersembahkan kepada pengguna memiliki text alternatif yang tujuannya sama. Kecuali pada kondisi sebagai berikut :
\begin{itemize}
	\item content...
\end{itemize} 
Level A.

\subsection{Kriteria Sukses 1.2.1 Audio-only dan Video-only (Prerecorded)}
\label{sec:kriteria_1.2.1}
- kurang ngerti
Level A.

\subsection{Kriteria Sukses 1.2.2 Captions (Prerecorded)}
\label{sec:kriteria_1.2.2}
Caption disediakan untuk semua konten \textit{prerecorded audio} di \textit{synchronize media} kecuali medianya adalah media alternatif untuk text dan diberi label dengan jelas seperti itu.
Level A.

\subsection{Kriteria Sukses 1.2.3 Audio Descriptive atau Media Alternative (Prerecorded)}
\label{sec:kriteria_1.2.3}
Alternatif untuk \textit{time-based media} atau deskripsi audio dari konten \textit{prerecorded} video disediakan untuk \textit{synchronize media} kecuali medianya adalah media alternatif untuk text dan diberi label dengan jelas seperti itu.
Level A.

\subsection{Kriteria Sukses 1.2.4 Captions (Live)}
\label{sec:kriteria_1.2.4}
Caption disediakan untuk semua konten \textit{live audio} di \textit{synchronize media}.
Level AA.

\subsection{Kriteria Sukses 1.2.5 Audio Description (Prerecorded)}
\label{sec:kriteria_1.2.5}
Deskripsi audio disediakan untuk semua konten \textit{prerecorded video} di \textit{synchronize media}.
Level AA.

\subsection{Kriteria Sukses 1.2.6 Sign Language (Prerecorded)}
\label{sec:kriteria_1.2.6}
Intepretasi bahasa isyarat disediakan untuk semua konten \textit{prerecorded audio} di \textit{synchronize media}.
Level AAA.

\subsection{Kriteria Sukses 1.2.7 Extended Audio Description (Prerecorded)}
\label{sec:kriteria_1.2.7}
- kurang ngerti
Level AAA.

\subsection{Kriteria Sukses 1.2.8 Media Alternative (Prerecorded)}
\label{sec:kriteria_1.2.8}
Alternatif untuk media \textit{time-based} disediakan untuk semua \textit{prerecorded synchronize media} dan untuk semua \textit{prerecorded video-only media}
Level AAA.

\subsection{Kriteria Sukses 1.2.9 Audio-only (Live)}
\label{sec:kriteria_1.2.9}
Menyediakan alternatif untuk \textit{time-based media} yang menyediakan informasi yang setara untuk konten \textit{live audio-only}.
Level AAA.

\subsection{Kriteria Sukses 1.3.1 Info dan Relationships}
\label{sec:kriteria_1.3.1}
Informasi , struktur, dan hubungan yang ditampilkan melalui presentasi dapat ditentukan secara program atau tersedia dalam text.
Level A.

\subsection{Kriteria Sukses 1.3.2 Meaningful Sequence}
\label{sec:kriteria_1.3.2}
Ketika urutan di mana konten disajikan mempengaruhi maknanya, urutan bacaan yang benar dapat ditentukan secara program.
Level A.

\subsection{Kriteria Sukses 1.3.3 Sensory Characteristics}
\label{sec:kriteria_1.3.3}
Petunjuk yang diberikan untuk memahami dan mengoperasikan konten tidak hanya bergantung pada komponen karakteristik sensorik seperti bentuk, warna, ukuran, lokasi visual, orientasi, atau suara.
Level A.

\subsection{Kriteria Sukses 1.3.4 Orientation}
\label{sec:kriteria_1.3.4}
Tampilan dan pengoperasian konten tidak bergantung pada satu orientasi tampilan , seperti \textit{portrait} atau \textit{landscape}, kecuali jika orientasi tampilan tertentu esensiil.
Level AA.

\subsection{Kriteria Sukses 1.3.5 Identify Input Purpose}
\label{sec:kriteria_1.3.5}
Tujuan untuk setiap \textit{input field} yang digunakan untuk mendapatkan informasi pengguna dapat ditentukan secara program ketika :
\begin{itemize}
	\item content
\end{itemize}
Level AA.

\subsection{Kriteria Sukses 1.3.6 Identify Purpose}
\label{sec:kriteria_1.3.6}
Dalam konten yang diimplementasi dengan \textit{markups languages}, tujuan dari komponen antarmuka, ikon, dan \textit{regions} dapat ditentukan secara program.
Level AAA.

\subsection{Kriteria Sukses 1.4.1 Use of Color}
\label{sec:kriteria_1.4.1}
Level A.

\subsection{Kriteria Sukses 1.4.2 Audio Control}
\label{sec:kriteria_1.4.2}
Level A.

\subsection{Kriteria Sukses 1.4.3 Contrast (Minimum)}
\label{sec:kriteria_1.4.3}
Level AA.

\subsection{Kriteria Sukses 1.4.4 Resize text}
\label{sec:kriteria_1.4.4}
Level AA.

\subsection{Kriteria Sukses 1.4.5 Images of Text}
\label{sec:kriteria_1.4.5}
Level AA.

\subsection{Kriteria Sukses 1.4.6 Contrast (Enhanced)}
\label{sec:kriteria_1.4.6}
Level AAA.

\subsection{Kriteria Sukses 1.4.7 Low atau No Background Audio}
\label{sec:kriteria_1.4.7}
Level AAA.

\subsection{Kriteria Sukses 1.4.8 Visual Presentation}
\label{sec:kriteria_1.4.8}
Level AAA.

\subsection{Kriteria Sukses 1.4.9 Images of Text (No Exception)}
\label{sec:kriteria_1.4.9}
Level AAA.

\subsection{Kriteria Sukses 1.4.10 Reflow}
\label{sec:kriteria_1.4.10}
Level AA.

\subsection{Kriteria Sukses 1.4.11 Non-text Contrast}
\label{sec:kriteria_1.4.11}
Level AA.

\subsection{Kriteria Sukses 1.4.12 Text Spacing}
\label{sec:kriteria_1.4.12}
Level AA.

\subsection{Kriteria Sukses 1.4.13 Content on Hover or Focus}
\label{sec:kriteria_1.4.13}
Level AA.

\subsection{Kriteria Sukses 2.1.1 Keyboard}
\label{sec:kriteria_2.1.1}
Level A.

\subsection{Kriteria Sukses 2.1.2 No Keyboard Trap}
\label{sec:kriteria_2.1.2}
Level A.

\subsection{Kriteria Sukses 2.1.3 Keyboard (No Exception)}
\label{sec:kriteria_2.1.3}
Level AAA.

\subsection{Kriteria Sukses 2.1.4 Character Key Shortcuts}
\label{sec:kriteria_2.1.4}
Level A.

\subsection{Kriteria Sukses 2.2.1 Timing Adjustable}
\label{sec:kriteria_2.2.1}
Level A.

\subsection{Kriteria Sukses 2.2.2 Pause, Stop, Hide}
\label{sec:kriteria_2.2.2}
Level A.

\subsection{Kriteria Sukses 2.2.3 No Timing}
\label{sec:kriteria_2.2.3}
Level AAA.

\subsection{Kriteria Sukses 2.2.4 Interruptions}
\label{sec:kriteria_2.2.4}
Level AAA.

\subsection{Kriteria Sukses 2.2.5 Re-authenticating}
\label{sec:kriteria_2.2.5}
Level AAA.

\subsection{Kriteria Sukses 2.2.6 Timeouts}
\label{sec:kriteria_2.2.6}
Level AAA.

\subsection{Kriteria Sukses 2.3.1 Three Flashes or Below Threshold}
\label{sec:kriteria_2.3.1}
Level A.

\subsection{Kriteria Sukses 2.3.2 Three Flashes}
\label{sec:kriteria_2.3.2}
Level AAA.

\subsection{Kriteria Sukses 2.3.3 Animation from Interactions}
\label{sec:kriteria_2.3.3}
Level AAA.

\subsection{Kriteria Sukses 2.4.1 Bypass Blocks}
\label{sec:kriteria_2.4.1}
Level A.

\subsection{Kriteria Sukses 2.4.2 Page Titled}
\label{sec:kriteria_2.4.2}
Level A.

\subsection{Kriteria Sukses 2.4.3 Focus Order}
\label{sec:kriteria_2.4.3}
Level A.

\subsection{Kriteria Sukses 2.4.4 Link Purpose (In Context)}
\label{sec:kriteria_2.4.4}
Level A.

\subsection{Kriteria Sukses 2.4.5 Multiple Ways}
\label{sec:kriteria_2.4.5}
Level AA.

\subsection{Kriteria Sukses 2.4.6 Headings and Labels}
\label{sec:kriteria_2.4.6}
Level AA.

\subsection{Kriteria Sukses 2.4.7 Focus Visible}
\label{sec:kriteria_2.4.7}
Level AA.

\subsection{Kriteria Sukses 2.4.8 Location}
\label{sec:kriteria_2.4.8}
Level AAA.

\subsection{Kriteria Sukses 2.4.9 Link Purpose (Link Only)}
\label{sec:kriteria_2.4.9}
Level AAA.

\subsection{Kriteria Sukses 2.4.10 Section Headings}
\label{sec:kriteria_2.4.10}
Level AAA.

\subsection{Kriteria Sukses 2.5.1 Pointer Gestures}
\label{sec:kriteria_2.5.1}
Level A.

\subsection{Kriteria Sukses 2.5.2 Pointer Cancellation}
\label{sec:kriteria_2.5.2}
Level A.

\subsection{Kriteria Sukses 2.5.3 Label in Name}
\label{sec:kriteria_2.5.3}
Level A.

\subsection{Kriteria Sukses 2.5.4 Motion Actuation}
\label{sec:kriteria_2.5.4}
Level A.

\subsection{Kriteria Sukses 2.5.5 Target Size}
\label{sec:kriteria_2.5.5}
Level AAA.

\subsection{Kriteria Sukses 2.5.6 Concurrent Input Mechanisms}
\label{sec:kriteria_2.5.6}
Level AAA.

\subsection{Kriteria Sukses 3.1.1 Language of Page}
\label{sec:kriteria_3.1.1}
Level A.

\subsection{Kriteria Sukses 3.1.2 Language of Parts}
\label{sec:kriteria_3.1.2}
Level AA.

\subsection{Kriteria Sukses 3.1.3 Unusual Words}
\label{sec:kriteria_3.1.3}
Level AAA.

\subsection{Kriteria Sukses 3.1.4 Abbreviations}
\label{sec:kriteria_3.1.4}
Level AAA.

\subsection{Kriteria Sukses 3.1.5 Reading Level}
\label{sec:kriteria_3.1.5}
Level AAA.

\subsection{Kriteria Sukses 3.1.6 Pronunciation}
\label{sec:kriteria_3.1.6}
Level AAA.

\subsection{Kriteria Sukses 3.2.1 On Focus}
\label{sec:kriteria_3.2.1}
Level A.

\subsection{Kriteria Sukses 3.2.2 On Input}
\label{sec:kriteria_3.2.2}
Level A.

\subsection{Kriteria Sukses 3.2.3 Consistent Navigation}
\label{sec:kriteria_3.2.3}
Level AA.

\subsection{Kriteria Sukses 3.2.4 Consistent Identification}
\label{sec:kriteria_3.2.4}
Level AA.

\subsection{Kriteria Sukses 3.2.5 Change on Request}
\label{sec:kriteria_3.2.5}
Level AAA.

\subsection{Kriteria Sukses 3.3.1 Error Identification}
\label{sec:kriteria_3.3.1}
Level A.

\subsection{Kriteria Sukses 3.3.2 Labels or Instructions}
\label{sec:kriteria_3.3.2}
Level A.

\subsection{Kriteria Sukses 3.3.3 Error Suggestion}
\label{sec:kriteria_3.3.3}
Level AA.

\subsection{Kriteria Sukses 3.3.4 Error Prevention (Legal, Financial, Data)}
\label{sec:kriteria_3.3.4}
Level AA.

\subsection{Kriteria Sukses 3.3.5 Help}
\label{sec:kriteria_3.3.5}
Level AAA.

\subsection{Kriteria Sukses 3.3.6 Error Prevention (All)}
\label{sec:kriteria_3.3.6}
Level AAA.

\subsection{Kriteria Sukses 4.1.1 Parsing}
\label{sec:kriteria_4.1.1}
Level A.

\subsection{Kriteria Sukses 4.1.2 Name, Role, Value}
\label{sec:kriteria_4.1.2}
Level A.

\subsection{Kriteria Sukses 4.1.3 Status Messages}
\label{sec:kriteria_4.1.3}
Level AA.

\section{\LaTeX}
\label{sec:latex}

Mengapa menggunakan \LaTeX{} untuk buku skripsi dan apa keunggulan/kerugiannya bagi mahasiswa dan pembuat template. 

\dtext{13-14}


\section{Template Skripsi FTIS UNPAR}
\label{sec:template}
 
Akan dipaparkan bagaimana menggunakan template ini, termasuk petunjuk singkat membuat referensi, gambar dan tabel.
Juga hal-hal lain yang belum terpikir sampai saat ini. 
 
\dtext{15-16}

\subsection{Tabel}  
Berikut adalah contoh pembuatan tabel. 
Penempatan tabel dan gambar secara umum diatur secara otomatis oleh \LaTeX{}, perhatikan contoh di file bab2.tex untuk melihat bagaimana cara memaksa tabel ditempatkan sesuai keinginan kita.

Perhatikan bawa berbeda dengan penempatan judul gambar gambar, keterangan tabel harus diletakkan di atas tabel!!
Lihat Tabel~\ref{tab:contoh1} berikut ini:

\begin{table}[H] %atau h saja untuk "kira kira di sini"
	\centering 
	\caption{Tabel contoh}
	\label{tab:contoh1}
	\begin{tabular}{cccc}
		\toprule
		& $v_{start}$ & $\mathcal{S}_{1}$ & $v_{end}$\\

		\midrule
		$\tau_{1}$ & 1 & 12& 20\\
		$\tau_{2}$ & 1 &  & 20\\
		$\tau_{3}$ & 1 & 9 & 20\\
		$\tau_{4}$ & 1 &  & 20\\

		\bottomrule
		
	\end{tabular} 
\end{table}
Tabel~\ref{tab:cthwarna1} dan Tabel~\ref{tab:cthwarna2} berikut ini adalah tabel dengan sel yang berwarna dan ada dua tabel yang bersebelahan. 
\begin{table}[H]
	\begin{minipage}[c]{0.49\linewidth}
		\centering
		\caption{Tabel bewarna(1)}
		\label{tab:cthwarna1}
		\begin{tabular}{ccccc}
			\toprule
			 & $v_{start}$ & $\mathcal{S}_{2}$ & $\mathcal{S}_{1}$ & $v_{end}$\\
			
			\midrule
			$\tau_{1}$ & 1 & 5 \cellcolor{green}& 12& 20\\
			$\tau_{2}$ & 1 & 8 \cellcolor{green}& & 20\\
			$\tau_{3}$ & 1 & 2/8/17 \cellcolor{green}& 9 & 20\\
			$\tau_{4}$ & 1 & \cellcolor{red}& & 20\\
			
			\bottomrule

		\end{tabular}
	\end{minipage}
	\begin{minipage}[c]{0.49\linewidth}
		
		\centering 
		\caption{Tabel bewarna(2)}
		\label{tab:cthwarna2}
		\begin{tabular}{ccccc}
			\toprule
			 & $v_{start}$ & $\mathcal{S}_{1}$ & $\mathcal{S}_{2}$ & $v_{end}$\\
			
			\midrule
			$\tau_{1}$ & 1 & 12& 5 \cellcolor{red} &20\\
			$\tau_{2}$ & 1 &  &  8 \cellcolor{green} &20\\
			$\tau_{3}$ & 1 & 9 & 2/8/17 \cellcolor{green} &20\\
			$\tau_{4}$ & 1 &   & \cellcolor{red} &20\\
			
			\bottomrule
		
		\end{tabular}
	\end{minipage}
\end{table}

 
\subsection{Kutipan}
\label{subs:kutipan} 
Berikut contoh kutipan dari berbagai sumber, untuk keterangan lebih lengkap, silahkan membaca file referensi.bib yang disediakan juga di template ini.
Contoh kutipan:
\begin{itemize}
	\item Buku:~\cite{berg:08:compgeom} 
	\item Bab dalam buku:~\cite{kreveld:04:GIS}
	\item Artikel dari Jurnal:~\cite{buchin:13:median}
	\item Artikel dari prosiding seminar/konferensi:~\cite{kreveld:11:median}
	\item Skripsi/Thesis/Disertasi:~\cite{lionov:02:animasi}~\cite{wiratma:10:following}~\cite{wiratma:22:later}
	\item Technical/Scientific Report:~\cite{kreveld:07:watertight}
	\item RFC (Request For Comments):~\cite{RFC1654}
	\item Technical Documentation/Technical Manual:~\cite{Z.500}~\cite{unicode:16:stdv9}~\cite{google:16:and7}
	\item Paten:~\cite{webb:12:comm}
	\item Tidak dipublikasikan:~\cite{wiratma:09:median}~\cite{lionov:11:cpoly}
	\item Laman web:~\cite{erickson:03:cgmodel}  
	\item Lain-lain:~\cite{agung:12:tango}
\end{itemize}    
  
\subsection{Gambar}

Pada hampir semua editor, penempatan gambar di dalam dokumen \LaTeX{} tidak dapat dilakukan melalui proses {\it drag and drop}.
Perhatikan contoh pada file bab2.tex untuk melihat bagaimana cara menempatkan gambar.
Beberapa hal yang harus diperhatikan pada saat menempatkan gambar:
\begin{itemize}
	\item Setiap gambar {\bf harus} diacu di dalam teks (gunakan {\it field} {\sc label})
	\item {\it Field} {\sc caption} digunakan untuk teks pengantar pada gambar. Terdapat dua bagian yaitu yang ada di antara tanda $[$ dan $]$ dan yang ada di antara tanda $\{$ dan $\}$. Yang pertama akan muncul di Daftar Gambar, sedangkan yang kedua akan muncul di teks pengantar gambar. Untuk skripsi ini, samakan isi keduanya.
	\item Jenis file yang dapat digunakan sebagai gambar cukup banyak, tetapi yang paling populer adalah tipe {\sc png} (lihat Gambar~\ref{fig:ularpng}), tipe {\sc jpg} (Gambar~\ref{fig:ularjpg}) dan tipe {\sc pdf} (Gambar~\ref{fig:ularpdf})
	\item Besarnya gambar dapat diatur dengan {\it field} {\sc scale}.
	\item Penempatan gambar diatur menggunakan {\it placement specifier} (di antara tanda  $[$ dan $]$ setelah deklarasi gambar.
	Yang umum digunakan adalah {\bf H} untuk menempatkan gambar {\bf sesuai} penempatannya di file .tex atau  {\bf h} yang berarti "kira-kira" di sini. \\
	Jika tidak menggunakan {\it placement specifier}, \LaTeX{} akan menempatkan gambar secara otomatis untuk menghindari bagian kosong pada dokumen anda.
	Walaupun cara ini sangat mudah, hindarkan terjadinya penempatan dua gambar secara berurutan. 	
	\begin{itemize}
		\item Gambar~\ref{fig:ularpng} ditempatkan di bagian atas halaman, walaupun penempatannya dilakukan setelah penulisan 3 paragraf setelah penjelasan ini.
		\item Gambar~\ref{fig:ularjpg} dengan skala 0.5 ditempatkan di antara dua buah paragraf. Perhatikan penulisannya di dalam file bab2.tex!
		\item Gambar~\ref{fig:ularpdf} ditempatkan menggunakan {\it specifier} {\bf h}.
	\end{itemize}
\end{itemize}
 
\dtext{17-18}
\begin{figure} 
	\centering  
	\includegraphics[scale=1]{ular-png}  
	\caption[Gambar {\it Serpentes} dalam format png]{Gambar {\it Serpentes} dalam format png} 
	\label{fig:ularpng} 
\end{figure} 

\dtext{19-20}
\begin{figure}[H]
	\centering  
	\includegraphics[scale=0.5]{ular-jpg}  
	\caption[Ular kecil]{Ular kecil} 
	\label{fig:ularjpg} 
\end{figure} 
\dtext{21-22}

\begin{figure}[ht] 
	\centering  
	\includegraphics[scale=1]{ular-pdf}  
	\caption[ {\it Serpentes} betina]{ {\it Serpentes} jantan} 
	\label{fig:ularpdf} 
\end{figure} 
 
